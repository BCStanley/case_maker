
\documentclass[twoside]{article}

\usepackage{verbatim}
\usepackage{longtable}
\usepackage{hyperref}
\hypersetup{
    colorlinks=false,
    linkcolor=blue,
    filecolor=magenta,
    urlcolor=cyan,
    bookmarks=true,
    hidelinks=true,
}

% 1. Sets the font
\usepackage{fontspec}
\setmainfont{Charter}
\renewcommand{\baselinestretch}{1.25}

%2. Sets the page size, etc.

\usepackage{geometry}
\geometry{
a4paper,
left = 20mm,
right = 20mm,
top = 20mm,
bottom = 20mm,
}

\usepackage{titlesec}
\titleformat{\section}
  {\normalfont\normalsize\bfseries\scshape}{\thesection}{1em}{}
\titleformat{\subsection}
  {\normalfont\normalsize\itshape}{\thesubsection}{1em}{}
\titleformat{\subsubsection}
  {\normalfont\normalsize}{\thesubsubsection}{1em}{}

\def \Title{Table of Conflict of Laws Cases}
\def \Author{Benedict Stanley}

\title{\Title}
\author{\Author}
\date{\today}

\usepackage{fancyhdr}
\pagestyle{fancy}
\fancyhf{}
\fancyhead{}
\fancyhead[L]{}
\fancyhead[R]{}
\fancyhead[C]{}

\fancyfoot[L]{\Title}
\fancyfoot[R]{\thepage}
\renewcommand{\headrulewidth}{0pt}
\renewcommand{\footrulewidth}{0pt}


\usepackage{tocloft}

\addtolength{\cftsecnumwidth}{2pt}
\addtolength{\cftsubsecnumwidth}{5pt}
\addtolength{\cftsubsubsecnumwidth}{9pt}

\begin{document}

\maketitle

\tableofcontents
\section{Contract: Resolution in Favour of Place of Contract's Inception}

These are all of the cases I have seen which either (1) seem to resolve issues themselves with reference to the law of the place where the contractual obligation originated, \textit{or} (2) are cited by others as such. It should be noted that very few of these cases actually make use of the wording “\textit{lex loci contractus}” within them.
\\ 
\begin{enumerate}
\item{\textit{Dungannon v Hackett} (1702) Eq Ca Abr 289, 23 ER 855}
\item{\textit{Foubert v Turst} (1703) 1 Brown Parl Cas,, 1 ER 464,  24 ER 101}
\item{\textit{Ekins v East-India Co} (1717) 1 P Wms 394, 24 ER 441}
\item{\textit{Tremoult v Dedire} (1718) 1 P Wms 429, 24 ER 458}
\item{\textit{Phipps v Earl of Angelsea} (1721) 1 P Wms 697, 5 Bro PC 45, 24 ER 576}
\item{\textit{Saunders v Drake} (1742) 2 Atk 465, 26 ER 681}
\item{\textit{Ballantine v Golding} (1784) Cooke’s Bankrupt Laws 419}
\item{\textit{Bodham v Ryley} (1787) 4 Brown 561, 2 ER 382}
\item{\textit{Melan v Fitzjames} (1792) 1 Bos \& Pul 138, 126 ER 822}
\item{\textit{Smith v Buchanan} (1800) 1 East 6, 102 ER 3}
\item{\textit{Innes v Dunlop} (1800) 8 TR 595, 101 ER 1565}
\item{\textit{Potter v Brown} (1804) 5 East 124, 102 ER 1016}
\item{\textit{O Callaghan v Thomond} (1810) 3 Taunt 82, 128 ER 33}
\item{\textit{Snaith v Mingay} (1813) 1 M\&S 87, 105 ER 33}
\item{\textit{Power v Whitmore} (1815) 4 M\&S 141, 105 ER 787}
\item{\textit{Jeffery v McTaggart} (1817) 6 M\&S 126, 105 ER 1190}
\item{\textit{Arnott v Redfern} (1825) 2 Car \& P 88, 172 ER 40}
\item{\textit{Pattison v Mills} (1828) 1 Dow \& Clark 342, 6 ER 553}
\item{\textit{Phillips v Allan} (1828) 7 B\&C 477, 108 ER 1120}
\item{\textit{De La Chaumette v Bank of England} (1829) 9 B\&C 208, 109 ER 78}
\end{enumerate}
\section{Chronological List of All Cases}

The following is a simple table containing all of the cases, sorted by date. Each entry includes all the relevant obtained information.
\\ 


        \begin{small}
        \begin{center}
        \href{https://heinonline.org/HOL/P?h=hein.engrep/engrg0123&i=789}{\textit{Anon} (1611) 2 B\&G 10, 123 ER 785} \label{5} \\ 
\textit{ ()}\\
        \end{center}
        \textbf{Admiralty}. Showing a fraught understanding of the jurisdiction of Admiralty for contracts “beyond the sea” (referring to France) but not on the deep sea. See Sack (1937)\\\\\textit{Authors refered to: }anewauthor.
        \end{small}\\
        \rule{\textwidth}{0.5pt}
        

        \begin{small}
        \begin{center}
        \href{https://heinonline.org/HOL/P?h=hein.engrep/engrg0124&i=24}{\textit{Van Heath v Turner} (1621) Winch 23, 124 ER 20} \label{3} \\ 
\textit{Law Merchant (Bill of Exchange)}\\
        \end{center}
        \textbf{CP}. Report is in French. It is discussed in Sack (1937)\\\\No known authors cited.
        \end{small}\\
        \rule{\textwidth}{0.5pt}
        

        \begin{small}
        \begin{center}
        \href{https://heinonline.org/HOL/P?h=hein.engrep/engrf0081&i=937}{\textit{Slane \& Colbery v Ralph Cotton} (1625) 2 Rolle 486, 81 ER 933} \label{6} \\ 
\textit{Admiralty Jurisdiction (Contract---Carriage)}\\
        \end{center}
        \textbf{Admiralty}. Report is in French. It is discussed in Sack (1937)\\\\No known authors cited.
        \end{small}\\
        \rule{\textwidth}{0.5pt}
        

        \begin{small}
        \begin{center}
        \href{https://heinonline.org/HOL/P?h=hein.engrep/engrf0083&i=343}{\textit{Beven v Clapham1664} (1664) Lev 143, 83 ER 339} \label{15} \\ 
\textit{ (Assumpsit---Contract)}\\
        \end{center}
        \textbf{KB}. Claim clearly pleaded fictionally “Tenerif, in the Ward of Cheap”, but the statute of limitations said not to extent to it.\\\\No known authors cited.
        \end{small}\\
        \rule{\textwidth}{0.5pt}
        

        \begin{small}
        \begin{center}
        \href{https://heinonline.org/HOL/P?h=hein.engrep/engrc0036&i=649}{\textit{Cottington’s Case} (1678) 2 Swans 326, 36 ER 640} \label{8} \\ 
\textit{Foreign Judgements (Marriage Divorce)}\\
        \end{center}
        \textbf{HL}. The clear idea is presented that foreign judgement (related to nullity of marriage) from Turin cannot be examined by an English court.\\\\No known authors cited.
        \end{small}\\
        \rule{\textwidth}{0.5pt}
        

        \begin{small}
        \begin{center}
        \href{https://heinonline.org/HOL/P?h=hein.engrep/engrc0036&i=649}{\textit{Gold v Canahan} (1679) 2 Swans 326, 36 ER 640} \label{7} \\ 
\textit{ (Bill of Exchange---Partnership)}\\
        \end{center}
        \textbf{HL}. A very brief report. The interesting detail is that the “justice” of the Florentine judgement “is not examinable here.” There appears to be some attempt to indemnify. No clear discussion of “choice of law.”\\\\No known authors cited.
        \end{small}\\
        \rule{\textwidth}{0.5pt}
        

        \begin{small}
        \begin{center}
        \href{https://heinonline.org/HOL/P?h=hein.engrep/engrf0089&i=601}{\textit{Magadra v Holt} (1691) 1 Show 318, 89 ER 597} \label{1} \\ 
\textit{ (Bill of Exchange)}\\
        \end{center}
        \textbf{KB}.  \textbf{Uses terms: }[\textit{ius gentium}]. An early instance of simple application of the “law merchant.”\\\\No known authors cited.
        \end{small}\\
        \rule{\textwidth}{0.5pt}
        

        \begin{small}
        \begin{center}
        \href{https://heinonline.org/HOL/P?h=hein.engrep/engrf0090&i=763}{\textit{Williams v Williams} (1693) Carth 268, 90 ER 759} \label{4} \\ 
\textit{Law Merchant (Bill of Exchange)}\\
        \end{center}
        \textbf{KB}. Noteworthy for the way it treats law merchant as part of the law of England. Can still see use of fiction “Mariae de Arcubus in Warda de Cheap.”\\\\No known authors cited.
        \end{small}\\
        \rule{\textwidth}{0.5pt}
        

        \begin{small}
        \begin{center}
        \href{https://heinonline.org/HOL/P?h=hein.engrep/engrf0091&i=361}{\textit{Blankard v Goldy} (1693) 2 Salk 411, 91 ER 356} \label{16} \\ 
\textit{Application of English Statute (Contract---Illegality)}\\
        \end{center}
        \textbf{KB}. Seeming to see Jamaica, as an “uninhabited country” as taking on the law of England, but not a particular statute forbidding purchasing of offices.\\\\No known authors cited.
        \end{small}\\
        \rule{\textwidth}{0.5pt}
        

        \begin{small}
        \begin{center}
        \href{https://heinonline.org/HOL/P?h=hein.engrep/engrg0125&i=874}{\textit{Bromwhich v Loyd} (1699) 2 Lutw 1582, 125 ER 870} \label{2} \\ 
\textit{Law Merchant (Bill of Exchange)}\\
        \end{center}
        \textbf{KB}. Another instance of application of the “law merchant.”\\\\No known authors cited.
        \end{small}\\
        \rule{\textwidth}{0.5pt}
        

        \begin{small}
        \begin{center}
        \href{https://heinonline.org/HOL/P?h=hein.engrep/engrc0023&i=863}{\textit{Ranelaugh v Champante} (1700) 2 Vern 395, 23 ER 855} \label{17} \\ 
\textit{Contract---Application of English Statute (Bond---Real Estate)}\\
        \end{center}
        \textbf{Ch}. Bond for a debt “in Ireland” executed in England, leading to the application of English interest. There is a hint of a lex fori rule.\\\\No known authors cited.
        \end{small}\\
        \rule{\textwidth}{0.5pt}
        

        \begin{small}
        \begin{center}
        \href{https://heinonline.org/HOL/P?h=hein.engrep/engrc0023&i=863}{\textit{Dungannon v Hackett} (1702) Eq Ca Abr 289, 23 ER 855} \label{18} \\ 
\textit{Contract Interest---Contract---Application of Lex Loci Con (Debt---Contract)}\\
        \end{center}
        \textbf{Ch}. Implication is that the interest should be determined by the place where it was contracted for.\\\\No known authors cited.
        \end{small}\\
        \rule{\textwidth}{0.5pt}
        

        \begin{small}
        \begin{center}
        \href{https://heinonline.org/HOL/P?h=hein.engrep/engra0001&i=472}{\textit{Foubert v Turst} (1703) 1 Brown Parl Cas,, 1 ER 464,  24 ER 101} \label{27} \\ 
\textit{Contract---Contract Intention---Lex Loci Con (Marriage Contract)}\\
        \end{center}
        \textbf{HL}. French Marriage Contract affirmed, by its express terms, to refer to the custom of Paris. Actual reasoning unclear, but it seems to be based on ideas of intent.\\\\No known authors cited.
        \end{small}\\
        \rule{\textwidth}{0.5pt}
        

        \begin{small}
        \begin{center}
        \href{https://heinonline.org/HOL/P?h=hein.engrep/engrf0087&i=952}{\textit{Wey v Rally} (1705) 2 Salk 651 6 Mod 194, 87 ER 948} \label{14} \\ 
\textit{Jurisdiction Land (Rent on Lands)}\\
        \end{center}
        \textbf{KB}.  \textbf{Uses terms: }[\textit{privity of estate, privity of contract}]. Claims for rents of lands in Jamaica was a transitory and not local action.\\\textit{Cited in: }Westlake Ed1 (At arts 120-122, in connection to “local” and “transitory” actions and the rules of jurisdiction)\\No known authors cited.
        \end{small}\\
        \rule{\textwidth}{0.5pt}
        

        \begin{small}
        \begin{center}
        \href{https://heinonline.org/HOL/P?h=hein.engrep/engrf0091&i=570}{\textit{Smith v Brown \& Cooper} (1706) 2 Salk 665, 91 ER 566} \label{29} \\ 
\textit{Contract---Illegality (Sale---Slaves)}\\
        \end{center}
        \textbf{KB}. A very short report. The issue appears to have been the allegation that slaves were sold in London (where no notice could be taken of them), and that the plea should have been that the contract was in London but the slave in Virginia.\\\textit{Cited in: }Westlake Ed1 (arts 192-200, somewhat critically, in connection to non application of morally repugnant laws for the consideration of contracts.)\\No known authors cited.
        \end{small}\\
        \rule{\textwidth}{0.5pt}
        

        \begin{small}
        \begin{center}
        \href{https://heinonline.org/HOL/P?h=hein.engrep/engrc0024&i=449}{\textit{Ekins v East-India Co} (1717) 1 P Wms 394, 24 ER 441} \label{19} \\ 
\textit{Contract Interest---Contract---Application of Lex Loci Con (Tresspass---Goods Taken)}\\
        \end{center}
        \textbf{Ch}. Action of taking and selling goods in India carried Indian interest. Not be a choice of law rule “must be presumed to have common advantage” of money there.\\\textit{Cited in: }Westlake Ed1 (arts 230-236 for breach of obligations)\\No known authors cited.
        \end{small}\\
        \rule{\textwidth}{0.5pt}
        

        \begin{small}
        \begin{center}
        \href{https://heinonline.org/HOL/P?h=hein.engrep/engrc0024&i=466}{\textit{Tremoult v Dedire} (1718) 1 P Wms 429, 24 ER 458} \label{28} \\ 
\textit{Contract---Contract Law of Terms---Marriage---Contract Intention---Lex Loci Con (Marriage Contract)}\\
        \end{center}
        \textbf{Ch}. Clear implication that Dutch marriage articles could be construed and applied according to the laws of Holland. Evidence of this is required “to take notice of foreign laws” (contrast with Foubert). Unclear what the basis of this is, though it seems assumed.\\\\No known authors cited.
        \end{small}\\
        \rule{\textwidth}{0.5pt}
        

        \begin{small}
        \begin{center}
        \href{https://heinonline.org/HOL/P?h=hein.engrep/engrc0024&i=584}{\textit{Phipps v Earl of Angelsea} (1721) 1 P Wms 697, 5 Bro PC 45, 24 ER 576} \label{31} \\ 
\textit{Contract---Lex Loci Con (Marriage Settlement)}\\
        \end{center}
        \textbf{Ch}. English interest, as this was the place where the contract was made (and where it was to be performed.) Very little specific reasoning on the issue.\\\textit{Cited in: }Story Ed1 (Cited at §279-290 to say that the lex loci rule is not circumvented by the location of the security)\\No known authors cited.
        \end{small}\\
        \rule{\textwidth}{0.5pt}
        

        \begin{small}
        \begin{center}
        \href{https://heinonline.org/HOL/P?h=hein.engrep/engrc0024&i=660}{\textit{Wallis v Brightwell} (1722) 2 P Wms 87, 24 ER 652} \label{23} \\ 
\textit{Contract (Wills)}\\
        \end{center}
        \textbf{Ch}. An annuity (paid and made in England) out of lands in Ireland. “English money” owed, intention the guiding factor (looking at place of contracting and performance.)\\\\No known authors cited.
        \end{small}\\
        \rule{\textwidth}{0.5pt}
        

        \begin{small}
        \begin{center}
        \href{https://heinonline.org/HOL/P?h=hein.engrep/engrc0025&i=243, https://heinonline.org/HOL/P?h=hein.engrep/engrf0093&i=819}{\textit{Burroughs v Jamineau} (1726) 2 Str 733, 25 ER 235 93 ER 815} \label{10} \\ 
\textit{Contract---Foreign Judgements (Bill of Exchange)}\\
        \end{center}
        \textbf{Ch}. A bill of exchange that was discharged by the law of Livorno could not be sued for in England. Clearly of the view that it had to be determined by the place where the bill was negotiated. Injunction granted to prevent suing on the bill.\\\textit{Cited in: }Westlake Ed1 (with apparent approval, arts 225-228 in connection to international law of obligations); Story Ed1 (At §263-266, for the substantive requirements of the contract and the lex loci contractus.)\\No known authors cited.
        \end{small}\\
        \rule{\textwidth}{0.5pt}
        

        \begin{small}
        \begin{center}
        \href{https://heinonline.org/HOL/P?h=hein.engrep/engrc0026&i=639}{\textit{Connor v Bellamont} (1742) 2 Atk 382, 26 ER 631} \label{20} \\ 
\textit{Contract Interest---Contract (Bond---Real Estate)}\\
        \end{center}
        \textbf{Ch}. Debt contracted for in England, but bond taken out for its enforcement in Ireland – leading to Irish interest being applied. Appears to be a circumstantial test, place of security not sufficient – but currency and other factors enough.\\\textit{Cited in: }Story Ed1 (§291-298, on the rules for interest to suggest performance of place of contract unless performance was due elsewhere)\\No known authors cited.
        \end{small}\\
        \rule{\textwidth}{0.5pt}
        

        \begin{small}
        \begin{center}
        \href{https://heinonline.org/HOL/P?h=hein.engrep/engrc0026&i=689}{\textit{Saunders v Drake} (1742) 2 Atk 465, 26 ER 681} \label{24} \\ 
\textit{Contract Currency---Contract---Application of Lex Loci Con---Contract Intention (Wills)}\\
        \end{center}
        \textbf{Ch}. “Jamaican Money” applied to testators estate, with strong reliance on intention of the parties.\\\\No known authors cited.
        \end{small}\\
        \rule{\textwidth}{0.5pt}
        

        \begin{small}
        \begin{center}
        \href{https://heinonline.org/HOL/P?h=hein.engrep/engrc0027&i=1130}{\textit{Stapleton v Conway} (1750) 1 Ves Sen 427, 27 ER 1122} \label{25} \\ 
\textit{Contract Interest---Contract---Application of English Statute---Application of Lex Fori (Wills)}\\
        \end{center}
        \textbf{Ch}. Interest on charge of lands in Nevis. West Indian interest refused, with strong reliance of the potential for avoidance of usuary laws.  This is said to rely on a kind of “discretion,” as opposed to where a contract was made in England or America. (Hinting at a choice of law idea.)\\\textit{Cited in: }Story Ed1 (§291-298, on the rules for interest to suggest performance of place of contract unless performance was due elsewhere)\\No known authors cited.
        \end{small}\\
        \rule{\textwidth}{0.5pt}
        

        \begin{small}
        \begin{center}
        \href{https://heinonline.org/HOL/P?h=hein.engrep/engrf0098&i=1025}{\textit{Mostyn v Fabrigas} (1774) 1 Cowp 161, 98 ER 1021} \label{12} \\ 
\textit{Local and Transitory Actions (Tresspass)}\\
        \end{center}
        \textbf{KB}.  \textbf{Uses terms: }[\textit{lex loci}]. Allowing an action by a Minorcan for wrongs done in Minorca. Quite substantive reasoning with Lord Mansfield, dealing with the role of legal fictions and forms.\\\textit{Cited in: }Westlake Ed1 (At arts 120-122, in connection to “local” and “transitory” actions and the rules of jurisdiction)\\No known authors cited.
        \end{small}\\
        \rule{\textwidth}{0.5pt}
        

        \begin{small}
        \begin{center}
        \href{https://heinonline.org/HOL/P?h=hein.engrep/engrf0096&i=632}{\textit{Rafael v Verlest} (1776) 2 Black W 1055, 96 ER 628} \label{11} \\ 
\textit{ (Tresspass)}\\
        \end{center}
        \textbf{KB}. Very complex and hard decision to parse. There is some reference to “acts of princes” meaning there is no jurisdiction to try the action.\\\textit{Cited in: }Westlake Ed1 (At arts 120-122, in connection to “local” and “transitory” actions and the rules of jurisdiction)\\No known authors cited.
        \end{small}\\
        \rule{\textwidth}{0.5pt}
        

        \begin{small}
        \begin{center}
        \href{https://link.gale.com/apps/doc/CW0125544801/ECCO?u=oxford&sid=gale_marc&xid=19b67222&pg=534}{\textit{Ballantine v Golding} (1784) Cooke’s Bankrupt Laws 419} \label{39} \\ 
\textit{Contract---Contract Discharge---Bankruptcy---Application of Lex Loci Con (Debt---Contract)}\\
        \end{center}
        \textbf{KB}. Lord Mansfield giving the rule of a discharge of debts of a bankrupt being effective where the the debts arose there (though the actual rule might be wider.)\\\textit{Cited in: }Story Ed1 (§330-351, with approval); Westlake Ed1 (arts 235-256, with approval.)\\No known authors cited.
        \end{small}\\
        \rule{\textwidth}{0.5pt}
        

        \begin{small}
        \begin{center}
        \href{https://heinonline.org/HOL/P?h=hein.engrep/engra0002&i=390}{\textit{Bodham v Ryley} (1787) 4 Brown 561, 2 ER 382} \label{21} \\ 
\textit{Contract Interest---Contract---Application of Lex Loci Con (Partnership---Debts)}\\
        \end{center}
        \textbf{HL}. Report references some wide propositions, including a note on Huber and a general lex loci contractus principle. The actual grounds on which Indian interest was allowed seem less clear, and more related to presumed custom and intent of the parties.\\\\No known authors cited.
        \end{small}\\
        \rule{\textwidth}{0.5pt}
        

        \begin{small}
        \begin{center}
        \href{https://heinonline.org/HOL/P?h=hein.engrep/engrf0100&i=660}{\textit{Dewar v Span} (1789) 3 TR 425, 100 ER 656} \label{26} \\ 
\textit{Contract Interest---Contract---Application of English Statute---Application of Lex Fori (Contract---Vendor Purchaser)}\\
        \end{center}
        \textbf{KB}. Case entirely concerns the application of usury statutes to the West Indes.\\\\No known authors cited.
        \end{small}\\
        \rule{\textwidth}{0.5pt}
        

        \begin{small}
        \begin{center}
        \href{https://heinonline.org/HOL/P?h=hein.engrep/engrf0100&i=1147}{\textit{Doulson v Matthews} (1792) 4 TR 503, 100 ER 1143} \label{13} \\ 
\textit{Local and Transitory Actions (Tresspass)}\\
        \end{center}
        \textbf{KB}. An action could not lie for entering a house in Canada.\\\textit{Cited in: }Westlake Ed1 (At arts 120-122, in connection to “local” and “transitory” actions and the rules of jurisdiction)\\No known authors cited.
        \end{small}\\
        \rule{\textwidth}{0.5pt}
        

        \begin{small}
        \begin{center}
        \href{https://heinonline.org/HOL/P?h=hein.engrep/engrg0126&i=826}{\textit{Melan v Fitzjames} (1792) 1 Bos \& Pul 138, 126 ER 822} \label{30} \\ 
\textit{Contract---Application of Lex Loci Con (Bond---Real Estate)}\\
        \end{center}
        \textbf{CP}. Demonstrates a very clear lex loci contractus understanding, the Chief Justice clearly reasoning on such a line. There is a division in opinion. There is also some hint of the idea that reference to another law could be relevant.\\\textit{Cited in: }Story Ed1 (Cited with approval at §263-266, as to the applicable law for determining the nature of a contract, and the locus regit actum principle)\\No known authors cited.
        \end{small}\\
        \rule{\textwidth}{0.5pt}
        

        \begin{small}
        \begin{center}
        \href{https://heinonline.org/HOL/P?h=hein.engrep/engrk0170&i=578}{\textit{Male v Roberts} (1800) 3 Esp 163, 170 ER 574} \label{9} \\ 
\textit{ (Assumpsit Debt---Contract---Infancy)}\\
        \end{center}
        \textbf{CP}. Lord Eldon gives a clear understanding of the contract being determined by the laws of Scotland, based on the idea that it “arose” there.\\\textit{Cited in: }Story Ed1 (with approval, for rules on validity §241-249 and §330-351 for non-exclusive application of lex loci contractus for discharge of obligations)\\No known authors cited.
        \end{small}\\
        \rule{\textwidth}{0.5pt}
        

        \begin{small}
        \begin{center}
        \href{https://heinonline.org/HOL/P?h=hein.engrep/engrf0102&i=7}{\textit{Smith v Buchanan} (1800) 1 East 6, 102 ER 3} \label{41} \\ 
\textit{Contract---Contract Discharge---Bankruptcy---Application of Lex Loci Con (Bankruptcy---Sale of Goods)}\\
        \end{center}
        \textbf{KB}. Maryland discharge by bankruptcy held to not affect debts contracted for in England, with a strong preference for lex loci contractus “It is impossible to say that a contract made in one country is to be governed by the laws of another.” Ballatine v Golding distinguished.\\\textit{Cited in: }Story Ed1 (§330-351, with approval); Westlake Ed1 (arts 235-256, with approval.)\\No known authors cited.
        \end{small}\\
        \rule{\textwidth}{0.5pt}
        

        \begin{small}
        \begin{center}
        \href{https://heinonline.org/HOL/P?h=hein.engrep/engrf0101&i=1569}{\textit{Innes v Dunlop} (1800) 8 TR 595, 101 ER 1565} \label{46} \\ 
\textit{Contract---Application of Lex Loci Con---Contract Assignment (Debt---Contract)}\\
        \end{center}
        \textbf{KB}. Assignee of a Scottish bond allowed to sue in his own name in England. Assignments valid under Scots, but not English law. Unclear where assignment took place, though bond was clearly Scottish.\\\textit{Cited in: }Westlake Ed1 (arts 241-245, for the assignability of debts being judged by the point of their inception.)\\No known authors cited.
        \end{small}\\
        \rule{\textwidth}{0.5pt}
        

        \begin{small}
        \begin{center}
        \href{https://heinonline.org/HOL/P?h=hein.engrep/engrc0032&i=880}{\textit{Bourke v Rickets} (1804) 10 Ves Jun 330, 32 ER 872} \label{22} \\ 
\textit{Contract Interest---Contract---Application of Lex Fori (Wills)}\\
        \end{center}
        \textbf{Ch}. A rather confusing report. The relevant legacies appear to have been executed in both Jamaica and England, and English interest was applied. There is said to not be a general rule, and the circumstances of the case are relied on (including that it was sued for in England.)\\\\No known authors cited.
        \end{small}\\
        \rule{\textwidth}{0.5pt}
        

        \begin{small}
        \begin{center}
        \href{https://heinonline.org/HOL/P?h=hein.engrep/engrf0102&i=1020}{\textit{Potter v Brown} (1804) 5 East 124, 102 ER 1016} \label{38} \\ 
\textit{Contract---Contract Discharge---Application of Lex Loci Con (Bill of Exchange)}\\
        \end{center}
        \textbf{KB}. A bill drawn in America on someone in England, was discharged by American certificate of bankruptcy, thereby discharging the defendant in the event of dishonour in England.\\\textit{Cited in: }Story Ed1 (With approval at §330-351, for defences and discharge of contract.); Westlake Ed1 (arts 235-256, with approval as for where the law of the contract coincides with the law of the bankruptcy.)\\No known authors cited.
        \end{small}\\
        \rule{\textwidth}{0.5pt}
        

        \begin{small}
        \begin{center}
        \href{https://heinonline.org/HOL/P?h=hein.engrep/engrc0032&i=1117}{\textit{Cash v Kennion} (1805) 11 Ves 314, 32 ER 1109} \label{37} \\ 
\textit{Currency (Debt---Contract)}\\
        \end{center}
        \textbf{Ch}. A bond which was payable in London but contracted for in Jamaica. The costs of remitting currency were permitted. Rather uninteresting.\\\textit{Cited in: }Westlake Ed1 (arts 230-236, as part of a discussion of currency issues and damages.)\\No known authors cited.
        \end{small}\\
        \rule{\textwidth}{0.5pt}
        

        \begin{small}
        \begin{center}
        \href{https://heinonline.org/HOL/P?h=hein.engrep/engrg0128&i=37}{\textit{O Callaghan v Thomond} (1810) 3 Taunt 82, 128 ER 33} \label{44} \\ 
\textit{Contract---Application of Lex Loci Con---Contract Assignment (Debt---Contract)}\\
        \end{center}
        \textbf{CP}. Certain judgement debts assignable by Irish statute, held to be suable in England in the name of the asignee. Unclear if the debts themselves were Irish in origin (seem to be?).\\\textit{Cited in: }Story Ed1 (§362-373, for assignment of debts); Westlake Ed1 (arts 241-245, for the assignability of debts being judged by the point of their inception.)\\No known authors cited.
        \end{small}\\
        \rule{\textwidth}{0.5pt}
        

        \begin{small}
        \begin{center}
        \href{https://heinonline.org/HOL/P?h=hein.engrep/engrf0105&i=37}{\textit{Snaith v Mingay} (1813) 1 M\&S 87, 105 ER 33} \label{32} \\ 
\textit{Contract---Application of Lex Loci Con (Bill of Exchange)}\\
        \end{center}
        \textbf{KB}. A bill of exchange was signed – leaving blank dates and sums – in Ireland, and transmitted to England were it was filled by a partner of a firm, and thereafter indorsed etc. The bill was “an Irish bill” not requiring an English revenue stamp to be valid. The basis appears to be that the bill came into creation in Ireland on being drawn up, though this is not framed exactly as a lex loci contractus rule.\\\textit{Cited in: }Story Ed1 (Cited with approval at §279-290, to suggest a potential exception to lex loci contractus); Westlake Ed1 (arts 180-183, on the strong basis that the place of drawing should govern a bill of exchange.)\\No known authors cited.
        \end{small}\\
        \rule{\textwidth}{0.5pt}
        

        \begin{small}
        \begin{center}
        \href{https://heinonline.org/HOL/P?h=hein.engrep/engrf0105&i=791}{\textit{Power v Whitmore} (1815) 4 M\&S 141, 105 ER 787} \label{34} \\ 
\textit{Contract---Lex Loci Con---Foreign Judgements---Contract Intention (General Average---Insurance)}\\
        \end{center}
        \textbf{KB}.  \textbf{Uses terms: }[\textit{comity, law of nations}]. The whole case is framed in terms of the “custom of merchants” and the presumed intentions of the parties, and seems also to suggest some potential application of lex loci solutionis. Foreign judgements also come into play, since the sums demanded related to a judgement of a Lisbon court.\\\textit{Cited in: }Westlake Ed1 (arts 225-228, as showing the place of contracting as defining the requirements of the obligation)\\No known authors cited.
        \end{small}\\
        \rule{\textwidth}{0.5pt}
        

        \begin{small}
        \begin{center}
        \href{https://heinonline.org/HOL/P?h=hein.engrep/engrg0127&i=187}{\textit{Splitberger v Kohn} (1815) 1 Star 125, 171 ER 422} \label{47} \\ 
\textit{Contract ()}\\
        \end{center}
        \textbf{NP}. Very vague nisi prius report concerning a promissory note made in Prussia. Only point concerns the inclusion of certain details on the note.\\\textit{Cited in: }Westlake Ed1 (arts 241-245, for the idea that the required form of the validity of an assignment is determined by the place of assignment)\\No known authors cited.
        \end{small}\\
        \rule{\textwidth}{0.5pt}
        

        \begin{small}
        \begin{center}
        \href{https://heinonline.org/HOL/P?h=hein.engrep/engrf0105&i=1181}{\textit{Wolf v Oxholm} (1817) 7 M\&S 92, 105 ER 1177} \label{40} \\ 
\textit{Contract---Contract Discharge (Bankruptcy---Assignment)}\\
        \end{center}
        \textbf{KB}.  \textbf{Uses terms: }[\textit{law of nations}]. Debts owed by Danish subject to English partnership, contracted for in England. Confiscation order by the Danish Crown – sequestering debts owed to English subjects – found not to have lead to discharge of the debts, “not being conformable to the usage of nations.” Strong reliance on International law authorities. Some noting also of the law of assignment.\\\textit{Cited in: }Story Ed1 (§330-351, with approval, to suggest application of the law of nations to prevent discharge of certain debts.)\\\textit{Authors refered to: }Vattel, Grotius, Puffendorf, Bynkerschoek.
        \end{small}\\
        \rule{\textwidth}{0.5pt}
        

        \begin{small}
        \begin{center}
        \href{https://heinonline.org/HOL/P?h=hein.engrep/engrf0105&i=1194}{\textit{Jeffery v McTaggart} (1817) 6 M\&S 126, 105 ER 1190} \label{45} \\ 
\textit{Contract---Application of Lex Loci Con---Bankruptcy (Bankruptcy)}\\
        \end{center}
        \textbf{KB}. Choses in action (unclear where from or of what nature) deemed not to be assigned by Scottish bankruptcy proceedings, thereby not allowing the plaintiff to bring the action in his own name. Actual reasoning based on the language of the act in question.\\\textit{Cited in: }Westlake Ed1 (arts 241-245, for the non-assignability of contracts by the point of their inception)\\No known authors cited.
        \end{small}\\
        \rule{\textwidth}{0.5pt}
        

        \begin{small}
        \begin{center}
        \href{https://heinonline.org/HOL/P?h=hein.engrep/engrf0107&i=76}{\textit{Milne v Graham} (1823) 1 B\&C 192, 107 ER 72} \label{48} \\ 
\textit{Contract---Assignment (Debt---Contract)}\\
        \end{center}
        \textbf{KB}. Allowing for an action on a Scottish promissory note brought in England. The entire question appears to relate to the application of an English statute.\\\textit{Cited in: }Westlake Ed1 (arts 241-245, for the idea that the required form of the validity of an assignment is determined by the place of assignment)\\No known authors cited.
        \end{small}\\
        \rule{\textwidth}{0.5pt}
        

        \begin{small}
        \begin{center}
        \href{None}{\textit{Arnott v Redfern} (1825) 2 Car \& P 88, 172 ER 40} \label{35} \\ 
\textit{Contract---Lex Loci Con---Interest ()}\\
        \end{center}
        \textbf{NP}. A very short (nisi prius) decision – though strongly suggesting that the place of contracting (England) was determinative of the issue.\\\textit{Cited in: }Westlake Ed1 (arts 230-236, cited critically in connection to an understanding of the laws of interest.)\\No known authors cited.
        \end{small}\\
        \rule{\textwidth}{0.5pt}
        

        \begin{small}
        \begin{center}
        \href{https://heinonline.org/HOL/P?h=hein.engrep/engrk0173&i=1087}{\textit{Bentley v Northouse} (1827) M\&M 66, 173 ER 1083} \label{49} \\ 
\textit{Contract---Assignment (Bill of Exchange)}\\
        \end{center}
        \textbf{NP}. Bills of Exchange made in Scotland could be transferred by indorsement in England. The whole reasoning seems to hinge on the application of statute, this being said to also cover notes made outside of England.\\\textit{Cited in: }Westlake Ed1 (arts 241-245, for the idea that the required form of the validity of an assignment is determined by the place of assignment)\\No known authors cited.
        \end{small}\\
        \rule{\textwidth}{0.5pt}
        

        \begin{small}
        \begin{center}
        \href{https://heinonline.org/HOL/P?h=hein.engrep/engra0006&i=565}{\textit{Pattison v Mills} (1828) 1 Dow \& Clark 342, 6 ER 553} \label{33} \\ 
\textit{Contract---Application of English Statute---Application of Lex Loci Con---Illegality (Insurance)}\\
        \end{center}
        \textbf{HL(SC)}. A contract made in Glasgow was not subject to an English statute giving a monopoly for insuring marine risks. Strong dictum by Lord Lyndhurst, strongly hinting at a lex loci contractus rule.\\\textit{Cited in: }Story Ed1 (§279-290, for the principle that lex loci refers to place where an agent goes to make a contract); Westlake Ed1 (Cited critically at §192-200, for a misunderstanding of the rules on illegality.)\\No known authors cited.
        \end{small}\\
        \rule{\textwidth}{0.5pt}
        

        \begin{small}
        \begin{center}
        \href{https://heinonline.org/HOL/P?h=hein.engrep/engrf0108&i=1124}{\textit{Phillips v Allan} (1828) 7 B\&C 477, 108 ER 1120} \label{42} \\ 
\textit{Contract---Contract Discharge---Bankruptcy---Application of Lex Loci Con (Bankruptcy---Bill of Exchange)}\\
        \end{center}
        \textbf{KB}. Dsicharge by cessio in bonorum by the Court of Session in Scotland not effective to discharge debt on bill of exchange contracted in England. Actual decision not clearly reasoned in terms of applicable law; greater emphasis placed on jurisdictional powers of courts, and the lack of benefit by the plaintiff from the Scotch proceedings.\\\textit{Cited in: }Story Ed1 (§330-351, with approval); Westlake Ed1 (arts 235-256, with approval – for the position of where the place of contracting does not coincide with the place of discharge.)\\No known authors cited.
        \end{small}\\
        \rule{\textwidth}{0.5pt}
        

        \begin{small}
        \begin{center}
        \href{https://heinonline.org/HOL/P?h=hein.engrep/engrf0109&i=82}{\textit{De La Chaumette v Bank of England} (1829) 9 B\&C 208, 109 ER 78} \label{43} \\ 
\textit{Contract---Application of Lex Loci Con (Bill of Exchange)}\\
        \end{center}
        \textbf{KB}. Action for trover and non-payment on a bearer note issued by the defendants. Defendants refuse to pay on the basis of the note having been stolen. Clear implication is that the rules governing the assignment / entitlement to the note subject to the English rules of giving value, though not a clear conclusion. New trial ordered, later proceedings noted.\\\textit{Cited in: }Story Ed1 (§330-351, with approval] Westlake Ed)\\No known authors cited.
        \end{small}\\
        \rule{\textwidth}{0.5pt}
        

        \begin{small}
        \begin{center}
        \href{https://heinonline.org/HOL/P?h=hein.engrep/engrf0109&i=1077}{\textit{Scott v Bevan} (1831) 2 Ba \& Ad 78, 109 ER 1073} \label{36} \\ 
\textit{ (Debt---Contract)}\\
        \end{center}
        \textbf{KB}. Actual exchange rate between England and Jamaica applied.\\\textit{Cited in: }Westlake Ed1 (arts 230-236, as part of a discussion of currency issues and damages.)\\No known authors cited.
        \end{small}\\
        \rule{\textwidth}{0.5pt}
        
\end{document}