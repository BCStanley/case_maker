
\documentclass[twoside]{article}

\usepackage{verbatim}
\usepackage{longtable}
\usepackage{hyperref}
\hypersetup{
    colorlinks=false,
    linkcolor=blue,
    filecolor=magenta,
    urlcolor=cyan,
    bookmarks=true,
    hidelinks=true,
}

% 1. Sets the font
\usepackage{fontspec}
\setmainfont{Charter}
\renewcommand{\baselinestretch}{1.25}

%2. Sets the page size, etc.

\usepackage{geometry}
\geometry{
a4paper,
left = 20mm,
right = 20mm,
top = 20mm,
bottom = 20mm,
}

\usepackage{titlesec}
\titleformat{\section}
  {\normalfont\normalsize\bfseries\scshape}{\thesection}{1em}{}
\titleformat{\subsection}
  {\normalfont\normalsize\itshape}{\thesubsection}{1em}{}
\titleformat{\subsubsection}
  {\normalfont\normalsize}{\thesubsubsection}{1em}{}

\def \Title{Table of Conflict of Laws Cases}
\def \Author{Benedict Stanley}

\title{\Title}
\author{\Author}
\date{\today}

\usepackage{fancyhdr}
\pagestyle{fancy}
\fancyhf{}
\fancyhead{}
\fancyhead[L]{}
\fancyhead[R]{}
\fancyhead[C]{}

\fancyfoot[L]{\Title}
\fancyfoot[R]{\thepage}
\renewcommand{\headrulewidth}{0pt}
\renewcommand{\footrulewidth}{0pt}


\usepackage{tocloft}

\addtolength{\cftsecnumwidth}{2pt}
\addtolength{\cftsubsecnumwidth}{5pt}
\addtolength{\cftsubsubsecnumwidth}{9pt}

\begin{document}

\maketitle

\tableofcontents

\section{Choice of Law: Contract}\
\subsection{Contract: Resolution in Favour of Place of Contract's Inception}

These are all of the cases I have seen which either (1) seem to resolve issues themselves with reference to the law of the place where the contractual obligation originated, \textit{or} (2) are cited by others as such. It should be noted that very few of these cases actually make use of the wording “\textit{lex loci contractus}” within them.
\\ 
\begin{longtable}{p{4cm} p{1.7cm} p{2.5cm} p{5cm} p{2.5cm}}
\hline
Name & Court & Subject(s) & Comment & Cited By\\
\hline\textit{Dungannon v Hackett} (1702) Eq Ca Abr 289, 23 ER 855 & \small{Ch} & \small{Contract Interest---Contract---Application of Lex Loci Con} & \small{Implication is that the interest should be determined by the place where it was contracted for.} &  \\ 
\textit{Foubert v Turst} (1703) 1 Brown Parl Cas,, 1 ER 464,  24 ER 101 & \small{HL} & \small{Contract---Contract Intention---Lex Loci Con} & \small{French Marriage Contract affirmed, by its express terms, to refer to the custom of Paris. Actual reasoning unclear, but it seems to be based on ideas of intent.} &  \\ 
\textit{Ekins v East-India Co} (1717) 1 P Wms 394, 24 ER 441 & \small{Ch} & \small{Contract Interest---Contract---Application of Lex Loci Con} & \small{Action of taking and selling goods in India carried Indian interest. Not be a choice of law rule “must be presumed to have common advantage” of money there.} & \small{Westlake Ed1 (arts 230-236 for breach of obligations)}\\ 
\textit{Tremoult v Dedire} (1718) 1 P Wms 429, 24 ER 458 & \small{Ch} & \small{Contract---Contract Law of Terms---Marriage---Contract Intention---Lex Loci Con} & \small{Clear implication that Dutch marriage articles could be construed and applied according to the laws of Holland. Evidence of this is required “to take notice of foreign laws” (contrast with Foubert). Unclear what the basis of this is, though it seems assumed.} &  \\ 
\textit{Phipps v Earl of Angelsea} (1721) 1 P Wms 697, 5 Bro PC 45, 24 ER 576 & \small{Ch} & \small{Contract---Lex Loci Con} & \small{English interest, as this was the place where the contract was made (and where it was to be performed.) Very little specific reasoning on the issue.} & \small{Story Ed1 (Cited at §279-290 to say that the lex loci rule is not circumvented by the location of the security)}\\ 
\textit{Saunders v Drake} (1742) 2 Atk 465, 26 ER 681 & \small{Ch} & \small{Contract Currency---Contract---Application of Lex Loci Con---Contract Intention} & \small{“Jamaican Money” applied to testators estate, with strong reliance on intention of the parties.} &  \\ 
\textit{Ballantine v Golding} (1784) Cooke’s Bankrupt Laws 419 & \small{KB} & \small{Contract---Contract Discharge---Bankruptcy---Application of Lex Loci Con} & \small{Lord Mansfield giving the rule of a discharge of debts of a bankrupt being effective where the the debts arose there (though the actual rule might be wider.)} & \small{Story Ed1 (§330-351, with approval] Westlake Ed)}\\ 
\textit{Bodham v Ryley} (1787) 4 Brown 561, 2 ER 382 & \small{HL} & \small{Contract Interest---Contract---Application of Lex Loci Con} & \small{Report references some wide propositions, including a note on Huber and a general lex loci contractus principle. The actual grounds on which Indian interest was allowed seem less clear, and more related to presumed custom and intent of the parties.} &  \\ 
\textit{Melan v Fitzjames} (1792) 1 Bos \& Pul 138, 126 ER 822 & \small{CP} & \small{Contract---Application of Lex Loci Con} & \small{Demonstrates a very clear lex loci contractus understanding, the Chief Justice clearly reasoning on such a line. There is a division in opinion. There is also some hint of the idea that reference to another law could be relevant.} & \small{Story Ed1 (Cited with approval at §263-266, as to the applicable law for determining the nature of a contract, and the locus regit actum principle)}\\ 
\textit{Alves v Hodgson} (1797) 7 TR 241, 101 ER 953 & \small{KB} & \small{Contract---Contract Formality---Application of Lex Loci Con} & \small{Lord Kenyon clearly suggesting that one must resort to the law of where a contract was made (in this case, Jamaica) to determine validity and formality requirements.} & \small{Westlake Ed1 (arts 173-176, cited with general approval for the locus regit actum rule and formalities.)}\\ 
\textit{Smith v Buchanan} (1800) 1 East 6, 102 ER 3 & \small{KB} & \small{Contract---Contract Discharge---Bankruptcy---Application of Lex Loci Con---Property} & \small{Maryland discharge by bankruptcy held to not affect debts contracted for in England, with a strong preference for lex loci contractus “It is impossible to say that a contract made in one country is to be governed by the laws of another.” Ballatine v Golding distinguished.} & \small{Story Ed1 (§330-351, with approval); Westlake Ed1 (arts 235-256, with approval.); Story Ed1 (at §403-409, for the principles on the assignment of debts in bankruptcy.)}\\ 
\textit{Innes v Dunlop} (1800) 8 TR 595, 101 ER 1565 & \small{KB} & \small{Contract---Application of Lex Loci Con---Contract Assignment} & \small{Assignee of a Scottish bond allowed to sue in his own name in England. Assignments valid under Scots, but not English law. Unclear where assignment took place, though bond was clearly Scottish.} & \small{Westlake Ed1 (arts 241-245, for the assignability of debts being judged by the point of their inception.)}\\ 
\textit{Potter v Brown} (1804) 5 East 124, 102 ER 1016 & \small{KB} & \small{Contract---Contract Discharge---Application of Lex Loci Con---Bankruptcy} & \small{A bill drawn in America on someone in England, was discharged by American certificate of bankruptcy, thereby discharging the defendant in the event of dishonour in England.} & \small{Story Ed1 (With approval at §330-351, for defences and discharge of contract.); Westlake Ed1 (arts 235-256, with approval as for where the law of the contract coincides with the law of the bankruptcy.); Story Ed1 (at §403-409, for the English rules on assignment in bankruptcy, suggesting that this is determined by one law being the law of domicile)}\\ 
\textit{O Callaghan v Thomond} (1810) 3 Taunt 82, 128 ER 33 & \small{CP} & \small{Contract---Application of Lex Loci Con---Contract Assignment} & \small{Certain judgement debts assignable by Irish statute, held to be suable in England in the name of the asignee. Unclear if the debts themselves were Irish in origin (seem to be?).} & \small{Story Ed1 (§362-373, for assignment of debts); Westlake Ed1 (arts 241-245, for the assignability of debts being judged by the point of their inception.)}\\ 
\textit{Dalrymple v Dalrymple} (1811) 2 Hag Con 54, 161 ER 665 & \small{Delegates} & \small{Contract---Marriage---Lex Loci Contractus} & \small{A very famous dictum of Sir William Scott (Baron Stowell) invoking the express principle of English law retreating to refer matters to the Law of Scotland. The actual case is sprawling and complex, and mostly of a factual character (and dealing with the relevant points of Scots marriage law.) Extensive reference to continental jurists are added in the report (which is lengthy).} & \small{Story Ed1 (§270-278, in connection to the interpretation of contracts); Westlake Ed1 (generally, for the basic position of “choice of law”)}\\ 
\textit{Clegg v Levy} (1812) 3 Camp 166, 170 ER 1343 & \small{NP} & \small{Contract---Contract Formality---Application of Lex Loci Con} & \small{A very short (nisi prius) decision. Strongly suggesting that formality requirements (in this case, a stamp for a sale of goods contract in Surinam) are to be submitted to the place where the contract is made.} & \small{Story Ed1 (§260-262, with approval for rules on formality and the locus regit actum rule); Westlake Ed1 (arts 173-176, for  total preference of lex loci contractus in issues of formality.)}\\ 
\textit{Snaith v Mingay} (1813) 1 M\&S 87, 105 ER 33 & \small{KB} & \small{Contract---Application of Lex Loci Con} & \small{A bill of exchange was signed – leaving blank dates and sums – in Ireland, and transmitted to England were it was filled by a partner of a firm, and thereafter indorsed etc. The bill was “an Irish bill” not requiring an English revenue stamp to be valid. The basis appears to be that the bill came into creation in Ireland on being drawn up, though this is not framed exactly as a lex loci contractus rule.} & \small{Story Ed1 (Cited with approval at §279-290, to suggest a potential exception to lex loci contractus); Westlake Ed1 (arts 180-183, on the strong basis that the place of drawing should govern a bill of exchange.)}\\ 
\textit{Power v Whitmore} (1815) 4 M\&S 141, 105 ER 787 & \small{KB} & \small{Contract---Lex Loci Con---Foreign Judgements---Contract Intention} & \small{The whole case is framed in terms of the “custom of merchants” and the presumed intentions of the parties, and seems also to suggest some potential application of lex loci solutionis. Foreign judgements also come into play, since the sums demanded related to a judgement of a Lisbon court.} & \small{Westlake Ed1 (arts 225-228, as showing the place of contracting as defining the requirements of the obligation)}\\ 
\textit{Jeffery v McTaggart} (1817) 6 M\&S 126, 105 ER 1190 & \small{KB} & \small{Contract---Application of Lex Loci Con---Bankruptcy} & \small{Choses in action (unclear where from or of what nature) deemed not to be assigned by Scottish bankruptcy proceedings, thereby not allowing the plaintiff to bring the action in his own name. Actual reasoning based on the language of the act in question.} & \small{Westlake Ed1 (arts 241-245, for the non-assignability of contracts by the point of their inception)}\\ 
\textit{Arnott v Redfern} (1825) 2 Car \& P 88, 172 ER 40 & \small{NP} & \small{Contract---Lex Loci Con---Interest} & \small{A very short (nisi prius) decision – though strongly suggesting that the place of contracting (England) was determinative of the issue.} & \small{Westlake Ed1 (arts 230-236, cited critically in connection to an understanding of the laws of interest.)}\\ 
\textit{Pattison v Mills} (1828) 1 Dow \& Clark 342, 6 ER 553 & \small{HL(SC)} & \small{Contract---Application of English Statute---Application of Lex Loci Con---Illegality} & \small{A contract made in Glasgow was not subject to an English statute giving a monopoly for insuring marine risks. Strong dictum by Lord Lyndhurst, strongly hinting at a lex loci contractus rule.} & \small{Story Ed1 (§279-290, for the principle that lex loci refers to place where an agent goes to make a contract); Westlake Ed1 (Cited critically at §192-200, for a misunderstanding of the rules on illegality.)}\\ 
\textit{Phillips v Allan} (1828) 7 B\&C 477, 108 ER 1120 & \small{KB} & \small{Contract---Contract Discharge---Bankruptcy---Application of Lex Loci Con} & \small{Dsicharge by cessio in bonorum by the Court of Session in Scotland not effective to discharge debt on bill of exchange contracted in England. Actual decision not clearly reasoned in terms of applicable law; greater emphasis placed on jurisdictional powers of courts, and the lack of benefit by the plaintiff from the Scotch proceedings.} & \small{Story Ed1 (§330-351, with approval); Westlake Ed1 (arts 235-256, with approval – for the position of where the place of contracting does not coincide with the place of discharge.)}\\ 
\textit{De La Chaumette v Bank of England} (1829) 9 B\&C 208, 109 ER 78 & \small{KB} & \small{Contract---Application of Lex Loci Con} & \small{Action for trover and non-payment on a bearer note issued by the defendants. Defendants refuse to pay on the basis of the note having been stolen. Clear implication is that the rules governing the assignment / entitlement to the note subject to the English rules of giving value, though not a clear conclusion. New trial ordered, later proceedings noted.} & \small{Story Ed1 (§330-351, with approval); Westlake Ed1 (arts 235-256, with approva)}\\ 
\textit{Holdsworth v Hunter} (1830) 10 B \& C 449, 109 ER 517 & \small{KB} & \small{Contract---Contract Formality---Application of Lex Loci Con} & \small{Applying locus regit actum to reverse effect: the English Stamp Act did not effect a bill drawn in India. (Though much of the reasoning focuses on the statute itself).} & \small{Westlake Ed1 (arts 180-183, specifically in relation to the formality rules for indorsement.)}\\ 
\textit{Huber v Steiner} (1835) 2 Bing NC 202, 132 ER 80 & \small{CP} & \small{Contract---Application of Lex Loci Con} & \small{An interesting decision. The court relies on an express observation by Story, that laws of prescription are admitted to the lex fori unless they serve to make causes of action nullities or themselves extinguished (as opposed to merely time-bared). The French law of prescription (on a promisory note made in France, itself readily admitted to be subject to the terms of the Code de Commerce) was found to fall into the former category.} & \small{Westlake Ed1 (arts 250-252, critically, in connection to the English preference for the lex fori in rules of prescription)}\\ 
\textit{Don v Lippmann} (1837) 5 Cl\& F 1, 7 ER 303 & \small{HL (SC)} & \small{Contract---Prescription---Application of Lex Loci Con---Contract Intention} & \small{A very fully reasoned case (one of the most heavily in terms of the conflict of laws I have seen.) A very clear distinction is drawn between the concepts of lex loci contractus, lex loci solutionus, and lex fori. The clear understanding is that the lex loci contractus determined the substantive law (a bill of exchange made in France) though this is wrapped somewhat in language of intention of the parties. The lex fori is said to firmly govern the issues of prescription, in this case making the claim out of time. (There is a separate issue on the enforcement of a French Judgement).} & \small{Westlake Ed1 (arts 187-191, critically, to suggest and incorrect approach to the interpretation of obligations); Westlake Ed1 (arts 250-252, critically, in connection to the application of the lex fori for prescription (a position Westlake rejects))}\\ 
\textit{Campbell v Dent} (1838) 2 Moo PC 292, 12 ER 1016 & \small{PC} & \small{Contract---Application of Lex Loci Con---Application of Lex Loci Sol---Contract Discharge} & \small{Various points related to mortgages made on lands in Demerara, the contracts themselves being governed by the law of Scotland. They seem to have been made and to be performed in Scotland, so it is hard to come to a definitive point on the choice of law issue. This also seems to extend to the undoing of a contract.} & \small{Westlake Ed1 (art 229, in connection to the discharge of obligations)}\\ 
\textit{Quarrier v Colston} (1842) 1 Ph 147, 41 ER 587 & \small{Ch} & \small{Contract---Contract Validity---Application of Lex Loci Con} & \small{Debts owed from a debt contracted for in Germany in connection to gambling were recoverable in England. Though the reasoning of the case is mostly in terms of the presumption that the games played “at public tables” were lawful.} & \small{Westlake Ed1 (at arts 192-200, with approval, to suggest the place of contracting as determinative for the sufficiency and validity of consideration.)}\\ 
\textit{Benham v Mornington} (1846) 3 CB 133, 136 ER 54 & \small{CP} & \small{Contract---Contract Formality---Application of Lex Loci Con} & \small{Quite interesting. A penal bond was entered into in France, though (it was alleged) without the required formality requirements under the French code. It was held that specific evidence needed to be shown of the relevant foreign law, but the clear implication is that proof of this would have been sufficient.} & \small{Westlake Ed1 (arts 173-176, cited with general approval for the locus regit actum rule and formalities, and that the domicile of the parties should not matter.)}\\ 
\textit{Allen v Kemble} (1848) 6 Moo PC 321, 13 ER 704 & \small{PC} & \small{Contract---Bill of Exchange---Application of Lex Loci Sol---Application of Lex Loci Con} & \small{Expressly favouring the lex loci contractus for a bill of exchange (Demerara, where it was drawn) over the lex loci solutionis (London, where it was to be paid) for determining liabilities under a bill of exchange. This meant that the Roman-Dutch law prevailing in Demerara governed the issues of set-off by the acceptor.} & \small{Westlake Ed1 (arts 225-288, with approval)}\\ 
\textit{Gibbs v Fremont} (1853) 9 Exch 25, 156 ER 11 & \small{Ex} & \small{Contract---Application of Lex Loci Con---Contract Intention} & \small{Bill of Exchange drawn in California but payable in Washington DC. California rate of interest applied, on the express approval of the lex loci contractus. Expressly noted however that this was a principle that only applied absent of express indication. Interest expressly held to be a question of law not fact.} & \small{Westlake Ed1 (arts 230-236, for the application of ex mora interest in bills of exchange.)}\\ 
\textit{Sudlow v Dutch Rhenish Railway Company} (1855) 21 Beav 43, 52 ER 774 & \small{Rolls} & \small{Contract---Lex Loci Con} & \small{Claim for relief against forefeiture of shares in a Dutch company; rejected, on the basis that “this is a Dutch contract”, there also being evidence of a similar case being tried before the Dutch courts. Not clear on what terms the choice of law was decided for?} & \small{Westlake Ed1 (at arts 230-236, as to the law of the contract itself defining the manner and extent of performance)}\\ 
\textit{Hope v Hope} (1856) 2 Beav 351, 52 ER 1143 & \small{Rolls} & \small{Contract---Illegality---Application of Lex Loci Con---Contract Intention} & \small{Actual decision seems to imply that the issues simply good not be decided on demurrer – though subsequent proceedings are apparently found elsewhere. The relevant foreign law related to certain French provisions related to marriage, and custodial rights of children. Important observations on the lex loci contractus rule, clearly seeing that it did not extent to a situation where the contract was to be performed in another place.} & \small{Westlake Ed1 (arts 192-200, cited as an instance whereby a foreign law will not be enforced in English courts on moral grounds)}\\ 
\textit{Brook v Brook} (1858) 3 Sm \& Grif 481, 65 ER 746 & \small{VC} & \small{Marriage---Contract---Contract Validity---Application of Lex Loci Con---Application of English Statute---Illegality} & \small{A marriage was entered into in Denmark between a widower and late wife’s sister. Such marriages are outlawed in England, but valid by the laws of Denmark. Man and woman had no domicile in Denmark. It was held that the marriage was invalid in English courts. The reasoning remarks specifically on the principle of lex loci contractus (and is very fully argued) citations to many jurists are included. The exact principle either seems to be on the application of the English statute or not admitting foreign law on principles of illegitimacy or immorality. Later appealed to House of Lords an upheld.} &  \\ 
\textit{Brook v Brook (no 2)} (1861) 9 HLC 193, 11 ER 703 & \small{HL} & \small{Marriage---Contract---Contract Validity---Application of Lex Loci Con---Application of English Statute---Illegality} & \small{Continuation of case below. A marriage was entered into in Denmark between a widower and late wife’s sister. Such marriages are outlawed in England, but valid by the laws of Denmark. Man and woman had no domicile in Denmark. It was held that the marriage was invalid in English courts. The reasoning is slightly modified. There is a more clear line drawn between the lex loci contractus (governing the forms of marriage) and the lex domicilii (governing its requirements and aspects). In particular also, a policy of preventing evasion of English public policy is put forward.} &  \\ 
\textit{Scott v Pilkington} (1862) 2 B\&S 11, 121 ER 978 & \small{QB} & \small{Contract---Application of Lex Loci Con---Foreign Judgements} & \small{Bill of Exchange drawn in New York but to be paid in London. Judgement obtained by the Law of New York (said to be erroneous on New York Law, and wrong to have applied New York Law.) (1) That a court would not examine a foreign judgement for being errenous, even on its own law. (2) That an appeal being pending might be a basis to say proceedings but would not bar them. (3) That, potentially, a failure to apply the correct law would violate the “comity of nations” and thereby lead the decision not to be enforced. (4) That a bill of exchange drawn in New York, but payable in London had to be governed by the lex loci contractus.} &  \\ 
\textit{In re Melbourn (no 1)} (1871) LR Ch App 64 & \small{CainCh} & \small{Contract---Marriage---Application of Lex Loci Con---Application of Lex Fori---Bankruptcy} & \small{Marriage in Batavia with marriage contract, excluding community of property. That marriage contract was not registered, which was required by the law of Batavia. Since this was a matter of proof (rather than validity) it did not affect the inter se admissability of the claim, but could not affect the distribution of assets amongst the creditors, which was submitted to the lex fori. Strong preference for the lex loci contractus as the effect of the marriage contract.} &  \\ 
\end{longtable}\subsection{Contract: Resolution in Favour of Loci Solutionis or Party Intention}

The following are the cases that lean the other way from the previous section. These either resolve issues by reference to party \textit{intention} or \textit{the loci solutionis}, or are referred to as such.
\\ 
\begin{longtable}{p{4cm} p{1.7cm} p{2.5cm} p{5cm} p{2.5cm}}
\hline
Name & Court & Subject(s) & Comment & Cited By\\
\hline\textit{Foubert v Turst} (1703) 1 Brown Parl Cas,, 1 ER 464,  24 ER 101 & \small{HL} & \small{Contract---Contract Intention---Lex Loci Con} & \small{French Marriage Contract affirmed, by its express terms, to refer to the custom of Paris. Actual reasoning unclear, but it seems to be based on ideas of intent.} &  \\ 
\textit{Tremoult v Dedire} (1718) 1 P Wms 429, 24 ER 458 & \small{Ch} & \small{Contract---Contract Law of Terms---Marriage---Contract Intention---Lex Loci Con} & \small{Clear implication that Dutch marriage articles could be construed and applied according to the laws of Holland. Evidence of this is required “to take notice of foreign laws” (contrast with Foubert). Unclear what the basis of this is, though it seems assumed.} &  \\ 
\textit{Saunders v Drake} (1742) 2 Atk 465, 26 ER 681 & \small{Ch} & \small{Contract Currency---Contract---Application of Lex Loci Con---Contract Intention} & \small{“Jamaican Money” applied to testators estate, with strong reliance on intention of the parties.} &  \\ 
\textit{Robinson v Bland} (1760) 2 Bur 1077, 97 ER 717 & \small{KB} & \small{Contract---Contract Validity---Application of Lex Loci Sol---Application of Lex Fori---Illegality---Property} & \small{An obviously very significant case. the defendant -- having engaged in gambling in Paris -- entered into a Bill of Exchange for £300 with the plaintiffs, payable on 10 days of sight in England. The plaintiffs then came to London to recover, and failed, on the basis that the bill of exchange was void by the laws of England. Lord Mansfield relies on Huber to give a rule other than lex loci contractus. The other judges simply apply the lex fori.} & \small{Story Ed1 (Cited with approval at §279-290, to suggest a potential exception to lex loci contractus); Westlake Ed1 (Critically, at arts 129-200, in connection to a discussion of illegality.); Story Ed1 (Cited at §383 in support of the idea that special classes of property, e.g. shares, are governed by the peculiar law they are associated with.)}\\ 
\textit{Power v Whitmore} (1815) 4 M\&S 141, 105 ER 787 & \small{KB} & \small{Contract---Lex Loci Con---Foreign Judgements---Contract Intention} & \small{The whole case is framed in terms of the “custom of merchants” and the presumed intentions of the parties, and seems also to suggest some potential application of lex loci solutionis. Foreign judgements also come into play, since the sums demanded related to a judgement of a Lisbon court.} & \small{Westlake Ed1 (arts 225-228, as showing the place of contracting as defining the requirements of the obligation)}\\ 
\textit{Don v Lippmann} (1837) 5 Cl\& F 1, 7 ER 303 & \small{HL (SC)} & \small{Contract---Prescription---Application of Lex Loci Con---Contract Intention} & \small{A very fully reasoned case (one of the most heavily in terms of the conflict of laws I have seen.) A very clear distinction is drawn between the concepts of lex loci contractus, lex loci solutionus, and lex fori. The clear understanding is that the lex loci contractus determined the substantive law (a bill of exchange made in France) though this is wrapped somewhat in language of intention of the parties. The lex fori is said to firmly govern the issues of prescription, in this case making the claim out of time. (There is a separate issue on the enforcement of a French Judgement).} & \small{Westlake Ed1 (arts 187-191, critically, to suggest and incorrect approach to the interpretation of obligations); Westlake Ed1 (arts 250-252, critically, in connection to the application of the lex fori for prescription (a position Westlake rejects))}\\ 
\textit{Campbell v Dent} (1838) 2 Moo PC 292, 12 ER 1016 & \small{PC} & \small{Contract---Application of Lex Loci Con---Application of Lex Loci Sol---Contract Discharge} & \small{Various points related to mortgages made on lands in Demerara, the contracts themselves being governed by the law of Scotland. They seem to have been made and to be performed in Scotland, so it is hard to come to a definitive point on the choice of law issue. This also seems to extend to the undoing of a contract.} & \small{Westlake Ed1 (art 229, in connection to the discharge of obligations)}\\ 
\textit{Cooper v Waldegrave} (1840) 2 Beav 282, 48 ER 1189 & \small{Rolls} & \small{Contract---Bill of Exchange---Application of Lex Loci Sol---Interest} & \small{English interest applied to non-payment of a bill of exchange in England (where payment was due), though the bill was drawn and accepted in France. The distinction appears to be between remedial issues (according to the law of England) and rights themselves (which are by the law of France). There is a strong preference in favour of the “general rule” favouring the place where the contract is made.} & \small{Westlake Ed1 (at arts 230-236, favouring the use of interest where payment is due – though unclear as whether a matter of fact or law)}\\ 
\textit{Rothschild v Currie} (1841) 1 QB 43, 113 ER 1045 & \small{QB} & \small{Contract---Bill of Exchange---Application of Lex Loci Sol} & \small{A bill was drawn in England against a French bank, and thereafter indorsed in England and transmitted to France for payment. French requirements for notice in going against indorser applied. Interestingly, though the language of “lex loci contractus” is used by the Court, the actual solution appears to be on the basis of the lex loci solutionis – it refers to the place of payment. Protest etc. are held to be substantive rights under the contract, not mere formalities against which suit can be brought.} & \small{Westlake Ed1 (arts 225-288, with disapproval)}\\ 
\textit{Allen v Kemble} (1848) 6 Moo PC 321, 13 ER 704 & \small{PC} & \small{Contract---Bill of Exchange---Application of Lex Loci Sol---Application of Lex Loci Con} & \small{Expressly favouring the lex loci contractus for a bill of exchange (Demerara, where it was drawn) over the lex loci solutionis (London, where it was to be paid) for determining liabilities under a bill of exchange. This meant that the Roman-Dutch law prevailing in Demerara governed the issues of set-off by the acceptor.} & \small{Westlake Ed1 (arts 225-288, with approval)}\\ 
\textit{Gibbs v Fremont} (1853) 9 Exch 25, 156 ER 11 & \small{Ex} & \small{Contract---Application of Lex Loci Con---Contract Intention} & \small{Bill of Exchange drawn in California but payable in Washington DC. California rate of interest applied, on the express approval of the lex loci contractus. Expressly noted however that this was a principle that only applied absent of express indication. Interest expressly held to be a question of law not fact.} & \small{Westlake Ed1 (arts 230-236, for the application of ex mora interest in bills of exchange.)}\\ 
\textit{Hope v Hope} (1856) 2 Beav 351, 52 ER 1143 & \small{Rolls} & \small{Contract---Illegality---Application of Lex Loci Con---Contract Intention} & \small{Actual decision seems to imply that the issues simply good not be decided on demurrer – though subsequent proceedings are apparently found elsewhere. The relevant foreign law related to certain French provisions related to marriage, and custodial rights of children. Important observations on the lex loci contractus rule, clearly seeing that it did not extent to a situation where the contract was to be performed in another place.} & \small{Westlake Ed1 (arts 192-200, cited as an instance whereby a foreign law will not be enforced in English courts on moral grounds)}\\ 
\end{longtable}\section{Delicts or Torts}

This is a vague list of any and all cases (there are not many of them) pertaining to torts or extra-contractual liability (these terms are obviously slippery) in some way.
\\ 
\begin{longtable}{p{4cm} p{1.7cm} p{2.5cm} p{5cm} p{2.5cm}}
\hline
Name & Court & Subject(s) & Comment & Cited By\\
\hline\textit{The Vernon} (1842) 1 W Rob 316, 166 ER 591 & \small{Admiralty} & \small{Delict---High Seas---Application of English Statute} & \small{A collision on the High Seas left firmly to the determination of the lex fori as set out in an English statute. Don v Lipman cited.} & \small{Westlake Ed1 (arts 149-158, cited in connection to looking to laws common to the defendants for the law of obligations)}\\ 
\textit{Caldwell v Vanvlissengen} (1851) 9 Hare 415, 68 ER 571 & \small{VC} & \small{Delict} & \small{Injunction issued against Dutch defendants to prevent them violating the terms of an English patent on some invention. The reasoning is quite rich with ideas, but not really firm conclusions. There are some general illusions to laws applying by virtue of someone’s nationality, but the firm conclusion is that the rights in question can be protected.} & \small{Westlake Ed1 (art 237-240, in connection to foreigners being held liable for English delicts.)}\\ 
\textit{The Zollverein} (1856) Swabey 96, 166 ER 1038 & \small{Admiralty} & \small{High Seas---Delict---Application of English Statute} & \small{Dr Lushington declining to apply the provisions of the Merchant Marine Act in a claim brought by a foreign vessel against a British one. Most of the case seems to turn on the application and meaning of the statute itself. Dr Lushington’s residual position is that “the law maritime” applies.} & \small{Westlake Ed1 (arts 148-152, used to suggest there are cases where the law “common to the parties” is used, though purpose of citation is unclear.)}\\ 
\textit{The Dumfries} (1856) Swabey 63, 166 ER 1021 & \small{Admiralty} & \small{Law Maritime---High Seas---Delict---Application of English Statute} & \small{Dr Lushington really repeats the observations made in The Zollverein, that for a collision on the High Seas between a British and Foreign vessel, the law maritime is to apply.} &  \\ 
\end{longtable}\section{Property}

These are all of the cases that on some level relate to property \textit{in general} --- without any specific reference to in what way (however passingly).
\\ 
\begin{longtable}{p{4cm} p{1.7cm} p{2.5cm} p{5cm} p{2.5cm}}
\hline
Name & Court & Subject(s) & Comment & Cited By\\
\hline\textit{Morrisons Case} (1749) 1 H Bl 677, 126 ER 385 & \small{CtS} & \small{Property---Property Domicile---Assignment---Bankruptcy} & \small{Very hard to tell much from the decision itself (we are only getting it second hand.) It seems to reflect a general notion that the authority assign is granted to the English court of bankruptcy, but not particular mention is made of domicile.} & \small{Story Ed1 (at §395-400, in support of the general position that the assignment of debts should be governed by the domicile of the person holding the debt.)}\\ 
\textit{Robinson v Bland} (1760) 2 Bur 1077, 97 ER 717 & \small{KB} & \small{Contract---Contract Validity---Application of Lex Loci Sol---Application of Lex Fori---Illegality---Property} & \small{An obviously very significant case. the defendant -- having engaged in gambling in Paris -- entered into a Bill of Exchange for £300 with the plaintiffs, payable on 10 days of sight in England. The plaintiffs then came to London to recover, and failed, on the basis that the bill of exchange was void by the laws of England. Lord Mansfield relies on Huber to give a rule other than lex loci contractus. The other judges simply apply the lex fori.} & \small{Story Ed1 (Cited with approval at §279-290, to suggest a potential exception to lex loci contractus); Westlake Ed1 (Critically, at arts 129-200, in connection to a discussion of illegality.); Story Ed1 (Cited at §383 in support of the idea that special classes of property, e.g. shares, are governed by the peculiar law they are associated with.)}\\ 
\textit{Solomons v Ross} (1764) 1 H Bl 131, 126 ER 79 & \small{CP} & \small{Property---Property Domicile---Bankruptcy---Assignment} & \small{A very short report annexed to another. It appears to suggest that a Dutch assignment in bankruptcy was effective as against a foreign debtor, and that an English court allowed certain matters to proceed before it in support.} & \small{Story Ed1 (at §395-400, in support of the general position that the assignment of debts should be governed by the domicile of the person holding the debt.)}\\ 
\textit{Neale v Cottingham} (1764) 1 H Bl 134, 126 ER 81 & \small{Ch-Ireland} & \small{Property---Assignment---Bankruptcy} & \small{Not much really to see in this case. It seems to stand for the very simple proposition that the Irish courts recognised an assignment by bankruptcy in England.} & \small{Story Ed1 (at §403-409, for the English rules on assignment in bankruptcy, suggesting that this is determined by one law being the law of domicile)}\\ 
\textit{Jollet v Deponthieu} (1769) 1 H Bl 131, 126 ER 79 & \small{Ch} & \small{Property---Assignment} & \small{Very hard decision to parse. It appears to relate to the terms of a Dutch bankruptcy on assets confiscated in America (we are told held by the state of New Jersey).} & \small{Story Ed1 (at §403-409, for the English rules on assignment in bankruptcy, suggesting that this is determined by one law being the law of domicile)}\\ 
\textit{ex parte Blakes} (1787) 1 Cox 398, 29 ER 1219 & \small{Ch} & \small{Property---Assignment---Bankruptcy} & \small{A very peculiar form of order, attempting to compel an affadavit to be presented in America so as to allow assigned debts to be claimed there (since the American courts would not recognise otherwise.)} & \small{Story Ed1 (at §403-409, for the English rules on assignment in bankruptcy, suggesting that this is determined by one law being the law of domicile)}\\ 
\textit{Bruce v Bruce} (1790) 2 B\&P 229, 126 ER 1251 & \small{HL} & \small{Property---Property Domicile} & \small{Applying the domicile of the deceased (Scotland) to determine succession issues when dead abroad (East Indes).} & \small{Story Ed1 (at §380-381, in support of the idea that movable property is determined by domicile)}\\ 
\textit{Hunter v Potts} (1791) 4 TR 182, 100 ER 962 & \small{KB} & \small{Property---Property Domicile---Bankruptcy} & \small{Lord Kenyon strongly favouring the application of the domicile of the owner of property as determining the manner and effectiveness of its transfer, English assignment in bankruptcy said to be effective over lands held abroad (in Rhode Island).} & \small{Story Ed1 (at §380-381, in support of the idea that movable property is determined by domicile); Story Ed1 (at §395-400, as showing the domicile of the owner of a debt as preferable for determining its assignment.)}\\ 
\textit{Sill v Worswick} (1791) 1 H Bl 655, 126 ER 379 & \small{CP} & \small{Property---Assignment---Bankruptcy---Property Domicile} & \small{A very complex case related to the assignment in England of certain debts in Scotland. Interestingly, the notion that these are governed by the lex domicilii does not appear to be settled.} & \small{Story Ed1 (at §380-381, in support of the idea that movable property is determined by domicile); Story Ed1 (at §395-400, as showing the domicile of the owner of a debt as preferable for determining its assignment.)}\\ 
\textit{Phillips v Hunter} (1795) 2 H Bl 402, 126 ER 618 & \small{CP} & \small{Property---Property Domicile---Bankruptcy} & \small{Assignment in Bankruptcy made in England effective against sums later recovered abroad (Pennsylvania). Case seems to be reasoned in terms of the effects and of the bankruptcy laws.} & \small{Story Ed1 (at §380-381, in support of the idea that movable property is determined by domicile)}\\ 
\textit{Smith v Buchanan} (1800) 1 East 6, 102 ER 3 & \small{KB} & \small{Contract---Contract Discharge---Bankruptcy---Application of Lex Loci Con---Property} & \small{Maryland discharge by bankruptcy held to not affect debts contracted for in England, with a strong preference for lex loci contractus “It is impossible to say that a contract made in one country is to be governed by the laws of another.” Ballatine v Golding distinguished.} & \small{Story Ed1 (§330-351, with approval); Westlake Ed1 (arts 235-256, with approval.); Story Ed1 (at §403-409, for the principles on the assignment of debts in bankruptcy.)}\\ 
\textit{Inglis v Underwood} (1801) 1 East 515, 102 ER 198 & \small{KB} & \small{Property} & \small{That assignees in bankruptcy took subject to a kind of lien, established in Russia, over a ship for mariner’s fees.} & \small{Story Ed1 (at §401-402, that interests established in one state persist in another)}\\ 
\textit{Nathan v Giles} (1814) 5 Taunt R 558, 128 ER 808 & \small{CP} & \small{Property---Property Protection of Third Parties---Assignment} & \small{Slightly unclear. It seems to be the case that an assignment made in Hamburg did not affect a lein that existed (it seems on a bill of lading?)} & \small{Story Ed1 (at §388-94, to suggest that preference might be made to the lex fori for the transfer of movable property where this has the effect of disadvantaging parties within the jurisdiction)}\\ 
\textit{Skelrig v Davis} (1814) 2 Dow 230, 3 ER 948 & \small{HL (SC)} & \small{Property---Property Domicile---Assignment---Bankruptcy---Property Real} & \small{English assignment in bankruptcy held effective over Scottish-held shares, without the formal requirements in Scotland. This case doesn’t seem to be reasoned on the terms that Story sets it out for. Some indication also that immovable property abroad cannot be assigned in bankruptcy.} & \small{Story Ed1 (at §395-400, in support of the general position that the assignment of debts should be governed by the domicile of the person holding the debt.); Story Ed1 (at §428, that immovable property is not subject to assignment in bankruptcy, creating only a moral obligation to convey title)}\\ 
\textit{Scott v Alnutt} (1831) 2 Dow \& Clark 404, 6 ER 778 & \small{HL (SC)} & \small{Property---Property Domicile---Assignment} & \small{Assignment of Scottish reversionary interest valid when done by English (but not Scottish) form. Holder of the interest appears to have been in England at the time.} & \small{Story Ed1 (at §395-400, in support of the general position that the assignment of debts should be governed by the domicile of the person holding the debt.)}\\ 
\textit{Bunbury v Bunbury} (1839) 1 Beav 318, 48 ER 963 & \small{Ch} & \small{Injunction---Property} & \small{An injunction granted to restrain proceedings to recover real estate in Demerara, apparently on the basis that several items of litigation were better dealt with together in England. Litigation appears to have arisen out of a marriage settlement and will by English domiciled parties, including both realty and personality. Master of the rolls expressly also notes the general position as regards movable (determined by lex domicili) and immovable (determined by lex situs) property.} & \small{Westlake Ed1 (arts 130-131, as illustrative of the principles governing the issue of anti-suit injunctions)}\\ 
\textit{The Johannes Christoph} (1854) 2 Sp Ecc \& Ad 93, 164 ER 325 & \small{Admiralty} & \small{Property---Application of Lex Fori} & \small{Freight was sold of a ship, which had a master and crew from Hamburg. This was in satisfaction of a number of claims, including salvage. The master claimed to be entitled to proceeds of the sale by virtue of a lien under Hamburg law for his wages and sundry other costs. This was not admitted. Dr Lushington, interestingly, phrases the position “in this court” (Admiralty) as being that foreign law is imported as a matter of discretion as regards the remedy in an action. Huber is cited in support of the proposition. Perhaps this relates to the particular understanding of the court of admiralty at this time?} &  \\ 
\textit{Simpson v Fogo} (1863) 1 H\&M 195, 71 ER 85 & \small{VC} & \small{Property---Property Protection of Third Parties---Application of Lex Fori---Foreign Judgements} & \small{A very full decision, and worth re-reading. A ship was mortgaged in Liverpool, and then taken to New Orleans. Another creditor of the ship owner, in effect, compelled her compulsory sale. Proceedings then commenced in Louisiana as to the relative interests of the different creditors. The Mortagees (the Bank of Liverpool) made representations before the court there, but they lost rather badly: the effect of the Louisiana court’s ruling was that their interest in the ship was extinguished. These proceedings started when the new owners took the ship back to Liverpool. The Louisiana Court’s ruling was effectively disregarded, leaving the Bank of Liverpool’s interests intact – the reasoning being that the ruling was contrary to the law of nations and should not be recognised. Some interesting point that emerge include a debate as to whether the lex loci contractus, lex domicili, or lex rei sitae applies to the determination of title to movable property, as well as the application of the lex fori to determine questions of priority.} &  \\ 
\end{longtable}\subsection{Real Property (usually lex situs)}

These are cases that in some level deal with real property. By in large, they resolve issues with reference to the \textit{lex rei sitae.}
\\ 
\begin{longtable}{p{4cm} p{1.7cm} p{2.5cm} p{5cm} p{2.5cm}}
\hline
Name & Court & Subject(s) & Comment & Cited By\\
\hline\textit{Coppin v Coppin} (1725) 2 P Wms 291, 24 ER 735 & \small{Ch} & \small{Property Real---Formality} & \small{Lands held in England could not be devised by insufficent form in will made “beyond the sea” (it seems in Persia by a member of the EIC).} & \small{Story Ed1 (at §435-444, for the preference of the English courts for application of lex situs for immovable property and formality)}\\ 
\textit{Skelrig v Davis} (1814) 2 Dow 230, 3 ER 948 & \small{HL (SC)} & \small{Property---Property Domicile---Assignment---Bankruptcy---Property Real} & \small{English assignment in bankruptcy held effective over Scottish-held shares, without the formal requirements in Scotland. This case doesn’t seem to be reasoned on the terms that Story sets it out for. Some indication also that immovable property abroad cannot be assigned in bankruptcy.} & \small{Story Ed1 (at §395-400, in support of the general position that the assignment of debts should be governed by the domicile of the person holding the debt.); Story Ed1 (at §428, that immovable property is not subject to assignment in bankruptcy, creating only a moral obligation to convey title)}\\ 
\textit{Birthwhistle v Vardill} (1826) 5 B\&C 438, 108 ER 163 & \small{KB} & \small{Property Real} & \small{A child was born out of Wedlock in Scotland (and later domiciled there) to parents who later married in England. By the laws of Scotland he could inherit lands, but not English law. He could not inherit lands in England. Specific distinction drawn between movable property (governed by the lex domicili) and immovable property, which is governed by the law of where it is situate.} & \small{Story Ed1 (at §428-434, for the strong preference on English courts in looking to the lex situs for capacity in dealing with immovable property)}\\ 
\textit{Dundas v Dundas} (1830) 2 Dow \& Cl 349, 6 ER 757 & \small{HL (SC)} & \small{Property Real} & \small{Lands held in England could not be devised by a Scotch trust deed that did not meet the English requirements of the Statute of Frauds.} & \small{Story Ed1 (at §435-444, for the preference of the English courts for application of lex situs for immovable property and formality)}\\ 
\textit{Attorney General v Mill} (1831) 2 Dow \& Cl 393, 6 ER 774 & \small{HL} & \small{Property Real} & \small{A divestment of lands that was in violation of the statutes of mortmain was invalid, despite the fact that the testator was domiciled in Scotland. Will was made of an English form and in England. The actual reasoning of the case is really in terms of the construction of the will and the statutes of mortmain, the “conflicts” issues are not really reasoned as such,} & \small{Story Ed1 (at §445-446, for the lex situs rule being used to determine the extent of interests in immovable property, specifically related to the application of mortmain statutes in England)}\\ 
\end{longtable}\subsection{Movable Property (usually domicile)}

These are the cases that refer to movable property, and, usually, specifically the application of the law of the domicile being the relevant rule.
\\ 
\begin{longtable}{p{4cm} p{1.7cm} p{2.5cm} p{5cm} p{2.5cm}}
\hline
Name & Court & Subject(s) & Comment & Cited By\\
\hline\textit{Morrisons Case} (1749) 1 H Bl 677, 126 ER 385 & \small{CtS} & \small{Property---Property Domicile---Assignment---Bankruptcy} & \small{Very hard to tell much from the decision itself (we are only getting it second hand.) It seems to reflect a general notion that the authority assign is granted to the English court of bankruptcy, but not particular mention is made of domicile.} & \small{Story Ed1 (at §395-400, in support of the general position that the assignment of debts should be governed by the domicile of the person holding the debt.)}\\ 
\textit{Solomons v Ross} (1764) 1 H Bl 131, 126 ER 79 & \small{CP} & \small{Property---Property Domicile---Bankruptcy---Assignment} & \small{A very short report annexed to another. It appears to suggest that a Dutch assignment in bankruptcy was effective as against a foreign debtor, and that an English court allowed certain matters to proceed before it in support.} & \small{Story Ed1 (at §395-400, in support of the general position that the assignment of debts should be governed by the domicile of the person holding the debt.)}\\ 
\textit{Bruce v Bruce} (1790) 2 B\&P 229, 126 ER 1251 & \small{HL} & \small{Property---Property Domicile} & \small{Applying the domicile of the deceased (Scotland) to determine succession issues when dead abroad (East Indes).} & \small{Story Ed1 (at §380-381, in support of the idea that movable property is determined by domicile)}\\ 
\textit{Hunter v Potts} (1791) 4 TR 182, 100 ER 962 & \small{KB} & \small{Property---Property Domicile---Bankruptcy} & \small{Lord Kenyon strongly favouring the application of the domicile of the owner of property as determining the manner and effectiveness of its transfer, English assignment in bankruptcy said to be effective over lands held abroad (in Rhode Island).} & \small{Story Ed1 (at §380-381, in support of the idea that movable property is determined by domicile); Story Ed1 (at §395-400, as showing the domicile of the owner of a debt as preferable for determining its assignment.)}\\ 
\textit{Sill v Worswick} (1791) 1 H Bl 655, 126 ER 379 & \small{CP} & \small{Property---Assignment---Bankruptcy---Property Domicile} & \small{A very complex case related to the assignment in England of certain debts in Scotland. Interestingly, the notion that these are governed by the lex domicilii does not appear to be settled.} & \small{Story Ed1 (at §380-381, in support of the idea that movable property is determined by domicile); Story Ed1 (at §395-400, as showing the domicile of the owner of a debt as preferable for determining its assignment.)}\\ 
\textit{Phillips v Hunter} (1795) 2 H Bl 402, 126 ER 618 & \small{CP} & \small{Property---Property Domicile---Bankruptcy} & \small{Assignment in Bankruptcy made in England effective against sums later recovered abroad (Pennsylvania). Case seems to be reasoned in terms of the effects and of the bankruptcy laws.} & \small{Story Ed1 (at §380-381, in support of the idea that movable property is determined by domicile)}\\ 
\textit{Skelrig v Davis} (1814) 2 Dow 230, 3 ER 948 & \small{HL (SC)} & \small{Property---Property Domicile---Assignment---Bankruptcy---Property Real} & \small{English assignment in bankruptcy held effective over Scottish-held shares, without the formal requirements in Scotland. This case doesn’t seem to be reasoned on the terms that Story sets it out for. Some indication also that immovable property abroad cannot be assigned in bankruptcy.} & \small{Story Ed1 (at §395-400, in support of the general position that the assignment of debts should be governed by the domicile of the person holding the debt.); Story Ed1 (at §428, that immovable property is not subject to assignment in bankruptcy, creating only a moral obligation to convey title)}\\ 
\textit{Scott v Alnutt} (1831) 2 Dow \& Clark 404, 6 ER 778 & \small{HL (SC)} & \small{Property---Property Domicile---Assignment} & \small{Assignment of Scottish reversionary interest valid when done by English (but not Scottish) form. Holder of the interest appears to have been in England at the time.} & \small{Story Ed1 (at §395-400, in support of the general position that the assignment of debts should be governed by the domicile of the person holding the debt.)}\\ 
\end{longtable}
\section{Particular Contexts}\
\subsection{Bills of Exchange}

The following are all of the cases (in a simple list) related to bills of exchange (on their facts). One will see that there are quite a number of them.
\\ 
\begin{enumerate}
\item{\textit{Van Heath v Turner} (1621) Winch 23, 124 ER 20}
\item{\textit{Gold v Canahan} (1679) 2 Swans 326, 36 ER 640}
\item{\textit{Magadra v Holt} (1691) 1 Show 318, 89 ER 597}
\item{\textit{Williams v Williams} (1693) Carth 268, 90 ER 759}
\item{\textit{Bromwhich v Loyd} (1699) 2 Lutw 1582, 125 ER 870}
\item{\textit{Burroughs v Jamineau} (1726) 2 Str 733, 25 ER 235 93 ER 815}
\item{\textit{Robinson v Bland} (1760) 2 Bur 1077, 97 ER 717}
\item{\textit{Alves v Hodgson} (1797) 7 TR 241, 101 ER 953}
\item{\textit{Potter v Brown} (1804) 5 East 124, 102 ER 1016}
\item{\textit{Snaith v Mingay} (1813) 1 M\&S 87, 105 ER 33}
\item{\textit{Boehm v Campbell} (1818) Gow 56, 8 Taun 679, 171 ER 837, 129 ER 548}
\item{\textit{Wynne v Jackson} (1826) 2 Russ 351, 38 ER 368}
\item{\textit{Wynne v Callander} (1826) 1 Russ 293, 38 ER 113}
\item{\textit{Bentley v Northouse} (1827) M\&M 66, 173 ER 1083}
\item{\textit{Phillips v Allan} (1828) 7 B\&C 477, 108 ER 1120}
\item{\textit{De La Chaumette v Bank of England} (1829) 9 B\&C 208, 109 ER 78}
\item{\textit{Holdsworth v Hunter} (1830) 10 B \& C 449, 109 ER 517}
\item{\textit{Novelli v Rossi} (1831) 2 B \& Ad, 109 ER 1326}
\item{\textit{Portarlington v Soulby} (1834) 3 My \& Ke 104, 40 ER 40}
\item{\textit{Huber v Steiner} (1835) 2 Bing NC 202, 132 ER 80}
\item{\textit{Don v Lippmann} (1837) 5 Cl\& F 1, 7 ER 303}
\item{\textit{Abrahams v Skinner} (1840) 12 A\&E 763, 113 ER 1003}
\item{\textit{Cooper v Waldegrave} (1840) 2 Beav 282, 48 ER 1189}
\item{\textit{Rothschild v Currie} (1841) 1 QB 43, 113 ER 1045}
\item{\textit{Bartlett v Smith} (1843) 11 M\&W 483, 152 ER 895}
\item{\textit{Steadman v Duhamel} (1845) 1 CB 888, 135 ER 792}
\item{\textit{Gibbs v Fremont} (1853) 9 Exch 25, 156 ER 11}
\item{\textit{Sharples v Rickard} (1857) 2 H\&N 57, 157 ER 24}
\item{\textit{Scott v Pilkington} (1862) 2 B\&S 11, 121 ER 978}
\end{enumerate}
\subsection{Sale of Goods}

The following are all of the cases (in a simple list) related on some level to sale of goods. A couple of these, it should be noted (problematically) include \textbf{slaves}.
\\ 
\begin{enumerate}
\item{\textit{Smith v Brown \& Cooper} (1706) 2 Salk 665, 91 ER 566}
\item{\textit{Holman v Johnson} (1775) 1 Cowp 342, 98 ER 1120}
\item{\textit{Planché v Fletcher} (1779) 1 Doug 251, 99 ER 164}
\item{\textit{Dewar v Span} (1789) 3 TR 425, 100 ER 656}
\item{\textit{Biggs v Lawrence} (1789) 3 TR 454, 100 ER 673}
\item{\textit{Clugas v Penaluna} (1791) 4 TR 466, 100 ER 1122}
\item{\textit{Waymell v Reed} (1794) 5 TR 599, 101 ER 335}
\item{\textit{Clegg v Levy} (1812) 3 Camp 166, 170 ER 1343}
\item{\textit{Acebal v Levy} (1834) 10 Bing 376, 131 ER 949}
\item{\textit{Pellecat v Angell} (1835) 2 CM\&R 312, 150 ER 135}
\item{\textit{Heriz v Riera} (1840) 1 Sim 318, 59 ER 896}
\item{\textit{Bristow v Sequeville} (1850) 5 Exch 279, 155 ER 118}
\end{enumerate}
\subsection{Marriage}

These are all of the cases related to marriage on some level.
\\ 
\begin{enumerate}
\item{\textit{Cottington’s Case} (1678) 2 Swans 326, 36 ER 640}
\item{\textit{Foubert v Turst} (1703) 1 Brown Parl Cas,, 1 ER 464,  24 ER 101}
\item{\textit{Tremoult v Dedire} (1718) 1 P Wms 429, 24 ER 458}
\item{\textit{Phipps v Earl of Angelsea} (1721) 1 P Wms 697, 5 Bro PC 45, 24 ER 576}
\item{\textit{Dalrymple v Dalrymple} (1811) 2 Hag Con 54, 161 ER 665}
\item{\textit{Ruding v Smith} (1821) 2 Hag Con 371, 161 ER 774}
\item{\textit{Hope v Hope} (1856) 2 Beav 351, 52 ER 1143}
\item{\textit{Brook v Brook} (1858) 3 Sm \& Grif 481, 65 ER 746}
\item{\textit{Brook v Brook (no 2)} (1861) 9 HLC 193, 11 ER 703}
\item{\textit{In re Melbourn (no 1)} (1871) LR Ch App 64}
\end{enumerate}
\section{Cases Using Certain Key Terms}

These are cases that use certain key terms, such as \textit{lex situs} or \textit{lex loci contractus} -- either by counsel (which is the most common) or by the bench itself. Noting when and how these arise is interesting for tracking the absortion of conflict-of-laws terminology in the English cases.
\\ 
\begin{enumerate}
\item{\textit{Magadra v Holt} (1691) 1 Show 318, 89 ER 597 [ius gentium]}
\item{\textit{Mostyn v Fabrigas} (1774) 1 Cowp 161, 98 ER 1021 [lex loci]}
\item{\textit{Bruce v Bruce} (1790) 2 B\&P 229, 126 ER 1251 [domicilum]}
\item{\textit{Power v Whitmore} (1815) 4 M\&S 141, 105 ER 787 [comity, law of nations]}
\item{\textit{Wolf v Oxholm} (1817) 7 M\&S 92, 105 ER 1177 [law of nations]}
\item{\textit{Ruding v Smith} (1821) 2 Hag Con 371, 161 ER 774 [lex loci]}
\item{\textit{Birthwhistle v Vardill} (1826) 5 B\&C 438, 108 ER 163 [lex domicilii]}
\item{\textit{Scott v Alnutt} (1831) 2 Dow \& Clark 404, 6 ER 778 [lex domicilii]}
\item{\textit{De Wutz v Hendricks} (1834) 2 Bing 314, 130 ER 326 [law of nations]}
\item{\textit{Trimbey v Vignier} (1834) 1 Bing NC 151, 131 ER 1075 [lex loci contractus]}
\item{\textit{Huber v Steiner} (1835) 2 Bing NC 202, 132 ER 80 [lex loci contractus, lex fori]}
\item{\textit{Don v Lippmann} (1837) 5 Cl\& F 1, 7 ER 303 [lex fori, lex loci contractus, lex loci solutionis]}
\item{\textit{Bent v Young} (1838) 9 Sim 180, 59 ER 327 [lex loci rei sitae]}
\item{\textit{Bunbury v Bunbury} (1839) 1 Beav 318, 48 ER 963 [lex loci contractus]}
\item{\textit{Rothschild v Currie} (1841) 1 QB 43, 113 ER 1045 [lex loci contractus]}
\item{\textit{General Steam Navigation Co v Guillou} (1843) 11 Mee \& Wel 877, 152 ER 1061 [lex fori]}
\item{\textit{Allen v Kemble} (1848) 6 Moo PC 321, 13 ER 704 [lex loci solutionis, lex loci contractus]}
\item{\textit{Gibbs v Fremont} (1853) 9 Exch 25, 156 ER 11 [lex loci contractus, lex loci solutionis]}
\item{\textit{The Johannes Christoph} (1854) 2 Sp Ecc \& Ad 93, 164 ER 325 [lex loci contractus, lex fori]}
\item{\textit{Hope v Hope} (1856) 2 Beav 351, 52 ER 1143 [lex loci contractus]}
\item{\textit{Brook v Brook (no 2)} (1861) 9 HLC 193, 11 ER 703 [lex loci contractus, lex domicilii]}
\item{\textit{MacFarlane v Norris} (1862) 2 B\&S 783, 121 ER 1263 [lex fori, lex loci contractus]}
\item{\textit{Scott v Pilkington} (1862) 2 B\&S 11, 121 ER 978 [lex loci contractus, comity]}
\item{\textit{Simpson v Fogo} (1863) 1 H\&M 195, 71 ER 85 [lex fori, lex loci contractus, lex domicilii, lex rei sitae]}
\item{\textit{In re Melbourn (no 1)} (1871) LR Ch App 64 [lex loci contractus, lex fori]}
\end{enumerate}
\section{Cases Citing Certain Important Writers}

These are all of the cases that themselves show citation to (either by counsel or the bench) some continental writer.
\\ 
\begin{enumerate}
\item{\textit{Robinson v Bland} (1760) 2 Bur 1077, 97 ER 717 [Huber, Voet.]}
\item{\textit{Holman v Johnson} (1775) 1 Cowp 342, 98 ER 1120 [Huber.]}
\item{\textit{Hunter v Potts} (1791) 4 TR 182, 100 ER 962 [Voet.]}
\item{\textit{Dalrymple v Dalrymple} (1811) 2 Hag Con 54, 161 ER 665 [Huber, Voet.]}
\item{\textit{Wolf v Oxholm} (1817) 7 M\&S 92, 105 ER 1177 [Vattel, Grotius, Puffendorf, Bynkerschoek.]}
\item{\textit{Ruding v Smith} (1821) 2 Hag Con 371, 161 ER 774 [Huber.]}
\item{\textit{Birthwhistle v Vardill} (1826) 5 B\&C 438, 108 ER 163 [Huber.]}
\item{\textit{De la Vega v Vianna} (1830) 1 B \& Ad 284, 109 ER 792 [Huber, Voet.]}
\item{\textit{Scott v Alnutt} (1831) 2 Dow \& Clark 404, 6 ER 778 [Voet, Erskine.]}
\item{\textit{Trimbey v Vignier} (1834) 1 Bing NC 151, 131 ER 1075 [Huber.]}
\item{\textit{Huber v Steiner} (1835) 2 Bing NC 202, 132 ER 80 [Huber, Story.]}
\item{\textit{Don v Lippmann} (1837) 5 Cl\& F 1, 7 ER 303 [Huber, Story, Voet.]}
\item{\textit{Caldwell v Vanvlissengen} (1851) 9 Hare 415, 68 ER 571 [Huber, Vattel, Story, Boullenois.]}
\item{\textit{Leroux v Brown} (1852) 12 CB 801, 138 ER 1119 [Huber, Story, Burge.]}
\item{\textit{The Johannes Christoph} (1854) 2 Sp Ecc \& Ad 93, 164 ER 325 [Huber.]}
\item{\textit{Brook v Brook} (1858) 3 Sm \& Grif 481, 65 ER 746 [Huber, Story, Voet, Sanchez, Gayll.]}
\item{\textit{Brook v Brook (no 2)} (1861) 9 HLC 193, 11 ER 703 [Huber, Story.]}
\item{\textit{MacFarlane v Norris} (1862) 2 B\&S 783, 121 ER 1263 [Huber, Story.]}
\item{\textit{Simpson v Fogo} (1863) 1 H\&M 195, 71 ER 85 [Huber, Story, Burge.]}
\end{enumerate}
\newpage\section{Chronological List of All Cases}

The following is a simple table containing all of the cases, sorted by date. Each entry includes all the relevant obtained information.
\\ 


        \begin{small}
        \begin{center}
        \href{https://heinonline.org/HOL/P?h=hein.engrep/engrg0123&i=789}{\textit{Anon} (1611) 2 B\&G 10, 123 ER 785} \label{5} \\ 
\textit{ (Contract---Non Performance)}\\
        \end{center}
        \textbf{Admiralty}. Showing a fraught understanding of the jurisdiction of Admiralty for contracts “beyond the sea” (referring to France) but not on the deep sea. See Sack (1937)\\\\No known authors cited.
        \end{small}\\
        \rule{\textwidth}{0.5pt}
        

        \begin{small}
        \begin{center}
        \href{https://heinonline.org/HOL/P?h=hein.engrep/engrg0124&i=24}{\textit{Van Heath v Turner} (1621) Winch 23, 124 ER 20} \label{3} \\ 
\textit{Law Merchant (Bill of Exchange)}\\
        \end{center}
        \textbf{CP}. Report is in French. It is discussed in Sack (1937)\\\\No known authors cited.
        \end{small}\\
        \rule{\textwidth}{0.5pt}
        

        \begin{small}
        \begin{center}
        \href{https://heinonline.org/HOL/P?h=hein.engrep/engrf0081&i=937}{\textit{Slane \& Colbery v Ralph Cotton} (1625) 2 Rolle 486, 81 ER 933} \label{6} \\ 
\textit{Admiralty Jurisdiction (Contract---Carriage)}\\
        \end{center}
        \textbf{Admiralty}. Report is in French. It is discussed in Sack (1937)\\\\No known authors cited.
        \end{small}\\
        \rule{\textwidth}{0.5pt}
        

        \begin{small}
        \begin{center}
        \href{https://heinonline.org/HOL/P?h=hein.engrep/engrf0083&i=343}{\textit{Beven v Clapham} (1664) Lev 143, 83 ER 339} \label{15} \\ 
\textit{ (Assumpsit---Contract)}\\
        \end{center}
        \textbf{KB}. Claim clearly pleaded fictionally “Tenerif, in the Ward of Cheap”, but the statute of limitations said not to extent to it.\\\\No known authors cited.
        \end{small}\\
        \rule{\textwidth}{0.5pt}
        

        \begin{small}
        \begin{center}
        \href{https://heinonline.org/HOL/P?h=hein.engrep/engrc0036&i=649}{\textit{Cottington’s Case} (1678) 2 Swans 326, 36 ER 640} \label{8} \\ 
\textit{Foreign Judgements (Marriage Divorce)}\\
        \end{center}
        \textbf{HL}. The clear idea is presented that foreign judgement (related to nullity of marriage) from Turin cannot be examined by an English court.\\\\No known authors cited.
        \end{small}\\
        \rule{\textwidth}{0.5pt}
        

        \begin{small}
        \begin{center}
        \href{https://heinonline.org/HOL/P?h=hein.engrep/engrc0036&i=649}{\textit{Gold v Canahan} (1679) 2 Swans 326, 36 ER 640} \label{7} \\ 
\textit{ (Bill of Exchange---Partnership)}\\
        \end{center}
        \textbf{HL}. A very brief report. The interesting detail is that the “justice” of the Florentine judgement “is not examinable here.” There appears to be some attempt to indemnify. No clear discussion of “choice of law.”\\\\No known authors cited.
        \end{small}\\
        \rule{\textwidth}{0.5pt}
        

        \begin{small}
        \begin{center}
        \href{https://heinonline.org/HOL/P?h=hein.engrep/engrf0089&i=601}{\textit{Magadra v Holt} (1691) 1 Show 318, 89 ER 597} \label{1} \\ 
\textit{ (Bill of Exchange)}\\
        \end{center}
        \textbf{KB}.  \textbf{Uses terms: }[\textit{ius gentium}]. An early instance of simple application of the “law merchant.”\\\\No known authors cited.
        \end{small}\\
        \rule{\textwidth}{0.5pt}
        

        \begin{small}
        \begin{center}
        \href{https://heinonline.org/HOL/P?h=hein.engrep/engrf0090&i=763}{\textit{Williams v Williams} (1693) Carth 268, 90 ER 759} \label{4} \\ 
\textit{Law Merchant (Bill of Exchange)}\\
        \end{center}
        \textbf{KB}. Noteworthy for the way it treats law merchant as part of the law of England. Can still see use of fiction “Mariae de Arcubus in Warda de Cheap.”\\\\No known authors cited.
        \end{small}\\
        \rule{\textwidth}{0.5pt}
        

        \begin{small}
        \begin{center}
        \href{https://heinonline.org/HOL/P?h=hein.engrep/engrf0091&i=361}{\textit{Blankard v Goldy} (1693) 2 Salk 411, 91 ER 356} \label{16} \\ 
\textit{Application of English Statute (Contract---Illegality)}\\
        \end{center}
        \textbf{KB}. Seeming to see Jamaica, as an “uninhabited country” as taking on the law of England, but not a particular statute forbidding purchasing of offices.\\\\No known authors cited.
        \end{small}\\
        \rule{\textwidth}{0.5pt}
        

        \begin{small}
        \begin{center}
        \href{https://heinonline.org/HOL/P?h=hein.engrep/engrg0125&i=874}{\textit{Bromwhich v Loyd} (1699) 2 Lutw 1582, 125 ER 870} \label{2} \\ 
\textit{Law Merchant (Bill of Exchange)}\\
        \end{center}
        \textbf{KB}. Another instance of application of the “law merchant.”\\\\No known authors cited.
        \end{small}\\
        \rule{\textwidth}{0.5pt}
        

        \begin{small}
        \begin{center}
        \href{https://heinonline.org/HOL/P?h=hein.engrep/engrc0023&i=863}{\textit{Ranelaugh v Champante} (1700) 2 Vern 395, 23 ER 855} \label{17} \\ 
\textit{Contract---Application of English Statute (Bond---Real Estate)}\\
        \end{center}
        \textbf{Ch}. Bond for a debt “in Ireland” executed in England, leading to the application of English interest. There is a hint of a lex fori rule.\\\\No known authors cited.
        \end{small}\\
        \rule{\textwidth}{0.5pt}
        

        \begin{small}
        \begin{center}
        \href{https://heinonline.org/HOL/P?h=hein.engrep/engrc0023&i=863}{\textit{Dungannon v Hackett} (1702) Eq Ca Abr 289, 23 ER 855} \label{18} \\ 
\textit{Contract Interest---Contract---Application of Lex Loci Con (Debt---Contract)}\\
        \end{center}
        \textbf{Ch}. Implication is that the interest should be determined by the place where it was contracted for.\\\\No known authors cited.
        \end{small}\\
        \rule{\textwidth}{0.5pt}
        

        \begin{small}
        \begin{center}
        \href{https://heinonline.org/HOL/P?h=hein.engrep/engra0001&i=472}{\textit{Foubert v Turst} (1703) 1 Brown Parl Cas,, 1 ER 464,  24 ER 101} \label{27} \\ 
\textit{Contract---Contract Intention---Lex Loci Con (Marriage Contract)}\\
        \end{center}
        \textbf{HL}. French Marriage Contract affirmed, by its express terms, to refer to the custom of Paris. Actual reasoning unclear, but it seems to be based on ideas of intent.\\\\No known authors cited.
        \end{small}\\
        \rule{\textwidth}{0.5pt}
        

        \begin{small}
        \begin{center}
        \href{https://heinonline.org/HOL/P?h=hein.engrep/engrf0087&i=952}{\textit{Wey v Rally} (1705) 2 Salk 651 6 Mod 194, 87 ER 948} \label{14} \\ 
\textit{ (Rent on Lands)}\\
        \end{center}
        \textbf{KB}.  \textbf{Uses terms: }[\textit{privity of estate, privity of contract}]. Claims for rents of lands in Jamaica was a transitory and not local action.\\\textit{Cited in: }Westlake Ed1 (At arts 120-122, in connection to “local” and “transitory” actions and the rules of jurisdiction)\\No known authors cited.
        \end{small}\\
        \rule{\textwidth}{0.5pt}
        

        \begin{small}
        \begin{center}
        \href{https://heinonline.org/HOL/P?h=hein.engrep/engrf0091&i=570}{\textit{Smith v Brown \& Cooper} (1706) 2 Salk 665, 91 ER 566} \label{29} \\ 
\textit{Contract---Illegality (Sale---Slaves)}\\
        \end{center}
        \textbf{KB}. A very short report. The issue appears to have been the allegation that slaves were sold in London (where no notice could be taken of them), and that the plea should have been that the contract was in London but the slave in Virginia.\\\textit{Cited in: }Westlake Ed1 (arts 192-200, somewhat critically, in connection to non application of morally repugnant laws for the consideration of contracts.)\\No known authors cited.
        \end{small}\\
        \rule{\textwidth}{0.5pt}
        

        \begin{small}
        \begin{center}
        \href{https://heinonline.org/HOL/P?h=hein.engrep/engrc0024&i=449}{\textit{Ekins v East-India Co} (1717) 1 P Wms 394, 24 ER 441} \label{19} \\ 
\textit{Contract Interest---Contract---Application of Lex Loci Con (Tresspass---Goods Taken)}\\
        \end{center}
        \textbf{Ch}. Action of taking and selling goods in India carried Indian interest. Not be a choice of law rule “must be presumed to have common advantage” of money there.\\\textit{Cited in: }Westlake Ed1 (arts 230-236 for breach of obligations)\\No known authors cited.
        \end{small}\\
        \rule{\textwidth}{0.5pt}
        

        \begin{small}
        \begin{center}
        \href{https://heinonline.org/HOL/P?h=hein.engrep/engrc0024&i=466}{\textit{Tremoult v Dedire} (1718) 1 P Wms 429, 24 ER 458} \label{28} \\ 
\textit{Contract---Contract Law of Terms---Marriage---Contract Intention---Lex Loci Con (Marriage Contract)}\\
        \end{center}
        \textbf{Ch}. Clear implication that Dutch marriage articles could be construed and applied according to the laws of Holland. Evidence of this is required “to take notice of foreign laws” (contrast with Foubert). Unclear what the basis of this is, though it seems assumed.\\\\No known authors cited.
        \end{small}\\
        \rule{\textwidth}{0.5pt}
        

        \begin{small}
        \begin{center}
        \href{https://heinonline.org/HOL/P?h=hein.engrep/engrc0024&i=584}{\textit{Phipps v Earl of Angelsea} (1721) 1 P Wms 697, 5 Bro PC 45, 24 ER 576} \label{31} \\ 
\textit{Contract---Lex Loci Con (Marriage Settlement)}\\
        \end{center}
        \textbf{Ch}. English interest, as this was the place where the contract was made (and where it was to be performed.) Very little specific reasoning on the issue.\\\textit{Cited in: }Story Ed1 (Cited at §279-290 to say that the lex loci rule is not circumvented by the location of the security)\\No known authors cited.
        \end{small}\\
        \rule{\textwidth}{0.5pt}
        

        \begin{small}
        \begin{center}
        \href{https://heinonline.org/HOL/P?h=hein.engrep/engrc0024&i=660}{\textit{Wallis v Brightwell} (1722) 2 P Wms 87, 24 ER 652} \label{23} \\ 
\textit{Contract (Wills)}\\
        \end{center}
        \textbf{Ch}. An annuity (paid and made in England) out of lands in Ireland. “English money” owed, intention the guiding factor (looking at place of contracting and performance.)\\\\No known authors cited.
        \end{small}\\
        \rule{\textwidth}{0.5pt}
        

        \begin{small}
        \begin{center}
        \href{https://heinonline.org/HOL/P?h=hein.engrep/engrc0024&i=743}{\textit{Coppin v Coppin} (1725) 2 P Wms 291, 24 ER 735} \label{135} \\ 
\textit{Property Real---Formality (Succession)}\\
        \end{center}
        \textbf{Ch}. Lands held in England could not be devised by insufficent form in will made “beyond the sea” (it seems in Persia by a member of the EIC).\\\textit{Cited in: }Story Ed1 (at §435-444, for the preference of the English courts for application of lex situs for immovable property and formality)\\No known authors cited.
        \end{small}\\
        \rule{\textwidth}{0.5pt}
        

        \begin{small}
        \begin{center}
        \href{https://heinonline.org/HOL/P?h=hein.engrep/engrc0025&i=243, https://heinonline.org/HOL/P?h=hein.engrep/engrf0093&i=819}{\textit{Burroughs v Jamineau} (1726) 2 Str 733, 25 ER 235 93 ER 815} \label{10} \\ 
\textit{Contract---Foreign Judgements (Bill of Exchange)}\\
        \end{center}
        \textbf{Ch}. A bill of exchange that was discharged by the law of Livorno could not be sued for in England. Clearly of the view that it had to be determined by the place where the bill was negotiated. Injunction granted to prevent suing on the bill.\\\textit{Cited in: }Westlake Ed1 (with apparent approval, arts 225-228 in connection to international law of obligations); Story Ed1 (At §263-266, for the substantive requirements of the contract and the lex loci contractus.)\\No known authors cited.
        \end{small}\\
        \rule{\textwidth}{0.5pt}
        

        \begin{small}
        \begin{center}
        \href{https://heinonline.org/HOL/P?h=hein.engrep/engrc0026&i=639}{\textit{Connor v Bellamont} (1742) 2 Atk 382, 26 ER 631} \label{20} \\ 
\textit{Contract Interest---Contract (Bond---Real Estate)}\\
        \end{center}
        \textbf{Ch}. Debt contracted for in England, but bond taken out for its enforcement in Ireland – leading to Irish interest being applied. Appears to be a circumstantial test, place of security not sufficient – but currency and other factors enough.\\\textit{Cited in: }Story Ed1 (§291-298, on the rules for interest to suggest performance of place of contract unless performance was due elsewhere)\\No known authors cited.
        \end{small}\\
        \rule{\textwidth}{0.5pt}
        

        \begin{small}
        \begin{center}
        \href{https://heinonline.org/HOL/P?h=hein.engrep/engrc0026&i=689}{\textit{Saunders v Drake} (1742) 2 Atk 465, 26 ER 681} \label{24} \\ 
\textit{Contract Currency---Contract---Application of Lex Loci Con---Contract Intention (Wills)}\\
        \end{center}
        \textbf{Ch}. “Jamaican Money” applied to testators estate, with strong reliance on intention of the parties.\\\\No known authors cited.
        \end{small}\\
        \rule{\textwidth}{0.5pt}
        

        \begin{small}
        \begin{center}
        \href{https://heinonline.org/HOL/P?h=hein.engrep/engrg0126&i=389}{\textit{Morrisons Case} (1749) 1 H Bl 677, 126 ER 385} \label{127} \\ 
\textit{Property---Property Domicile---Assignment---Bankruptcy (Lunacy---Bankruptcy)}\\
        \end{center}
        \textbf{CtS}. Very hard to tell much from the decision itself (we are only getting it second hand.) It seems to reflect a general notion that the authority assign is granted to the English court of bankruptcy, but not particular mention is made of domicile.\\\textit{Cited in: }Story Ed1 (at §395-400, in support of the general position that the assignment of debts should be governed by the domicile of the person holding the debt.)\\No known authors cited.
        \end{small}\\
        \rule{\textwidth}{0.5pt}
        

        \begin{small}
        \begin{center}
        \href{https://heinonline.org/HOL/P?h=hein.engrep/engrc0027&i=1130}{\textit{Stapleton v Conway} (1750) 1 Ves Sen 427, 27 ER 1122} \label{25} \\ 
\textit{Contract Interest---Contract---Application of English Statute---Application of Lex Fori (Wills)}\\
        \end{center}
        \textbf{Ch}. Interest on charge of lands in Nevis. West Indian interest refused, with strong reliance of the potential for avoidance of usuary laws.  This is said to rely on a kind of “discretion,” as opposed to where a contract was made in England or America. (Hinting at a choice of law idea.)\\\textit{Cited in: }Story Ed1 (§291-298, on the rules for interest to suggest performance of place of contract unless performance was due elsewhere)\\No known authors cited.
        \end{small}\\
        \rule{\textwidth}{0.5pt}
        

        \begin{small}
        \begin{center}
        \href{https://heinonline.org/HOL/P?h=hein.engrep/engrf0097&i=721}{\textit{Robinson v Bland} (1760) 2 Bur 1077, 97 ER 717} \label{50} \\ 
\textit{Contract---Contract Validity---Application of Lex Loci Sol---Application of Lex Fori---Illegality---Property (Bill of Exchange---Gambling)}\\
        \end{center}
        \textbf{KB}. An obviously very significant case. the defendant -- having engaged in gambling in Paris -- entered into a Bill of Exchange for £300 with the plaintiffs, payable on 10 days of sight in England. The plaintiffs then came to London to recover, and failed, on the basis that the bill of exchange was void by the laws of England. Lord Mansfield relies on Huber to give a rule other than lex loci contractus. The other judges simply apply the lex fori.\\\textit{Cited in: }Story Ed1 (Cited with approval at §279-290, to suggest a potential exception to lex loci contractus); Westlake Ed1 (Critically, at arts 129-200, in connection to a discussion of illegality.); Story Ed1 (Cited at §383 in support of the idea that special classes of property, e.g. shares, are governed by the peculiar law they are associated with.)\\\textit{Authors refered to: }Huber, Voet.
        \end{small}\\
        \rule{\textwidth}{0.5pt}
        

        \begin{small}
        \begin{center}
        \href{https://heinonline.org/HOL/P?h=hein.engrep/engrg0126&i=83}{\textit{Solomons v Ross} (1764) 1 H Bl 131, 126 ER 79} \label{124} \\ 
\textit{Property---Property Domicile---Bankruptcy---Assignment (Partnership---Sale of Goods)}\\
        \end{center}
        \textbf{CP}. A very short report annexed to another. It appears to suggest that a Dutch assignment in bankruptcy was effective as against a foreign debtor, and that an English court allowed certain matters to proceed before it in support.\\\textit{Cited in: }Story Ed1 (at §395-400, in support of the general position that the assignment of debts should be governed by the domicile of the person holding the debt.)\\No known authors cited.
        \end{small}\\
        \rule{\textwidth}{0.5pt}
        

        \begin{small}
        \begin{center}
        \href{https://heinonline.org/HOL/P?h=hein.engrep/engrg0126&i=85}{\textit{Neale v Cottingham} (1764) 1 H Bl 134, 126 ER 81} \label{131} \\ 
\textit{Property---Assignment---Bankruptcy (Bankruptcy)}\\
        \end{center}
        \textbf{Ch-Ireland}. Not much really to see in this case. It seems to stand for the very simple proposition that the Irish courts recognised an assignment by bankruptcy in England.\\\textit{Cited in: }Story Ed1 (at §403-409, for the English rules on assignment in bankruptcy, suggesting that this is determined by one law being the law of domicile)\\No known authors cited.
        \end{small}\\
        \rule{\textwidth}{0.5pt}
        

        \begin{small}
        \begin{center}
        \href{https://heinonline.org/HOL/P?h=hein.engrep/engrg0126&i=83}{\textit{Jollet v Deponthieu} (1769) 1 H Bl 131, 126 ER 79} \label{130} \\ 
\textit{Property---Assignment (Confiscation)}\\
        \end{center}
        \textbf{Ch}. Very hard decision to parse. It appears to relate to the terms of a Dutch bankruptcy on assets confiscated in America (we are told held by the state of New Jersey).\\\textit{Cited in: }Story Ed1 (at §403-409, for the English rules on assignment in bankruptcy, suggesting that this is determined by one law being the law of domicile)\\No known authors cited.
        \end{small}\\
        \rule{\textwidth}{0.5pt}
        

        \begin{small}
        \begin{center}
        \href{https://heinonline.org/HOL/P?h=hein.engrep/engrf0098&i=1025}{\textit{Mostyn v Fabrigas} (1774) 1 Cowp 161, 98 ER 1021} \label{12} \\ 
\textit{Local and Transitory Actions (Tresspass)}\\
        \end{center}
        \textbf{KB}.  \textbf{Uses terms: }[\textit{lex loci}]. Allowing an action by a Minorcan for wrongs done in Minorca. Quite substantive reasoning with Lord Mansfield, dealing with the role of legal fictions and forms.\\\textit{Cited in: }Westlake Ed1 (At arts 120-122, in connection to “local” and “transitory” actions and the rules of jurisdiction)\\No known authors cited.
        \end{small}\\
        \rule{\textwidth}{0.5pt}
        

        \begin{small}
        \begin{center}
        \href{https://heinonline.org/HOL/P?h=hein.engrep/engrf0098&i=1124}{\textit{Holman v Johnson} (1775) 1 Cowp 342, 98 ER 1120} \label{51} \\ 
\textit{Contract---Illegality (Sale)}\\
        \end{center}
        \textbf{KB}. A very famous decision by Lord Mansfield on the relevance of illegality to a contract for the sale of tea in France, to be smuggled into England. Assumpsit was allowed on the basis that the contract could be completed in France. Of prime relevance is the fact that Lord Mansfield clearly makes use of a “choice of law” understanding within the judgement, though expresses it as not relevant in this case.\\\textit{Cited in: }Westlake Ed1 (Critically, at arts 129-200, favouring a more general understanding of the application of the rules on illegality.)\\\textit{Authors refered to: }Huber.
        \end{small}\\
        \rule{\textwidth}{0.5pt}
        

        \begin{small}
        \begin{center}
        \href{https://heinonline.org/HOL/P?h=hein.engrep/engrf0096&i=632}{\textit{Rafael v Verlest} (1776) 2 Black W 1055, 96 ER 628} \label{11} \\ 
\textit{ (Tresspass)}\\
        \end{center}
        \textbf{KB}. Very complex and hard decision to parse. There is some reference to “acts of princes” meaning there is no jurisdiction to try the action.\\\textit{Cited in: }Westlake Ed1 (At arts 120-122, in connection to “local” and “transitory” actions and the rules of jurisdiction)\\No known authors cited.
        \end{small}\\
        \rule{\textwidth}{0.5pt}
        

        \begin{small}
        \begin{center}
        \href{https://heinonline.org/HOL/P?h=hein.engrep/engrf0099&i=168}{\textit{Planché v Fletcher} (1779) 1 Doug 251, 99 ER 164} \label{77} \\ 
\textit{Illegality (Sale)}\\
        \end{center}
        \textbf{KB}. Lord Mansfield allowing recovery on a contract clearly designed to evade the revenue laws of France, strongly relying on the principle that English courts do not take notice of the revenue laws of another country.\\\textit{Cited in: }Westlake Ed1 (arts 192-200, critically, in connection to the non-consideration of foreign revenue laws, though said to represent the law of the time.)\\No known authors cited.
        \end{small}\\
        \rule{\textwidth}{0.5pt}
        

        \begin{small}
        \begin{center}
        \href{https://link.gale.com/apps/doc/CW0125544801/ECCO?u=oxford&sid=gale_marc&xid=19b67222&pg=534}{\textit{Ballantine v Golding} (1784) Cooke’s Bankrupt Laws 419} \label{39} \\ 
\textit{Contract---Contract Discharge---Bankruptcy---Application of Lex Loci Con (Debt---Contract)}\\
        \end{center}
        \textbf{KB}. Lord Mansfield giving the rule of a discharge of debts of a bankrupt being effective where the the debts arose there (though the actual rule might be wider.)\\\textit{Cited in: }Story Ed1 (§330-351, with approval] Westlake Ed)\\No known authors cited.
        \end{small}\\
        \rule{\textwidth}{0.5pt}
        

        \begin{small}
        \begin{center}
        \href{https://heinonline.org/HOL/P?h=hein.engrep/engra0002&i=390}{\textit{Bodham v Ryley} (1787) 4 Brown 561, 2 ER 382} \label{21} \\ 
\textit{Contract Interest---Contract---Application of Lex Loci Con (Partnership---Debts)}\\
        \end{center}
        \textbf{HL}. Report references some wide propositions, including a note on Huber and a general lex loci contractus principle. The actual grounds on which Indian interest was allowed seem less clear, and more related to presumed custom and intent of the parties.\\\\No known authors cited.
        \end{small}\\
        \rule{\textwidth}{0.5pt}
        

        \begin{small}
        \begin{center}
        \href{https://heinonline.org/HOL/P?h=hein.engrep/engrc0029&i=1227}{\textit{ex parte Blakes} (1787) 1 Cox 398, 29 ER 1219} \label{132} \\ 
\textit{Property---Assignment---Bankruptcy (Bankruptcy)}\\
        \end{center}
        \textbf{Ch}. A very peculiar form of order, attempting to compel an affadavit to be presented in America so as to allow assigned debts to be claimed there (since the American courts would not recognise otherwise.)\\\textit{Cited in: }Story Ed1 (at §403-409, for the English rules on assignment in bankruptcy, suggesting that this is determined by one law being the law of domicile)\\No known authors cited.
        \end{small}\\
        \rule{\textwidth}{0.5pt}
        

        \begin{small}
        \begin{center}
        \href{https://heinonline.org/HOL/P?h=hein.engrep/engrf0100&i=660}{\textit{Dewar v Span} (1789) 3 TR 425, 100 ER 656} \label{26} \\ 
\textit{Contract Interest---Contract---Application of English Statute---Application of Lex Fori (Contract---Vendor Purchaser)}\\
        \end{center}
        \textbf{KB}. Case entirely concerns the application of usury statutes to the West Indes.\\\\No known authors cited.
        \end{small}\\
        \rule{\textwidth}{0.5pt}
        

        \begin{small}
        \begin{center}
        \href{https://heinonline.org/HOL/P?h=hein.engrep/engrf0100&i=677}{\textit{Biggs v Lawrence} (1789) 3 TR 454, 100 ER 673} \label{70} \\ 
\textit{Illegality (Sale---Smuggling)}\\
        \end{center}
        \textbf{KB}. A seller (who was one of several partners) sold in Guernsey (the rest being in England) goods to be smuggled into England, packing them in such a way as to enable smuggling. Assumpsit could not be a succesfull.\\\textit{Cited in: }Westlake Ed1 (arts 192-200, used to suggest that the test for illegality (domestic and foreign) should be a general one of connection rather than inherent connection.)\\No known authors cited.
        \end{small}\\
        \rule{\textwidth}{0.5pt}
        

        \begin{small}
        \begin{center}
        \href{https://heinonline.org/HOL/P?h=hein.engrep/engrg0126&i=1255}{\textit{Bruce v Bruce} (1790) 2 B\&P 229, 126 ER 1251} \label{120} \\ 
\textit{Property---Property Domicile (Succession)}\\
        \end{center}
        \textbf{HL}.  \textbf{Uses terms: }[\textit{domicilum}]. Applying the domicile of the deceased (Scotland) to determine succession issues when dead abroad (East Indes).\\\textit{Cited in: }Story Ed1 (at §380-381, in support of the idea that movable property is determined by domicile)\\No known authors cited.
        \end{small}\\
        \rule{\textwidth}{0.5pt}
        

        \begin{small}
        \begin{center}
        \href{https://heinonline.org/HOL/P?h=hein.engrep/engrf0100&i=1126}{\textit{Clugas v Penaluna} (1791) 4 TR 466, 100 ER 1122} \label{69} \\ 
\textit{Illegality (Sale---Smuggling)}\\
        \end{center}
        \textbf{KB}. A seller in Guernsey could not recover for goods sold and delivered there, where he knew that the object was to smuggle them into England, and assistance was provided there. Assistance appears to have been in the packing of the goods.\\\textit{Cited in: }Westlake Ed1 (arts 192-200, used to suggest that the test for illegality (domestic and foreign) should be a general one of connection rather than inherent connection.)\\No known authors cited.
        \end{small}\\
        \rule{\textwidth}{0.5pt}
        

        \begin{small}
        \begin{center}
        \href{https://heinonline.org/HOL/P?h=hein.engrep/engrf0100&i=966}{\textit{Hunter v Potts} (1791) 4 TR 182, 100 ER 962} \label{121} \\ 
\textit{Property---Property Domicile---Bankruptcy (Bankruptcy)}\\
        \end{center}
        \textbf{KB}. Lord Kenyon strongly favouring the application of the domicile of the owner of property as determining the manner and effectiveness of its transfer, English assignment in bankruptcy said to be effective over lands held abroad (in Rhode Island).\\\textit{Cited in: }Story Ed1 (at §380-381, in support of the idea that movable property is determined by domicile); Story Ed1 (at §395-400, as showing the domicile of the owner of a debt as preferable for determining its assignment.)\\\textit{Authors refered to: }Voet.
        \end{small}\\
        \rule{\textwidth}{0.5pt}
        

        \begin{small}
        \begin{center}
        \href{https://heinonline.org/HOL/P?h=hein.engrep/engrg0126&i=383}{\textit{Sill v Worswick} (1791) 1 H Bl 655, 126 ER 379} \label{129} \\ 
\textit{Property---Assignment---Bankruptcy---Property Domicile (Bankruptcy)}\\
        \end{center}
        \textbf{CP}. A very complex case related to the assignment in England of certain debts in Scotland. Interestingly, the notion that these are governed by the lex domicilii does not appear to be settled.\\\textit{Cited in: }Story Ed1 (at §380-381, in support of the idea that movable property is determined by domicile); Story Ed1 (at §395-400, as showing the domicile of the owner of a debt as preferable for determining its assignment.)\\No known authors cited.
        \end{small}\\
        \rule{\textwidth}{0.5pt}
        

        \begin{small}
        \begin{center}
        \href{https://heinonline.org/HOL/P?h=hein.engrep/engrf0100&i=1147}{\textit{Doulson v Matthews} (1792) 4 TR 503, 100 ER 1143} \label{13} \\ 
\textit{Local and Transitory Actions (Tresspass)}\\
        \end{center}
        \textbf{KB}. An action could not lie for entering a house in Canada.\\\textit{Cited in: }Westlake Ed1 (At arts 120-122, in connection to “local” and “transitory” actions and the rules of jurisdiction)\\No known authors cited.
        \end{small}\\
        \rule{\textwidth}{0.5pt}
        

        \begin{small}
        \begin{center}
        \href{https://heinonline.org/HOL/P?h=hein.engrep/engrg0126&i=826}{\textit{Melan v Fitzjames} (1792) 1 Bos \& Pul 138, 126 ER 822} \label{30} \\ 
\textit{Contract---Application of Lex Loci Con (Bond---Real Estate)}\\
        \end{center}
        \textbf{CP}. Demonstrates a very clear lex loci contractus understanding, the Chief Justice clearly reasoning on such a line. There is a division in opinion. There is also some hint of the idea that reference to another law could be relevant.\\\textit{Cited in: }Story Ed1 (Cited with approval at §263-266, as to the applicable law for determining the nature of a contract, and the locus regit actum principle)\\No known authors cited.
        \end{small}\\
        \rule{\textwidth}{0.5pt}
        

        \begin{small}
        \begin{center}
        \href{https://heinonline.org/HOL/P?h=hein.engrep/engrf0101&i=339}{\textit{Waymell v Reed} (1794) 5 TR 599, 101 ER 335} \label{68} \\ 
\textit{Illegality (Sale---Smuggling)}\\
        \end{center}
        \textbf{KB}. A seller could not recover against a buyer for goods sold and delivered which were packaged in such a way as to enable smuggling into England. Expressly relying on Holman.\\\textit{Cited in: }Westlake Ed1 (arts 192-200, used to suggest that the test for illegality (domestic and foreign) should be a general one of connection rather than inherent connection.)\\No known authors cited.
        \end{small}\\
        \rule{\textwidth}{0.5pt}
        

        \begin{small}
        \begin{center}
        \href{https://heinonline.org/HOL/P?h=hein.engrep/engrg0126&i=622}{\textit{Phillips v Hunter} (1795) 2 H Bl 402, 126 ER 618} \label{122} \\ 
\textit{Property---Property Domicile---Bankruptcy ()}\\
        \end{center}
        \textbf{CP}. Assignment in Bankruptcy made in England effective against sums later recovered abroad (Pennsylvania). Case seems to be reasoned in terms of the effects and of the bankruptcy laws.\\\textit{Cited in: }Story Ed1 (at §380-381, in support of the idea that movable property is determined by domicile)\\No known authors cited.
        \end{small}\\
        \rule{\textwidth}{0.5pt}
        

        \begin{small}
        \begin{center}
        \href{https://heinonline.org/HOL/P?h=hein.engrep/engrf0101&i=957}{\textit{Alves v Hodgson} (1797) 7 TR 241, 101 ER 953} \label{58} \\ 
\textit{Contract---Contract Formality---Application of Lex Loci Con (Bill of Exchange---Marine Employment)}\\
        \end{center}
        \textbf{KB}. Lord Kenyon clearly suggesting that one must resort to the law of where a contract was made (in this case, Jamaica) to determine validity and formality requirements.\\\textit{Cited in: }Westlake Ed1 (arts 173-176, cited with general approval for the locus regit actum rule and formalities.)\\No known authors cited.
        \end{small}\\
        \rule{\textwidth}{0.5pt}
        

        \begin{small}
        \begin{center}
        \href{https://heinonline.org/HOL/P?h=hein.engrep/engrc0031&i=462https://heinonline.org/HOL/Page?handle=hein.engrep/engrc0031&id=462}{\textit{Wharton v May} (1799) 5 Ves Jun 27, 71, 31 ER 454} \label{64} \\ 
\textit{Injunction ()}\\
        \end{center}
        \textbf{Ch}. Cited as an early instance of an injunction to restrain proceedings, though not clear from looking at the case.\\\textit{Cited in: }Westlake Ed1 (arts 130-131, as illustrative of the principles governing the issue of anti-suit injunctions)\\No known authors cited.
        \end{small}\\
        \rule{\textwidth}{0.5pt}
        

        \begin{small}
        \begin{center}
        \href{https://heinonline.org/HOL/P?h=hein.engrep/engrk0170&i=578}{\textit{Male v Roberts} (1800) 3 Esp 163, 170 ER 574} \label{9} \\ 
\textit{ (Assumpsit Debt---Contract---Infancy)}\\
        \end{center}
        \textbf{CP}. Lord Eldon gives a clear understanding of the contract being determined by the laws of Scotland, based on the idea that it “arose” there.\\\textit{Cited in: }Story Ed1 (with approval, for rules on validity §241-249 and §330-351 for non-exclusive application of lex loci contractus for discharge of obligations)\\No known authors cited.
        \end{small}\\
        \rule{\textwidth}{0.5pt}
        

        \begin{small}
        \begin{center}
        \href{https://heinonline.org/HOL/P?h=hein.engrep/engrf0102&i=7}{\textit{Smith v Buchanan} (1800) 1 East 6, 102 ER 3} \label{41} \\ 
\textit{Contract---Contract Discharge---Bankruptcy---Application of Lex Loci Con---Property (Bankruptcy---Sale of Goods)}\\
        \end{center}
        \textbf{KB}. Maryland discharge by bankruptcy held to not affect debts contracted for in England, with a strong preference for lex loci contractus “It is impossible to say that a contract made in one country is to be governed by the laws of another.” Ballatine v Golding distinguished.\\\textit{Cited in: }Story Ed1 (§330-351, with approval); Westlake Ed1 (arts 235-256, with approval.); Story Ed1 (at §403-409, for the principles on the assignment of debts in bankruptcy.)\\No known authors cited.
        \end{small}\\
        \rule{\textwidth}{0.5pt}
        

        \begin{small}
        \begin{center}
        \href{https://heinonline.org/HOL/P?h=hein.engrep/engrf0101&i=1569}{\textit{Innes v Dunlop} (1800) 8 TR 595, 101 ER 1565} \label{46} \\ 
\textit{Contract---Application of Lex Loci Con---Contract Assignment (Debt---Contract)}\\
        \end{center}
        \textbf{KB}. Assignee of a Scottish bond allowed to sue in his own name in England. Assignments valid under Scots, but not English law. Unclear where assignment took place, though bond was clearly Scottish.\\\textit{Cited in: }Westlake Ed1 (arts 241-245, for the assignability of debts being judged by the point of their inception.)\\No known authors cited.
        \end{small}\\
        \rule{\textwidth}{0.5pt}
        

        \begin{small}
        \begin{center}
        \href{https://heinonline.org/HOL/P?h=hein.engrep/engrf0102&i=202}{\textit{Inglis v Underwood} (1801) 1 East 515, 102 ER 198} \label{128} \\ 
\textit{Property (Ships---Lien)}\\
        \end{center}
        \textbf{KB}. That assignees in bankruptcy took subject to a kind of lien, established in Russia, over a ship for mariner’s fees.\\\textit{Cited in: }Story Ed1 (at §401-402, that interests established in one state persist in another)\\No known authors cited.
        \end{small}\\
        \rule{\textwidth}{0.5pt}
        

        \begin{small}
        \begin{center}
        \href{https://heinonline.org/HOL/P?h=hein.engrep/engrc0032&i=880}{\textit{Bourke v Rickets} (1804) 10 Ves Jun 330, 32 ER 872} \label{22} \\ 
\textit{Contract Interest---Contract---Application of Lex Fori (Wills)}\\
        \end{center}
        \textbf{Ch}. A rather confusing report. The relevant legacies appear to have been executed in both Jamaica and England, and English interest was applied. There is said to not be a general rule, and the circumstances of the case are relied on (including that it was sued for in England.)\\\\No known authors cited.
        \end{small}\\
        \rule{\textwidth}{0.5pt}
        

        \begin{small}
        \begin{center}
        \href{https://heinonline.org/HOL/P?h=hein.engrep/engrf0102&i=1020}{\textit{Potter v Brown} (1804) 5 East 124, 102 ER 1016} \label{38} \\ 
\textit{Contract---Contract Discharge---Application of Lex Loci Con---Bankruptcy (Bill of Exchange)}\\
        \end{center}
        \textbf{KB}. A bill drawn in America on someone in England, was discharged by American certificate of bankruptcy, thereby discharging the defendant in the event of dishonour in England.\\\textit{Cited in: }Story Ed1 (With approval at §330-351, for defences and discharge of contract.); Westlake Ed1 (arts 235-256, with approval as for where the law of the contract coincides with the law of the bankruptcy.); Story Ed1 (at §403-409, for the English rules on assignment in bankruptcy, suggesting that this is determined by one law being the law of domicile)\\No known authors cited.
        \end{small}\\
        \rule{\textwidth}{0.5pt}
        

        \begin{small}
        \begin{center}
        \href{https://heinonline.org/HOL/P?h=hein.engrep/engrc0032&i=1117}{\textit{Cash v Kennion} (1805) 11 Ves 314, 32 ER 1109} \label{37} \\ 
\textit{Currency (Debt---Contract)}\\
        \end{center}
        \textbf{Ch}. A bond which was payable in London but contracted for in Jamaica. The costs of remitting currency were permitted. Rather uninteresting.\\\textit{Cited in: }Westlake Ed1 (arts 230-236, as part of a discussion of currency issues and damages.)\\No known authors cited.
        \end{small}\\
        \rule{\textwidth}{0.5pt}
        

        \begin{small}
        \begin{center}
        \href{https://heinonline.org/HOL/P?h=hein.engrep/engrg0128&i=37}{\textit{O Callaghan v Thomond} (1810) 3 Taunt 82, 128 ER 33} \label{44} \\ 
\textit{Contract---Application of Lex Loci Con---Contract Assignment (Debt---Contract)}\\
        \end{center}
        \textbf{CP}. Certain judgement debts assignable by Irish statute, held to be suable in England in the name of the asignee. Unclear if the debts themselves were Irish in origin (seem to be?).\\\textit{Cited in: }Story Ed1 (§362-373, for assignment of debts); Westlake Ed1 (arts 241-245, for the assignability of debts being judged by the point of their inception.)\\No known authors cited.
        \end{small}\\
        \rule{\textwidth}{0.5pt}
        

        \begin{small}
        \begin{center}
        \href{https://heinonline.org/HOL/P?h=hein.engrep/engri0161&i=671}{\textit{Dalrymple v Dalrymple} (1811) 2 Hag Con 54, 161 ER 665} \label{52} \\ 
\textit{Contract---Marriage---Lex Loci Contractus (Marriage)}\\
        \end{center}
        \textbf{Delegates}. A very famous dictum of Sir William Scott (Baron Stowell) invoking the express principle of English law retreating to refer matters to the Law of Scotland. The actual case is sprawling and complex, and mostly of a factual character (and dealing with the relevant points of Scots marriage law.) Extensive reference to continental jurists are added in the report (which is lengthy).\\\textit{Cited in: }Story Ed1 (§270-278, in connection to the interpretation of contracts); Westlake Ed1 (generally, for the basic position of “choice of law”)\\\textit{Authors refered to: }Huber, Voet.
        \end{small}\\
        \rule{\textwidth}{0.5pt}
        

        \begin{small}
        \begin{center}
        \href{https://heinonline.org/HOL/P?h=hein.engrep/engrk0170&i=1347}{\textit{Clegg v Levy} (1812) 3 Camp 166, 170 ER 1343} \label{55} \\ 
\textit{Contract---Contract Formality---Application of Lex Loci Con (Sale)}\\
        \end{center}
        \textbf{NP}. A very short (nisi prius) decision. Strongly suggesting that formality requirements (in this case, a stamp for a sale of goods contract in Surinam) are to be submitted to the place where the contract is made.\\\textit{Cited in: }Story Ed1 (§260-262, with approval for rules on formality and the locus regit actum rule); Westlake Ed1 (arts 173-176, for  total preference of lex loci contractus in issues of formality.)\\No known authors cited.
        \end{small}\\
        \rule{\textwidth}{0.5pt}
        

        \begin{small}
        \begin{center}
        \href{https://heinonline.org/HOL/P?h=hein.engrep/engrf0105&i=37}{\textit{Snaith v Mingay} (1813) 1 M\&S 87, 105 ER 33} \label{32} \\ 
\textit{Contract---Application of Lex Loci Con (Bill of Exchange)}\\
        \end{center}
        \textbf{KB}. A bill of exchange was signed – leaving blank dates and sums – in Ireland, and transmitted to England were it was filled by a partner of a firm, and thereafter indorsed etc. The bill was “an Irish bill” not requiring an English revenue stamp to be valid. The basis appears to be that the bill came into creation in Ireland on being drawn up, though this is not framed exactly as a lex loci contractus rule.\\\textit{Cited in: }Story Ed1 (Cited with approval at §279-290, to suggest a potential exception to lex loci contractus); Westlake Ed1 (arts 180-183, on the strong basis that the place of drawing should govern a bill of exchange.)\\No known authors cited.
        \end{small}\\
        \rule{\textwidth}{0.5pt}
        

        \begin{small}
        \begin{center}
        \href{https://heinonline.org/HOL/P?h=hein.engrep/engrf0105&i=321}{\textit{Simeon v Bazett} (1813) 2 M\&S 94, 105 ER 317} \label{78} \\ 
\textit{Illegality---Revenue Laws (Insurance)}\\
        \end{center}
        \textbf{KB}. Recovery allowed on a contract of insurance for goods confiscated by the Prussian government in exercise of revenue laws there, resting on an application of the principle that one country does not take notice of the revenue laws of another (though maybe just on the terms of the contract of insurance?)\\\textit{Cited in: }Westlake Ed1 (arts 192-200, critically, in connection to the non-consideration of foreign revenue laws, though said to represent the law of the time.)\\No known authors cited.
        \end{small}\\
        \rule{\textwidth}{0.5pt}
        

        \begin{small}
        \begin{center}
        \href{https://heinonline.org/HOL/P?h=hein.engrep/engrg0128&i=812}{\textit{Nathan v Giles} (1814) 5 Taunt R 558, 128 ER 808} \label{123} \\ 
\textit{Property---Property Protection of Third Parties---Assignment (Bill of Lading---Assignment)}\\
        \end{center}
        \textbf{CP}. Slightly unclear. It seems to be the case that an assignment made in Hamburg did not affect a lein that existed (it seems on a bill of lading?)\\\textit{Cited in: }Story Ed1 (at §388-94, to suggest that preference might be made to the lex fori for the transfer of movable property where this has the effect of disadvantaging parties within the jurisdiction)\\No known authors cited.
        \end{small}\\
        \rule{\textwidth}{0.5pt}
        

        \begin{small}
        \begin{center}
        \href{https://heinonline.org/HOL/P?h=hein.engrep/engra0003&i=856}{\textit{Skelrig v Davis} (1814) 2 Dow 230, 3 ER 948} \label{126} \\ 
\textit{Property---Property Domicile---Assignment---Bankruptcy---Property Real (Shares---Bankruptcy)}\\
        \end{center}
        \textbf{HL (SC)}. English assignment in bankruptcy held effective over Scottish-held shares, without the formal requirements in Scotland. This case doesn’t seem to be reasoned on the terms that Story sets it out for. Some indication also that immovable property abroad cannot be assigned in bankruptcy.\\\textit{Cited in: }Story Ed1 (at §395-400, in support of the general position that the assignment of debts should be governed by the domicile of the person holding the debt.); Story Ed1 (at §428, that immovable property is not subject to assignment in bankruptcy, creating only a moral obligation to convey title)\\No known authors cited.
        \end{small}\\
        \rule{\textwidth}{0.5pt}
        

        \begin{small}
        \begin{center}
        \href{https://heinonline.org/HOL/P?h=hein.engrep/engrf0105&i=791}{\textit{Power v Whitmore} (1815) 4 M\&S 141, 105 ER 787} \label{34} \\ 
\textit{Contract---Lex Loci Con---Foreign Judgements---Contract Intention (General Average---Insurance)}\\
        \end{center}
        \textbf{KB}.  \textbf{Uses terms: }[\textit{comity, law of nations}]. The whole case is framed in terms of the “custom of merchants” and the presumed intentions of the parties, and seems also to suggest some potential application of lex loci solutionis. Foreign judgements also come into play, since the sums demanded related to a judgement of a Lisbon court.\\\textit{Cited in: }Westlake Ed1 (arts 225-228, as showing the place of contracting as defining the requirements of the obligation)\\No known authors cited.
        \end{small}\\
        \rule{\textwidth}{0.5pt}
        

        \begin{small}
        \begin{center}
        \href{https://heinonline.org/HOL/P?h=hein.engrep/engrg0127&i=187}{\textit{Splitberger v Kohn} (1815) 1 Star 125, 171 ER 422} \label{47} \\ 
\textit{Contract---Assignment ()}\\
        \end{center}
        \textbf{NP}. Very vague nisi prius report concerning a promissory note made in Prussia. Only point concerns the inclusion of certain details on the note.\\\textit{Cited in: }Westlake Ed1 (arts 241-245, for the idea that the required form of the validity of an assignment is determined by the place of assignment)\\No known authors cited.
        \end{small}\\
        \rule{\textwidth}{0.5pt}
        

        \begin{small}
        \begin{center}
        \href{https://heinonline.org/HOL/P?h=hein.engrep/engrf0105&i=1181}{\textit{Wolf v Oxholm} (1817) 7 M\&S 92, 105 ER 1177} \label{40} \\ 
\textit{Contract---Contract Discharge (Bankruptcy---Assignment)}\\
        \end{center}
        \textbf{KB}.  \textbf{Uses terms: }[\textit{law of nations}]. Debts owed by Danish subject to English partnership, contracted for in England. Confiscation order by the Danish Crown – sequestering debts owed to English subjects – found not to have lead to discharge of the debts, “not being conformable to the usage of nations.” Strong reliance on International law authorities. Some noting also of the law of assignment.\\\textit{Cited in: }Story Ed1 (§330-351, with approval, to suggest application of the law of nations to prevent discharge of certain debts.)\\\textit{Authors refered to: }Vattel, Grotius, Puffendorf, Bynkerschoek.
        \end{small}\\
        \rule{\textwidth}{0.5pt}
        

        \begin{small}
        \begin{center}
        \href{https://heinonline.org/HOL/P?h=hein.engrep/engrf0105&i=1194}{\textit{Jeffery v McTaggart} (1817) 6 M\&S 126, 105 ER 1190} \label{45} \\ 
\textit{Contract---Application of Lex Loci Con---Bankruptcy (Bankruptcy)}\\
        \end{center}
        \textbf{KB}. Choses in action (unclear where from or of what nature) deemed not to be assigned by Scottish bankruptcy proceedings, thereby not allowing the plaintiff to bring the action in his own name. Actual reasoning based on the language of the act in question.\\\textit{Cited in: }Westlake Ed1 (arts 241-245, for the non-assignability of contracts by the point of their inception)\\No known authors cited.
        \end{small}\\
        \rule{\textwidth}{0.5pt}
        

        \begin{small}
        \begin{center}
        \href{https://heinonline.org/HOL/P?h=hein.engrep/engrg0129&i=552}{\textit{Boehm v Campbell} (1818) Gow 56, 8 Taun 679, 171 ER 837, 129 ER 548} \label{60} \\ 
\textit{Contract---Contract Formality (Bill of Exchange)}\\
        \end{center}
        \textbf{CP}. Unclear if much is to be found in the report, it really seems to turn on the validity of consideration.\\\textit{Cited in: }Westlake Ed1 (arts 180-183, Cited to illustrate the position prior to 17and 18 Vict c.83, s 5 -- and the requirement of a British stamp on Irish and Great British Bills)\\No known authors cited.
        \end{small}\\
        \rule{\textwidth}{0.5pt}
        

        \begin{small}
        \begin{center}
        \href{https://heinonline.org/HOL/P?h=hein.engrep/engrc0037&i=751}{\textit{Harrison v Gurney} (1820) 2 Jac \& W 563, 37 ER 743} \label{65} \\ 
\textit{Injunction (Bankruptcy)}\\
        \end{center}
        \textbf{Ch}. Injunction granted to restrain proceedings in Ireland, seemingly on the basis that it was a second action to one already commenced in England.\\\textit{Cited in: }Westlake Ed1 (arts 130-131, as illustrative of the principles governing the issue of anti-suit injunctions)\\No known authors cited.
        \end{small}\\
        \rule{\textwidth}{0.5pt}
        

        \begin{small}
        \begin{center}
        \href{https://heinonline.org/HOL/P?h=hein.engrep/engrf0106&i=694}{\textit{Madrazo v Willes} (1820) 3 B \& Ald 353, 106 ER 692} \label{76} \\ 
\textit{Illegality---Slavery (Slavery)}\\
        \end{center}
        \textbf{KB}. A foreigner could recover for loss of slaves against a British subject in a court in England.\\\textit{Cited in: }Westlake Ed1 (arts 192-200, cited critically in connection to the English case law on foreign slavery.)\\No known authors cited.
        \end{small}\\
        \rule{\textwidth}{0.5pt}
        

        \begin{small}
        \begin{center}
        \href{https://heinonline.org/HOL/P?h=hein.engrep/engri0161&i=780}{\textit{Ruding v Smith} (1821) 2 Hag Con 371, 161 ER 774} \label{53} \\ 
\textit{Contract---Marriage (Marriage)}\\
        \end{center}
        \textbf{Delegates}.  \textbf{Uses terms: }[\textit{lex loci}]. A slightly odd decision. A marriage entered by a member of HM Services in Cape Colony was not adjudged by the Law of Holland, but by the Law of England, apparently on a special basis. Huber cited by Counsel (Dr Phillimore) and the bench.\\\textit{Cited in: }Westlake Ed1 (arts 148-152, used to suggest there are cases where the law “common to the parties” is used.)\\\textit{Authors refered to: }Huber.
        \end{small}\\
        \rule{\textwidth}{0.5pt}
        

        \begin{small}
        \begin{center}
        \href{https://heinonline.org/HOL/P?h=hein.engrep/engre0056&i=916}{\textit{Bushby v Munday} (1821) 5 Madd 297, 56 ER 908} \label{62} \\ 
\textit{ (Debt---Contract---Gambling)}\\
        \end{center}
        \textbf{Ch}. Injunction granted to restrain proceedings in Scotland, seemingly on the basis that English courts were the proper forum to consider the application of an English statute. Very extensive report.\\\textit{Cited in: }Westlake Ed1 (arts 130-131, as illustrative of the principles governing the issue of anti-suit injunctions)\\No known authors cited.
        \end{small}\\
        \rule{\textwidth}{0.5pt}
        

        \begin{small}
        \begin{center}
        \href{https://heinonline.org/HOL/P?h=hein.engrep/engre0057&i=11}{\textit{Beckford v Kemble} (1822) 1 Sim \& St 7, 57 ER 3} \label{63} \\ 
\textit{Injunction (Mortgage)}\\
        \end{center}
        \textbf{Ch}. Injunction granted to restrain proceedings in Jamaica, seemingly on the basis that all of the parties were England and proceedings were brought in England first.\\\textit{Cited in: }Westlake Ed1 (arts 130-131, as illustrative of the principles governing the issue of anti-suit injunctions)\\No known authors cited.
        \end{small}\\
        \rule{\textwidth}{0.5pt}
        

        \begin{small}
        \begin{center}
        \href{https://heinonline.org/HOL/P?h=hein.engrep/engrc0037&i=964}{\textit{Clarke v Ormonde} (1822) Jac 546, 37 ER 956} \label{66} \\ 
\textit{Injunction (Administration of Estates)}\\
        \end{center}
        \textbf{Ch}. Injunction granted to restrain proceedings in Ireland, seemingly on the basis that it was a second action to one already commenced in England.\\\textit{Cited in: }Westlake Ed1 (arts 130-131, as illustrative of the principles governing the issue of anti-suit injunctions)\\No known authors cited.
        \end{small}\\
        \rule{\textwidth}{0.5pt}
        

        \begin{small}
        \begin{center}
        \href{https://heinonline.org/HOL/P?h=hein.engrep/engrf0107&i=76}{\textit{Milne v Graham} (1823) 1 B\&C 192, 107 ER 72} \label{48} \\ 
\textit{Contract---Assignment (Debt---Contract)}\\
        \end{center}
        \textbf{KB}. Allowing for an action on a Scottish promissory note brought in England. The entire question appears to relate to the application of an English statute.\\\textit{Cited in: }Westlake Ed1 (arts 241-245, for the idea that the required form of the validity of an assignment is determined by the place of assignment)\\No known authors cited.
        \end{small}\\
        \rule{\textwidth}{0.5pt}
        

        \begin{small}
        \begin{center}
        \href{None}{\textit{James v Catterwood} (1823) 3 Dow \& Ryl 190} \label{56} \\ 
\textit{Contract---Contract Formality ()}\\
        \end{center}
        \textbf{NP}. I can’t find this case in the English reports, though is said to suggest an alternative position as regards the rules on formalities.\\\textit{Cited in: }Westlake Ed1 (Cited somewhat critically at arts 173-176 and 177, as supporting an alternative position to the application of lex loci contractus); Story Ed1 (§260-62, with sceptisism, said to point to non application of locus regit actum for formalities.)\\No known authors cited.
        \end{small}\\
        \rule{\textwidth}{0.5pt}
        

        \begin{small}
        \begin{center}
        \href{https://heinonline.org/HOL/P?h=hein.engrep/engrc0037&i=1121}{\textit{Jones v Garcia del Rio} (1823) T\&R 297, 37 ER 1113} \label{71} \\ 
\textit{Illegality (Loan)}\\
        \end{center}
        \textbf{Ch}. There could not be recovery for loans to the government of Peru, then not recognised by the Government of Great Britain.\\\textit{Cited in: }Westlake Ed1 (arts 192-200, used to suggest that the general connection to insurection abroad is sufficient to defeat an action, this being connected to the general rules on illegality.)\\No known authors cited.
        \end{small}\\
        \rule{\textwidth}{0.5pt}
        

        \begin{small}
        \begin{center}
        \href{https://heinonline.org/HOL/P?h=hein.engrep/engrf0107&i=454}{\textit{Forbes v Chochrane} (1824) 2 B\&C 448, 107 ER 450} \label{75} \\ 
\textit{Illegality (Slavery)}\\
        \end{center}
        \textbf{KB}. There could be no recovery for the recovery of Slaves abroad by a British subject in the Courts in England, even where the master of a ship had notice of the slaves aboard their ship.\\\textit{Cited in: }Westlake Ed1 (arts 192-200, to suggest the non-applicability of foreign morally repugnant laws or those contrary to English public policy)\\No known authors cited.
        \end{small}\\
        \rule{\textwidth}{0.5pt}
        

        \begin{small}
        \begin{center}
        \href{None}{\textit{Arnott v Redfern} (1825) 2 Car \& P 88, 172 ER 40} \label{35} \\ 
\textit{Contract---Lex Loci Con---Interest ()}\\
        \end{center}
        \textbf{NP}. A very short (nisi prius) decision – though strongly suggesting that the place of contracting (England) was determinative of the issue.\\\textit{Cited in: }Westlake Ed1 (arts 230-236, cited critically in connection to an understanding of the laws of interest.)\\No known authors cited.
        \end{small}\\
        \rule{\textwidth}{0.5pt}
        

        \begin{small}
        \begin{center}
        \href{https://heinonline.org/HOL/P?h=hein.engrep/engrc0038&i=376}{\textit{Wynne v Jackson} (1826) 2 Russ 351, 38 ER 368} \label{57} \\ 
\textit{Contract---Contract Formality---Application of Lex Fori (Bill of Exchange)}\\
        \end{center}
        \textbf{Ch}. Bills of Exchange drawn in France not made in compliance with requirements of the Cc; action for monies held in court for nonpayment nontheless allowed, actual reasoning rather unclear.\\\textit{Cited in: }Story Ed1 (§260-262, with disapproval – suggesting wrongful application of lex fori for rules on formality); Westlake Ed1 (arts 173-176, with disapproval on the rules of formalities.)\\No known authors cited.
        \end{small}\\
        \rule{\textwidth}{0.5pt}
        

        \begin{small}
        \begin{center}
        \href{https://heinonline.org/HOL/P?h=hein.engrep/engrc0038&i=121}{\textit{Wynne v Callander} (1826) 1 Russ 293, 38 ER 113} \label{67} \\ 
\textit{Illegality---Contract Validity (Bill of Exchange---Gambling)}\\
        \end{center}
        \textbf{Ch}. A slightly confusing decision. English bills seem to have been exchanged for French Bills in France, as payment of gaming debts. Delivery up of the bills awarded.\\\textit{Cited in: }Westlake Ed1 (arts 192-200, with approval, to suggests that the lawfulness or validity of consideration is determined by the place of contracting)\\No known authors cited.
        \end{small}\\
        \rule{\textwidth}{0.5pt}
        

        \begin{small}
        \begin{center}
        \href{https://heinonline.org/HOL/P?h=hein.engrep/engrf0108&i=167}{\textit{Birthwhistle v Vardill} (1826) 5 B\&C 438, 108 ER 163} \label{133} \\ 
\textit{Property Real (Succession)}\\
        \end{center}
        \textbf{KB}.  \textbf{Uses terms: }[\textit{lex domicilii}]. A child was born out of Wedlock in Scotland (and later domiciled there) to parents who later married in England. By the laws of Scotland he could inherit lands, but not English law. He could not inherit lands in England. Specific distinction drawn between movable property (governed by the lex domicili) and immovable property, which is governed by the law of where it is situate.\\\textit{Cited in: }Story Ed1 (at §428-434, for the strong preference on English courts in looking to the lex situs for capacity in dealing with immovable property)\\\textit{Authors refered to: }Huber.
        \end{small}\\
        \rule{\textwidth}{0.5pt}
        

        \begin{small}
        \begin{center}
        \href{https://heinonline.org/HOL/P?h=hein.engrep/engrk0173&i=1087}{\textit{Bentley v Northouse} (1827) M\&M 66, 173 ER 1083} \label{49} \\ 
\textit{Contract---Assignment (Bill of Exchange)}\\
        \end{center}
        \textbf{NP}. Bills of Exchange made in Scotland could be transferred by indorsement in England. The whole reasoning seems to hinge on the application of statute, this being said to also cover notes made outside of England.\\\textit{Cited in: }Westlake Ed1 (arts 241-245, for the idea that the required form of the validity of an assignment is determined by the place of assignment)\\No known authors cited.
        \end{small}\\
        \rule{\textwidth}{0.5pt}
        

        \begin{small}
        \begin{center}
        \href{https://heinonline.org/HOL/P?h=hein.engrep/engra0006&i=565}{\textit{Pattison v Mills} (1828) 1 Dow \& Clark 342, 6 ER 553} \label{33} \\ 
\textit{Contract---Application of English Statute---Application of Lex Loci Con---Illegality (Insurance)}\\
        \end{center}
        \textbf{HL(SC)}. A contract made in Glasgow was not subject to an English statute giving a monopoly for insuring marine risks. Strong dictum by Lord Lyndhurst, strongly hinting at a lex loci contractus rule.\\\textit{Cited in: }Story Ed1 (§279-290, for the principle that lex loci refers to place where an agent goes to make a contract); Westlake Ed1 (Cited critically at §192-200, for a misunderstanding of the rules on illegality.)\\No known authors cited.
        \end{small}\\
        \rule{\textwidth}{0.5pt}
        

        \begin{small}
        \begin{center}
        \href{https://heinonline.org/HOL/P?h=hein.engrep/engrf0108&i=1124}{\textit{Phillips v Allan} (1828) 7 B\&C 477, 108 ER 1120} \label{42} \\ 
\textit{Contract---Contract Discharge---Bankruptcy---Application of Lex Loci Con (Bankruptcy---Bill of Exchange)}\\
        \end{center}
        \textbf{KB}. Dsicharge by cessio in bonorum by the Court of Session in Scotland not effective to discharge debt on bill of exchange contracted in England. Actual decision not clearly reasoned in terms of applicable law; greater emphasis placed on jurisdictional powers of courts, and the lack of benefit by the plaintiff from the Scotch proceedings.\\\textit{Cited in: }Story Ed1 (§330-351, with approval); Westlake Ed1 (arts 235-256, with approval – for the position of where the place of contracting does not coincide with the place of discharge.)\\No known authors cited.
        \end{small}\\
        \rule{\textwidth}{0.5pt}
        

        \begin{small}
        \begin{center}
        \href{https://heinonline.org/HOL/P?h=hein.engrep/engre0057&i=769}{\textit{Thompson v Powels} (1828) 2 Sim 194, 57 ER 761} \label{73} \\ 
\textit{Illegality---Foreign Affairs (Loan)}\\
        \end{center}
        \textbf{Ch}. Bonds issued as loans to a Government not recognised by Great Britain, there could be no recovery.\\\textit{Cited in: }Westlake Ed1 (arts 192-200, used to suggest that the general connection to insurection abroad is sufficient to defeat an action, this being connected to the general rules on illegality.)\\No known authors cited.
        \end{small}\\
        \rule{\textwidth}{0.5pt}
        

        \begin{small}
        \begin{center}
        \href{https://heinonline.org/HOL/P?h=hein.engrep/engre0057&i=777}{\textit{Taylor v Barclay} (1828) 2 Sim 213, 57 ER 769} \label{74} \\ 
\textit{Illegality---Foreign Affairs (Loan)}\\
        \end{center}
        \textbf{Ch}. Bonds issued as loans to a Government not recognised by Great Britain, there could be no recovery, framed specifically as a rule of pleading.\\\textit{Cited in: }Westlake Ed1 (arts 192-200, used to suggest that the general connection to insurection abroad is sufficient to defeat an action, this being connected to the general rules on illegality.)\\No known authors cited.
        \end{small}\\
        \rule{\textwidth}{0.5pt}
        

        \begin{small}
        \begin{center}
        \href{https://heinonline.org/HOL/P?h=hein.engrep/engrf0109&i=82}{\textit{De La Chaumette v Bank of England} (1829) 9 B\&C 208, 109 ER 78} \label{43} \\ 
\textit{Contract---Application of Lex Loci Con (Bill of Exchange)}\\
        \end{center}
        \textbf{KB}. Action for trover and non-payment on a bearer note issued by the defendants. Defendants refuse to pay on the basis of the note having been stolen. Clear implication is that the rules governing the assignment / entitlement to the note subject to the English rules of giving value, though not a clear conclusion. New trial ordered, later proceedings noted.\\\textit{Cited in: }Story Ed1 (§330-351, with approval); Westlake Ed1 (arts 235-256, with approva)\\No known authors cited.
        \end{small}\\
        \rule{\textwidth}{0.5pt}
        

        \begin{small}
        \begin{center}
        \href{https://heinonline.org/HOL/P?h=hein.engrep/engrf0109&i=796}{\textit{De la Vega v Vianna} (1830) 1 B \& Ad 284, 109 ER 792} \label{54} \\ 
\textit{Contract (Debt---Contract)}\\
        \end{center}
        \textbf{KB}. The bench expressly ignores the continental authorities cited (Huber and Voet). The apparent importance of the case is the distinction between remedy and right, though the phrasing of the judgement is much wider than this.\\\textit{Cited in: }Westlake Ed1 (art 411, for the distinction between remedy and right.)\\\textit{Authors refered to: }Huber, Voet.
        \end{small}\\
        \rule{\textwidth}{0.5pt}
        

        \begin{small}
        \begin{center}
        \href{https://heinonline.org/HOL/P?h=hein.engrep/engrf0109&i=521}{\textit{Holdsworth v Hunter} (1830) 10 B \& C 449, 109 ER 517} \label{59} \\ 
\textit{Contract---Contract Formality---Application of Lex Loci Con (Bill of Exchange---Marine Employment)}\\
        \end{center}
        \textbf{KB}. Applying locus regit actum to reverse effect: the English Stamp Act did not effect a bill drawn in India. (Though much of the reasoning focuses on the statute itself).\\\textit{Cited in: }Westlake Ed1 (arts 180-183, specifically in relation to the formality rules for indorsement.)\\No known authors cited.
        \end{small}\\
        \rule{\textwidth}{0.5pt}
        

        \begin{small}
        \begin{center}
        \href{https://heinonline.org/HOL/P?h=hein.engrep/engra0006&i=769}{\textit{Dundas v Dundas} (1830) 2 Dow \& Cl 349, 6 ER 757} \label{134} \\ 
\textit{Property Real (Succession)}\\
        \end{center}
        \textbf{HL (SC)}. Lands held in England could not be devised by a Scotch trust deed that did not meet the English requirements of the Statute of Frauds.\\\textit{Cited in: }Story Ed1 (at §435-444, for the preference of the English courts for application of lex situs for immovable property and formality)\\No known authors cited.
        \end{small}\\
        \rule{\textwidth}{0.5pt}
        

        \begin{small}
        \begin{center}
        \href{https://heinonline.org/HOL/P?h=hein.engrep/engrf0109&i=1077}{\textit{Scott v Bevan} (1831) 2 Ba \& Ad 78, 109 ER 1073} \label{36} \\ 
\textit{ (Debt---Contract)}\\
        \end{center}
        \textbf{KB}. Actual exchange rate between England and Jamaica applied.\\\textit{Cited in: }Westlake Ed1 (arts 230-236, as part of a discussion of currency issues and damages.)\\No known authors cited.
        \end{small}\\
        \rule{\textwidth}{0.5pt}
        

        \begin{small}
        \begin{center}
        \href{https://heinonline.org/HOL/P?h=hein.engrep/engrf0109&i=1330}{\textit{Novelli v Rossi} (1831) 2 B \& Ad, 109 ER 1326} \label{61} \\ 
\textit{Contract---Foreign Judgements (Bill of Exchange)}\\
        \end{center}
        \textbf{KB}. An English court bound by a decision of a French court on English law.\\\textit{Cited in: }Story Ed1 (§269)\\No known authors cited.
        \end{small}\\
        \rule{\textwidth}{0.5pt}
        

        \begin{small}
        \begin{center}
        \href{https://heinonline.org/HOL/P?h=hein.engrep/engra0006&i=790}{\textit{Scott v Alnutt} (1831) 2 Dow \& Clark 404, 6 ER 778} \label{125} \\ 
\textit{Property---Property Domicile---Assignment (Reversionary Interests)}\\
        \end{center}
        \textbf{HL (SC)}.  \textbf{Uses terms: }[\textit{lex domicilii}]. Assignment of Scottish reversionary interest valid when done by English (but not Scottish) form. Holder of the interest appears to have been in England at the time.\\\textit{Cited in: }Story Ed1 (at §395-400, in support of the general position that the assignment of debts should be governed by the domicile of the person holding the debt.)\\\textit{Authors refered to: }Voet, Erskine.
        \end{small}\\
        \rule{\textwidth}{0.5pt}
        

        \begin{small}
        \begin{center}
        \href{https://heinonline.org/HOL/P?h=hein.engrep/engra0006&i=786}{\textit{Attorney General v Mill} (1831) 2 Dow \& Cl 393, 6 ER 774} \label{136} \\ 
\textit{Property Real (Mortmain)}\\
        \end{center}
        \textbf{HL}. A divestment of lands that was in violation of the statutes of mortmain was invalid, despite the fact that the testator was domiciled in Scotland. Will was made of an English form and in England. The actual reasoning of the case is really in terms of the construction of the will and the statutes of mortmain, the “conflicts” issues are not really reasoned as such,\\\textit{Cited in: }Story Ed1 (at §445-446, for the lex situs rule being used to determine the extent of interests in immovable property, specifically related to the application of mortmain statutes in England)\\No known authors cited.
        \end{small}\\
        \rule{\textwidth}{0.5pt}
        

        \begin{small}
        \begin{center}
        \href{https://heinonline.org/HOL/P?h=hein.engrep/engrg0130&i=330}{\textit{De Wutz v Hendricks} (1834) 2 Bing 314, 130 ER 326} \label{72} \\ 
\textit{Illegality---Foreign Affairs (Loan)}\\
        \end{center}
        \textbf{CP}.  \textbf{Uses terms: }[\textit{law of nations}]. There could be no recovery in connection to transactions made for the borrowing of money by the government of Greece, who was at war with the government of Porte.\\\textit{Cited in: }Westlake Ed1 (arts 192-200, used to suggest that the general connection to insurection abroad is sufficient to defeat an action, this being connected to the general rules on illegality.)\\No known authors cited.
        \end{small}\\
        \rule{\textwidth}{0.5pt}
        

        \begin{small}
        \begin{center}
        \href{https://heinonline.org/HOL/P?h=hein.engrep/engrg0131&i=1079}{\textit{Trimbey v Vignier} (1834) 1 Bing NC 151, 131 ER 1075} \label{80} \\ 
\textit{Assignment---Contract ()}\\
        \end{center}
        \textbf{CP}.  \textbf{Uses terms: }[\textit{lex loci contractus}]. A bill of exchange was indorsed in blank in England. By the French law, this did not allow transfer of the bill. The subsequent holder could not recover in English courts. The bills appear to have been made and drawn in France.\\\textit{Cited in: }Westlake Ed1 (arts 241-245, said to not clearly resolve the issue of the applicable law for an assigment (potentially going against Westlake’s own position)\\\textit{Authors refered to: }Huber.
        \end{small}\\
        \rule{\textwidth}{0.5pt}
        

        \begin{small}
        \begin{center}
        \href{https://heinonline.org/HOL/P?h=hein.engrep/engrg0131&i=953}{\textit{Acebal v Levy} (1834) 10 Bing 376, 131 ER 949} \label{84} \\ 
\textit{Contract---Contract Formality (Sale---Statute of Frauds)}\\
        \end{center}
        \textbf{CP}. A contract for the sale of nuts – effectively on cif terms – from Spain to England. The contract  did not comply with (English) evidentiary requirements of the statute of frauds, and was thus held to not found an action by the plaintiffs for non-payment and non acceptance of delivery. Other various issues argued.  The actual basis on which the statute of frauds was applicable was not really reasoned or clearly explained, though it seems to have been assumed. A number of other issues were argued.\\\textit{Cited in: }Westlake Ed1 (arts 178-179, in support of his treatment of postal contract, to suggest that where the initial offer was sent should define the rules.)\\No known authors cited.
        \end{small}\\
        \rule{\textwidth}{0.5pt}
        

        \begin{small}
        \begin{center}
        \href{https://heinonline.org/HOL/P?h=hein.engrep/engrc0040&i=48}{\textit{Portarlington v Soulby} (1834) 3 My \& Ke 104, 40 ER 40} \label{99} \\ 
\textit{Injunction (Litigation---Bill of Exchange---Gambling)}\\
        \end{center}
        \textbf{Ch}. An injunction granted to restrain proceedings in Ireland in connection to a bill of exchange. The actual basis for doing so is rather unclear: the only real facts we are given is that the facts involved seemed in some way suspect. What is interesting to note is that we get a very early understanding of the injunction not restraining the foreign court but the defendant, and thus not being objectionable.\\\textit{Cited in: }Westlake Ed1 (arts 130-131, as illustrative of the principles governing the issue of anti-suit injunctions)\\No known authors cited.
        \end{small}\\
        \rule{\textwidth}{0.5pt}
        

        \begin{small}
        \begin{center}
        \href{https://heinonline.org/HOL/P?h=hein.engrep/engrg0132&i=84}{\textit{Huber v Steiner} (1835) 2 Bing NC 202, 132 ER 80} \label{79} \\ 
\textit{Contract---Application of Lex Loci Con (Bill of Exchange)}\\
        \end{center}
        \textbf{CP}.  \textbf{Uses terms: }[\textit{lex loci contractus, lex fori}]. An interesting decision. The court relies on an express observation by Story, that laws of prescription are admitted to the lex fori unless they serve to make causes of action nullities or themselves extinguished (as opposed to merely time-bared). The French law of prescription (on a promisory note made in France, itself readily admitted to be subject to the terms of the Code de Commerce) was found to fall into the former category.\\\textit{Cited in: }Westlake Ed1 (arts 250-252, critically, in connection to the English preference for the lex fori in rules of prescription)\\\textit{Authors refered to: }Huber, Story.
        \end{small}\\
        \rule{\textwidth}{0.5pt}
        

        \begin{small}
        \begin{center}
        \href{https://heinonline.org/HOL/P?h=hein.engrep/engrh0150&i=139}{\textit{Pellecat v Angell} (1835) 2 CM\&R 312, 150 ER 135} \label{89} \\ 
\textit{Contract---Illegality (Sale---Smuggling)}\\
        \end{center}
        \textbf{Ex}. Action for non-payment for goods sold in France was good, notwithstanding that the goods were sold in knowledge of an intent to smuggle into England. The distinction is between knowing of an illegal purpose and actually being party to it.\\\textit{Cited in: }Westlake Ed1 (arts 192-200, cited critically in connection to the need to be directly involved in illegality to bar an action in the English courts.)\\No known authors cited.
        \end{small}\\
        \rule{\textwidth}{0.5pt}
        

        \begin{small}
        \begin{center}
        \href{https://heinonline.org/HOL/P?h=hein.engrep/engra0007&i=313}{\textit{Don v Lippmann} (1837) 5 Cl\& F 1, 7 ER 303} \label{86} \\ 
\textit{Contract---Prescription---Application of Lex Loci Con---Contract Intention (Bill of Exchange)}\\
        \end{center}
        \textbf{HL (SC)}.  \textbf{Uses terms: }[\textit{lex fori, lex loci contractus, lex loci solutionis}]. A very fully reasoned case (one of the most heavily in terms of the conflict of laws I have seen.) A very clear distinction is drawn between the concepts of lex loci contractus, lex loci solutionus, and lex fori. The clear understanding is that the lex loci contractus determined the substantive law (a bill of exchange made in France) though this is wrapped somewhat in language of intention of the parties. The lex fori is said to firmly govern the issues of prescription, in this case making the claim out of time. (There is a separate issue on the enforcement of a French Judgement).\\\textit{Cited in: }Westlake Ed1 (arts 187-191, critically, to suggest and incorrect approach to the interpretation of obligations); Westlake Ed1 (arts 250-252, critically, in connection to the application of the lex fori for prescription (a position Westlake rejects))\\\textit{Authors refered to: }Huber, Story, Voet.
        \end{small}\\
        \rule{\textwidth}{0.5pt}
        

        \begin{small}
        \begin{center}
        \href{https://heinonline.org/HOL/P?h=hein.engrep/engre0059&i=335}{\textit{Bent v Young} (1838) 9 Sim 180, 59 ER 327} \label{97} \\ 
\textit{Chancery Practice---Jurisdiction (Litigation)}\\
        \end{center}
        \textbf{Ch}.  \textbf{Uses terms: }[\textit{lex loci rei sitae}]. That discovery could not be compelled in Chancery in aid of a suit in a foreign court.\\\textit{Cited in: }Westlake Ed1 (arts 123-126, in connection to the general chancery practice in England.)\\No known authors cited.
        \end{small}\\
        \rule{\textwidth}{0.5pt}
        

        \begin{small}
        \begin{center}
        \href{https://heinonline.org/HOL/P?h=hein.engrep/engrb0012&i=1024}{\textit{Campbell v Dent} (1838) 2 Moo PC 292, 12 ER 1016} \label{102} \\ 
\textit{Contract---Application of Lex Loci Con---Application of Lex Loci Sol---Contract Discharge (Mortgage)}\\
        \end{center}
        \textbf{PC}. Various points related to mortgages made on lands in Demerara, the contracts themselves being governed by the law of Scotland. They seem to have been made and to be performed in Scotland, so it is hard to come to a definitive point on the choice of law issue. This also seems to extend to the undoing of a contract.\\\textit{Cited in: }Westlake Ed1 (art 229, in connection to the discharge of obligations)\\No known authors cited.
        \end{small}\\
        \rule{\textwidth}{0.5pt}
        

        \begin{small}
        \begin{center}
        \href{https://heinonline.org/HOL/P?h=hein.engrep/engrd0048&i=1025}{\textit{Dibbs v Goren} (1839) 1 Beav 457, 48 ER 1017} \label{95} \\ 
\textit{ (Litigation)}\\
        \end{center}
        \textbf{Ch}. A very short decision, simply on the issue of what proof needed to be provided of a particular party being outside of jurisdiction.\\\textit{Cited in: }Westlake Ed1 (arts 123-126, for Chancery practice on (not) assuming jurisdiction of parties out of jurisdiction)\\No known authors cited.
        \end{small}\\
        \rule{\textwidth}{0.5pt}
        

        \begin{small}
        \begin{center}
        \href{https://heinonline.org/HOL/P?h=hein.engrep/engrd0048&i=971}{\textit{Bunbury v Bunbury} (1839) 1 Beav 318, 48 ER 963} \label{98} \\ 
\textit{Injunction---Property ()}\\
        \end{center}
        \textbf{Ch}.  \textbf{Uses terms: }[\textit{lex loci contractus}]. An injunction granted to restrain proceedings to recover real estate in Demerara, apparently on the basis that several items of litigation were better dealt with together in England. Litigation appears to have arisen out of a marriage settlement and will by English domiciled parties, including both realty and personality. Master of the rolls expressly also notes the general position as regards movable (determined by lex domicili) and immovable (determined by lex situs) property.\\\textit{Cited in: }Westlake Ed1 (arts 130-131, as illustrative of the principles governing the issue of anti-suit injunctions)\\No known authors cited.
        \end{small}\\
        \rule{\textwidth}{0.5pt}
        

        \begin{small}
        \begin{center}
        \href{https://heinonline.org/HOL/P?h=hein.engrep/engrf0113&i=1007}{\textit{Abrahams v Skinner} (1840) 12 A\&E 763, 113 ER 1003} \label{83} \\ 
\textit{Bill of Exchange---Evidence (Bill of Exchange)}\\
        \end{center}
        \textbf{KB}. The case is actually rather unusal: it relates to the effects of a change in the stamp issued on bills of exchange. The conflicts dimension is not obvious, though comment is made on Sanith v Mingay. The main point seems to be that the drawing and acceptance of a bill are quite different things.\\\textit{Cited in: }Westlake Ed1 (arts 180-183, cited with approval to suggest that the place of drawing should govern the formalities of a bill.)\\No known authors cited.
        \end{small}\\
        \rule{\textwidth}{0.5pt}
        

        \begin{small}
        \begin{center}
        \href{https://heinonline.org/HOL/P?h=hein.engrep/engre0059&i=904}{\textit{Heriz v Riera} (1840) 1 Sim 318, 59 ER 896} \label{87} \\ 
\textit{Illegality---Contract Validity (Sale)}\\
        \end{center}
        \textbf{VC}. Not very strongly reasoned. The parties seem to have entered into a contract which was illegal and “null and void” by the laws of Spain (the contract was made by someone who was an officer of the Spanish government and stood to benefit from sale to them). The action seems to have been allowed, though the exact reason for doing so doesn’t seem all that clear.\\\textit{Cited in: }Westlake Ed1 (at arts 192-200, largely critically, to suggest the lack of development on the law of illegality in contracts)\\No known authors cited.
        \end{small}\\
        \rule{\textwidth}{0.5pt}
        

        \begin{small}
        \begin{center}
        \href{https://heinonline.org/HOL/P?h=hein.engrep/engrd0048&i=1197}{\textit{Cooper v Waldegrave} (1840) 2 Beav 282, 48 ER 1189} \label{94} \\ 
\textit{Contract---Bill of Exchange---Application of Lex Loci Sol---Interest (Bill of Exchange)}\\
        \end{center}
        \textbf{Rolls}. English interest applied to non-payment of a bill of exchange in England (where payment was due), though the bill was drawn and accepted in France. The distinction appears to be between remedial issues (according to the law of England) and rights themselves (which are by the law of France). There is a strong preference in favour of the “general rule” favouring the place where the contract is made.\\\textit{Cited in: }Westlake Ed1 (at arts 230-236, favouring the use of interest where payment is due – though unclear as whether a matter of fact or law)\\No known authors cited.
        \end{small}\\
        \rule{\textwidth}{0.5pt}
        

        \begin{small}
        \begin{center}
        \href{https://heinonline.org/HOL/P?h=hein.engrep/engrf0113&i=1049}{\textit{Rothschild v Currie} (1841) 1 QB 43, 113 ER 1045} \label{92} \\ 
\textit{Contract---Bill of Exchange---Application of Lex Loci Sol (Bill of Exchange)}\\
        \end{center}
        \textbf{QB}.  \textbf{Uses terms: }[\textit{lex loci contractus}]. A bill was drawn in England against a French bank, and thereafter indorsed in England and transmitted to France for payment. French requirements for notice in going against indorser applied. Interestingly, though the language of “lex loci contractus” is used by the Court, the actual solution appears to be on the basis of the lex loci solutionis – it refers to the place of payment. Protest etc. are held to be substantive rights under the contract, not mere formalities against which suit can be brought.\\\textit{Cited in: }Westlake Ed1 (arts 225-288, with disapproval)\\No known authors cited.
        \end{small}\\
        \rule{\textwidth}{0.5pt}
        

        \begin{small}
        \begin{center}
        \href{https://heinonline.org/HOL/Phttps://heinonline.org/HOL/Page?handle=hein.engrep/engrc0041&id=595?h=hein.engrep/engrc0041&i=595}{\textit{Quarrier v Colston} (1842) 1 Ph 147, 41 ER 587} \label{88} \\ 
\textit{Contract---Contract Validity---Application of Lex Loci Con (Debts---Gambling)}\\
        \end{center}
        \textbf{Ch}. Debts owed from a debt contracted for in Germany in connection to gambling were recoverable in England. Though the reasoning of the case is mostly in terms of the presumption that the games played “at public tables” were lawful.\\\textit{Cited in: }Westlake Ed1 (at arts 192-200, with approval, to suggest the place of contracting as determinative for the sufficiency and validity of consideration.)\\No known authors cited.
        \end{small}\\
        \rule{\textwidth}{0.5pt}
        

        \begin{small}
        \begin{center}
        \href{https://heinonline.org/HOL/P?h=hein.engrep/engre0066&i=1129}{\textit{Hughes v Eades} (1842) 1 Hare 486, 66 ER 1123} \label{96} \\ 
\textit{Chancery Practice---Jurisdiction (Litigation)}\\
        \end{center}
        \textbf{Ch}. Interrogatories issued to show whether a particular defendant was out of jurisdiction.\\\textit{Cited in: }Westlake Ed1 (arts 123-126, for Chancery practice on (not) assuming jurisdiction of parties out of jurisdiction)\\No known authors cited.
        \end{small}\\
        \rule{\textwidth}{0.5pt}
        

        \begin{small}
        \begin{center}
        \href{https://heinonline.org/HOL/P?h=hein.engrep/engri0166&i=595}{\textit{The Vernon} (1842) 1 W Rob 316, 166 ER 591} \label{101} \\ 
\textit{Delict---High Seas---Application of English Statute (Collision---High Seas)}\\
        \end{center}
        \textbf{Admiralty}. A collision on the High Seas left firmly to the determination of the lex fori as set out in an English statute. Don v Lipman cited.\\\textit{Cited in: }Westlake Ed1 (arts 149-158, cited in connection to looking to laws common to the defendants for the law of obligations)\\No known authors cited.
        \end{small}\\
        \rule{\textwidth}{0.5pt}
        

        \begin{small}
        \begin{center}
        \href{https://heinonline.org/HOL/P?h=hein.engrep/engrh0152&i=899}{\textit{Bartlett v Smith} (1843) 11 M\&W 483, 152 ER 895} \label{82} \\ 
\textit{Evidence (Bill of Exchange)}\\
        \end{center}
        \textbf{Ex}. That a judge ought to have received the evidence as to whether a bill was a foreign or inland one, having been made to look foreign for stamping purposes.\\\textit{Cited in: }Westlake Ed1 (arts 180-183, in connection to evidential rules around bills of exchange and their formalities])\\No known authors cited.
        \end{small}\\
        \rule{\textwidth}{0.5pt}
        

        \begin{small}
        \begin{center}
        \href{https://heinonline.org/HOL/P?h=hein.engrep/engrh0152&i=1065}{\textit{General Steam Navigation Co v Guillou} (1843) 11 Mee \& Wel 877, 152 ER 1061} \label{100} \\ 
\textit{Foreign Judgements (Collision---High Seas)}\\
        \end{center}
        \textbf{Ex}.  \textbf{Uses terms: }[\textit{lex fori}]. A very interesting and well argued case, though the exact points at issue are not all that clear. The case mostly turns on points of pleading. A collision occurred on the High Seas between England and France, the defendants being masters of a French ship that collided with that of the defendants. The claim is brought as one of action on the case. There seems to be some agreement that the claim is governed by French law (though why it is not clear). The two issues related to the grounds on which French courts hold masters of ships and/or companies liable for the actions of their servants (this issue was submitted to the lex fori.) Another issue is an existing judgement given in Havre. This was said to be pleaded badly, not being made by way of estoppel.\\\textit{Cited in: }Westlake Ed1 (arts 149-158, cited in connection to looking to laws common to the defendants for the law of obligations)\\No known authors cited.
        \end{small}\\
        \rule{\textwidth}{0.5pt}
        

        \begin{small}
        \begin{center}
        \href{https://heinonline.org/HOL/P?h=hein.engrep/engrg0135&i=796}{\textit{Steadman v Duhamel} (1845) 1 CB 888, 135 ER 792} \label{81} \\ 
\textit{ (Bill of Exchange)}\\
        \end{center}
        \textbf{CP}. Not a very interesting case, really just about the evidential requirements for showing that a bill was “foreign” or “English.” Seems to have been some active fraud in making the bill appear  French so as to get around the Stamping requirements.\\\textit{Cited in: }Westlake Ed1 (arts 180-183, in connection to evidential rules around bills of exchange and their formalities])\\No known authors cited.
        \end{small}\\
        \rule{\textwidth}{0.5pt}
        

        \begin{small}
        \begin{center}
        \href{https://heinonline.org/HOL/P?h=hein.engrep/engrg0136&i=58}{\textit{Benham v Mornington} (1846) 3 CB 133, 136 ER 54} \label{85} \\ 
\textit{Contract---Contract Formality---Application of Lex Loci Con (Bond)}\\
        \end{center}
        \textbf{CP}. Quite interesting. A penal bond was entered into in France, though (it was alleged) without the required formality requirements under the French code. It was held that specific evidence needed to be shown of the relevant foreign law, but the clear implication is that proof of this would have been sufficient.\\\textit{Cited in: }Westlake Ed1 (arts 173-176, cited with general approval for the locus regit actum rule and formalities, and that the domicile of the parties should not matter.)\\No known authors cited.
        \end{small}\\
        \rule{\textwidth}{0.5pt}
        

        \begin{small}
        \begin{center}
        \href{https://heinonline.org/HOL/P?h=hein.engrep/engrb0013&i=710}{\textit{Allen v Kemble} (1848) 6 Moo PC 321, 13 ER 704} \label{93} \\ 
\textit{Contract---Bill of Exchange---Application of Lex Loci Sol---Application of Lex Loci Con ()}\\
        \end{center}
        \textbf{PC}.  \textbf{Uses terms: }[\textit{lex loci solutionis, lex loci contractus}]. Expressly favouring the lex loci contractus for a bill of exchange (Demerara, where it was drawn) over the lex loci solutionis (London, where it was to be paid) for determining liabilities under a bill of exchange. This meant that the Roman-Dutch law prevailing in Demerara governed the issues of set-off by the acceptor.\\\textit{Cited in: }Westlake Ed1 (arts 225-288, with approval)\\No known authors cited.
        \end{small}\\
        \rule{\textwidth}{0.5pt}
        

        \begin{small}
        \begin{center}
        \href{https://heinonline.org/HOL/P?h=hein.engrep/engrc0041&i=1161}{\textit{Sharp v Taylor} (1849) 2 Ph 801, 41 ER 1153} \label{91} \\ 
\textit{Contract---Illegality (Charter)}\\
        \end{center}
        \textbf{Ch}. Important point appears to be the distinction between enforcing an illegal contract and asserting a title that has arisen under it. Profits were made between two partners in a joint venture involving a fradulently registered ship. It was held that profits could be recovered by one of the partners nothwithstanding any illegality.  There are also some remarks that the relevant issue seems to be that the laws violated in America were revenue laws only.\\\textit{Cited in: }Westlake Ed1 (arts 201-202, criticised, notwithstanding said to represent the current law, for the treatment of foreign revenue laws (that English courts can allow their violation).)\\No known authors cited.
        \end{small}\\
        \rule{\textwidth}{0.5pt}
        

        \begin{small}
        \begin{center}
        \href{https://heinonline.org/HOL/P?h=hein.engrep/engrh0155&i=122}{\textit{Bristow v Sequeville} (1850) 5 Exch 279, 155 ER 118} \label{106} \\ 
\textit{Contract---Contract Formality---Application of Lex Fori (Companies---Sale)}\\
        \end{center}
        \textbf{Ex}. Extremely interesting decision. The first noteworthy point is the distinction between formalities that make a contract void in a state (which are said to apply by the law applicable to the contract) and those which merely relate to evidence (which was found to be the case here, leaving the issues applicable to the lex fori. The other noteworthy point is the very extensive argumentation on a foreign witness. The relevant issues related to the French Civil Code as then in Force in Köln. The witness had studied in Leipzig, but put himself out as having read the French law. This was insufficent (study at a university was not enough.)\\\textit{Cited in: }Westlake Ed1 (at arts 173-176 and 177, in support of the role of foreign stamps and formalities and solemnities, with approval.)\\No known authors cited.
        \end{small}\\
        \rule{\textwidth}{0.5pt}
        

        \begin{small}
        \begin{center}
        \href{https://heinonline.org/HOL/P?h=hein.engrep/engre0068&i=577}{\textit{Caldwell v Vanvlissengen} (1851) 9 Hare 415, 68 ER 571} \label{114} \\ 
\textit{Delict (Patents)}\\
        \end{center}
        \textbf{VC}. Injunction issued against Dutch defendants to prevent them violating the terms of an English patent on some invention. The reasoning is quite rich with ideas, but not really firm conclusions. There are some general illusions to laws applying by virtue of someone’s nationality, but the firm conclusion is that the rights in question can be protected.\\\textit{Cited in: }Westlake Ed1 (art 237-240, in connection to foreigners being held liable for English delicts.)\\\textit{Authors refered to: }Huber, Vattel, Story, Boullenois.
        \end{small}\\
        \rule{\textwidth}{0.5pt}
        

        \begin{small}
        \begin{center}
        \href{https://heinonline.org/HOL/P?h=hein.engrep/engrg0138&i=1123}{\textit{Leroux v Brown} (1852) 12 CB 801, 138 ER 1119} \label{103} \\ 
\textit{Contract---Contract Formality---Application of English Statute (Statute of Frauds)}\\
        \end{center}
        \textbf{CP}. Applying the fourth section of the statute of frauds (contracts to be carried out after 1 year must be in writing) to a parole contract made in France. Though French law was said to govern the contract, the statute of frauds acted as a rule of evidence.\\\textit{Cited in: }Westlake Ed1 (cited, apparently with approval, at arts 171-172, in connection to the rules on formality)\\\textit{Authors refered to: }Huber, Story, Burge.
        \end{small}\\
        \rule{\textwidth}{0.5pt}
        

        \begin{small}
        \begin{center}
        \href{https://heinonline.org/HOL/P?h=hein.engrep/engre0061&i=557}{\textit{Moodie v Bannister} (1853) 1 Drew 514, 61 ER 549} \label{108} \\ 
\textit{Chancery Practice---Jurisdiction (Litigation)}\\
        \end{center}
        \textbf{VC}. Not terribly interesting, only for certain remarks on the ability to serve certain defendants out of jurisdiction in Scotland.\\\textit{Cited in: }Westlake Ed1 (arts 127, in connection to the general chancery practice in England.)\\No known authors cited.
        \end{small}\\
        \rule{\textwidth}{0.5pt}
        

        \begin{small}
        \begin{center}
        \href{https://heinonline.org/HOL/P?h=hein.engrep/engrh0156&i=15}{\textit{Gibbs v Fremont} (1853) 9 Exch 25, 156 ER 11} \label{113} \\ 
\textit{Contract---Application of Lex Loci Con---Contract Intention (Bill of Exchange)}\\
        \end{center}
        \textbf{Ex}.  \textbf{Uses terms: }[\textit{lex loci contractus, lex loci solutionis}]. Bill of Exchange drawn in California but payable in Washington DC. California rate of interest applied, on the express approval of the lex loci contractus. Expressly noted however that this was a principle that only applied absent of express indication. Interest expressly held to be a question of law not fact.\\\textit{Cited in: }Westlake Ed1 (arts 230-236, for the application of ex mora interest in bills of exchange.)\\No known authors cited.
        \end{small}\\
        \rule{\textwidth}{0.5pt}
        

        \begin{small}
        \begin{center}
        \href{https://heinonline.org/HOL/P?h=hein.engrep/engri0164&i=329}{\textit{The Johannes Christoph} (1854) 2 Sp Ecc \& Ad 93, 164 ER 325} \label{104} \\ 
\textit{Property---Application of Lex Fori (Lien---Freight)}\\
        \end{center}
        \textbf{Admiralty}.  \textbf{Uses terms: }[\textit{lex loci contractus, lex fori}]. Freight was sold of a ship, which had a master and crew from Hamburg. This was in satisfaction of a number of claims, including salvage. The master claimed to be entitled to proceeds of the sale by virtue of a lien under Hamburg law for his wages and sundry other costs. This was not admitted. Dr Lushington, interestingly, phrases the position “in this court” (Admiralty) as being that foreign law is imported as a matter of discretion as regards the remedy in an action. Huber is cited in support of the proposition. Perhaps this relates to the particular understanding of the court of admiralty at this time?\\\\\textit{Authors refered to: }Huber.
        \end{small}\\
        \rule{\textwidth}{0.5pt}
        

        \begin{small}
        \begin{center}
        \href{https://heinonline.org/HOL/P?h=hein.engrep/engra0010&i=973}{\textit{Carron Iron Company v Maclaren} (1855) 5 HLC 416, 10 ER 961} \label{109} \\ 
\textit{Injunction (Litigation)}\\
        \end{center}
        \textbf{HL}. An injunction granted to restrain proceedings in Scotland. Sufficient basis found in the fact that the defendant had lands in England, and there was “sufficient Equity” in doing so, including the fact that litigation was pending in England.\\\textit{Cited in: }Westlake Ed1 (arts 127, in connection to the general chancery practice in England.)\\No known authors cited.
        \end{small}\\
        \rule{\textwidth}{0.5pt}
        

        \begin{small}
        \begin{center}
        \href{https://heinonline.org/HOL/P?h=hein.engrep/engrd0052&i=782}{\textit{Sudlow v Dutch Rhenish Railway Company} (1855) 21 Beav 43, 52 ER 774} \label{112} \\ 
\textit{Contract---Lex Loci Con (Companies---Shares)}\\
        \end{center}
        \textbf{Rolls}. Claim for relief against forefeiture of shares in a Dutch company; rejected, on the basis that “this is a Dutch contract”, there also being evidence of a similar case being tried before the Dutch courts. Not clear on what terms the choice of law was decided for?\\\textit{Cited in: }Westlake Ed1 (at arts 230-236, as to the law of the contract itself defining the manner and extent of performance)\\No known authors cited.
        \end{small}\\
        \rule{\textwidth}{0.5pt}
        

        \begin{small}
        \begin{center}
        \href{https://heinonline.org/HOL/P?h=hein.engrep/engrd0052&i=1151}{\textit{Hope v Hope} (1856) 2 Beav 351, 52 ER 1143} \label{90} \\ 
\textit{Contract---Illegality---Application of Lex Loci Con---Contract Intention (Marriage)}\\
        \end{center}
        \textbf{Rolls}.  \textbf{Uses terms: }[\textit{lex loci contractus}]. Actual decision seems to imply that the issues simply good not be decided on demurrer – though subsequent proceedings are apparently found elsewhere. The relevant foreign law related to certain French provisions related to marriage, and custodial rights of children. Important observations on the lex loci contractus rule, clearly seeing that it did not extent to a situation where the contract was to be performed in another place.\\\textit{Cited in: }Westlake Ed1 (arts 192-200, cited as an instance whereby a foreign law will not be enforced in English courts on moral grounds)\\No known authors cited.
        \end{small}\\
        \rule{\textwidth}{0.5pt}
        

        \begin{small}
        \begin{center}
        \href{https://heinonline.org/HOL/P?h=hein.engrep/engri0166&i=1042}{\textit{The Zollverein} (1856) Swabey 96, 166 ER 1038} \label{110} \\ 
\textit{High Seas---Delict---Application of English Statute (High Seas---Collision)}\\
        \end{center}
        \textbf{Admiralty}. Dr Lushington declining to apply the provisions of the Merchant Marine Act in a claim brought by a foreign vessel against a British one. Most of the case seems to turn on the application and meaning of the statute itself. Dr Lushington’s residual position is that “the law maritime” applies.\\\textit{Cited in: }Westlake Ed1 (arts 148-152, used to suggest there are cases where the law “common to the parties” is used, though purpose of citation is unclear.)\\No known authors cited.
        \end{small}\\
        \rule{\textwidth}{0.5pt}
        

        \begin{small}
        \begin{center}
        \href{https://heinonline.org/HOL/P?h=hein.engrep/engri0166&i=1025}{\textit{The Dumfries} (1856) Swabey 63, 166 ER 1021} \label{111} \\ 
\textit{Law Maritime---High Seas---Delict---Application of English Statute (High Seas---Collision)}\\
        \end{center}
        \textbf{Admiralty}. Dr Lushington really repeats the observations made in The Zollverein, that for a collision on the High Seas between a British and Foreign vessel, the law maritime is to apply.\\\\No known authors cited.
        \end{small}\\
        \rule{\textwidth}{0.5pt}
        

        \begin{small}
        \begin{center}
        \href{https://heinonline.org/HOL/P?h=hein.engrep/engrh0157&i=28}{\textit{Sharples v Rickard} (1857) 2 H\&N 57, 157 ER 24} \label{107} \\ 
\textit{Contract---Contract Formality---Application of English Statute (Bill of Exchange)}\\
        \end{center}
        \textbf{Ex}. Not very interesting. Essentially just illustrating the statutory position that foreign bills do not require a British stamp for their presentation.\\\textit{Cited in: }Westlake Ed1 (arts 180-183, to illustrate the statutory position regarding stamps on foreign bills.)\\No known authors cited.
        \end{small}\\
        \rule{\textwidth}{0.5pt}
        

        \begin{small}
        \begin{center}
        \href{https://heinonline.org/HOL/Page?lname=&public=false&collection=engrep&handle=hein.engrep/engre0065&men_hide=false&men_tab=toc&kind=&page=746}{\textit{Brook v Brook} (1858) 3 Sm \& Grif 481, 65 ER 746} \label{105} \\ 
\textit{Marriage---Contract---Contract Validity---Application of Lex Loci Con---Application of English Statute---Illegality (Marriage)}\\
        \end{center}
        \textbf{VC}. A marriage was entered into in Denmark between a widower and late wife’s sister. Such marriages are outlawed in England, but valid by the laws of Denmark. Man and woman had no domicile in Denmark. It was held that the marriage was invalid in English courts. The reasoning remarks specifically on the principle of lex loci contractus (and is very fully argued) citations to many jurists are included. The exact principle either seems to be on the application of the English statute or not admitting foreign law on principles of illegitimacy or immorality. Later appealed to House of Lords an upheld.\\\\\textit{Authors refered to: }Huber, Story, Voet, Sanchez, Gayll.
        \end{small}\\
        \rule{\textwidth}{0.5pt}
        

        \begin{small}
        \begin{center}
        \href{https://heinonline.org/HOL/P?h=hein.engrep/engra0011&i=713}{\textit{Brook v Brook (no 2)} (1861) 9 HLC 193, 11 ER 703} \label{115} \\ 
\textit{Marriage---Contract---Contract Validity---Application of Lex Loci Con---Application of English Statute---Illegality (Marriage)}\\
        \end{center}
        \textbf{HL}.  \textbf{Uses terms: }[\textit{lex loci contractus, lex domicilii}]. Continuation of case below. A marriage was entered into in Denmark between a widower and late wife’s sister. Such marriages are outlawed in England, but valid by the laws of Denmark. Man and woman had no domicile in Denmark. It was held that the marriage was invalid in English courts. The reasoning is slightly modified. There is a more clear line drawn between the lex loci contractus (governing the forms of marriage) and the lex domicilii (governing its requirements and aspects). In particular also, a policy of preventing evasion of English public policy is put forward.\\\\\textit{Authors refered to: }Huber, Story.
        \end{small}\\
        \rule{\textwidth}{0.5pt}
        

        \begin{small}
        \begin{center}
        \href{https://heinonline.org/HOL/P?h=hein.engrep/engrf0121&i=1267}{\textit{MacFarlane v Norris} (1862) 2 B\&S 783, 121 ER 1263} \label{116} \\ 
\textit{Contract---Contract Discharge---Application of Lex Fori (Bankruptcy)}\\
        \end{center}
        \textbf{QB}.  \textbf{Uses terms: }[\textit{lex fori, lex loci contractus}]. A Scotch trustee in bankruptcy (sequestration) is suing a debtor in England. The issue is as to the set-off of debts between them. All three judges seem eager to apply the Scotch law, though their actual reasoning is rather slim. It seems to be conceded that set-off is a matter for the lex fori, but there is severe reluctance to not give the Scottish trustee the rights he would have had in Scotland.\\\\\textit{Authors refered to: }Huber, Story.
        \end{small}\\
        \rule{\textwidth}{0.5pt}
        

        \begin{small}
        \begin{center}
        \href{https://heinonline.org/HOL/P?h=hein.engrep/engrf0121&i=982}{\textit{Scott v Pilkington} (1862) 2 B\&S 11, 121 ER 978} \label{118} \\ 
\textit{Contract---Application of Lex Loci Con---Foreign Judgements (Bill of Exchange)}\\
        \end{center}
        \textbf{QB}.  \textbf{Uses terms: }[\textit{lex loci contractus, comity}]. Bill of Exchange drawn in New York but to be paid in London. Judgement obtained by the Law of New York (said to be erroneous on New York Law, and wrong to have applied New York Law.) (1) That a court would not examine a foreign judgement for being errenous, even on its own law. (2) That an appeal being pending might be a basis to say proceedings but would not bar them. (3) That, potentially, a failure to apply the correct law would violate the “comity of nations” and thereby lead the decision not to be enforced. (4) That a bill of exchange drawn in New York, but payable in London had to be governed by the lex loci contractus.\\\\No known authors cited.
        \end{small}\\
        \rule{\textwidth}{0.5pt}
        

        \begin{small}
        \begin{center}
        \href{https://heinonline.org/HOL/P?h=hein.engrep/engre0071&i=91}{\textit{Simpson v Fogo} (1863) 1 H\&M 195, 71 ER 85} \label{117} \\ 
\textit{Property---Property Protection of Third Parties---Application of Lex Fori---Foreign Judgements (Mortgage)}\\
        \end{center}
        \textbf{VC}.  \textbf{Uses terms: }[\textit{lex fori, lex loci contractus, lex domicilii, lex rei sitae}]. A very full decision, and worth re-reading. A ship was mortgaged in Liverpool, and then taken to New Orleans. Another creditor of the ship owner, in effect, compelled her compulsory sale. Proceedings then commenced in Louisiana as to the relative interests of the different creditors. The Mortagees (the Bank of Liverpool) made representations before the court there, but they lost rather badly: the effect of the Louisiana court’s ruling was that their interest in the ship was extinguished. These proceedings started when the new owners took the ship back to Liverpool. The Louisiana Court’s ruling was effectively disregarded, leaving the Bank of Liverpool’s interests intact – the reasoning being that the ruling was contrary to the law of nations and should not be recognised. Some interesting point that emerge include a debate as to whether the lex loci contractus, lex domicili, or lex rei sitae applies to the determination of title to movable property, as well as the application of the lex fori to determine questions of priority.\\\\\textit{Authors refered to: }Huber, Story, Burge.
        \end{small}\\
        \rule{\textwidth}{0.5pt}
        

        \begin{small}
        \begin{center}
        \href{None}{\textit{In re Melbourn (no 1)} (1871) LR Ch App 64} \label{119} \\ 
\textit{Contract---Marriage---Application of Lex Loci Con---Application of Lex Fori---Bankruptcy (Marriage---Bankruptcy)}\\
        \end{center}
        \textbf{CainCh}.  \textbf{Uses terms: }[\textit{lex loci contractus, lex fori}]. Marriage in Batavia with marriage contract, excluding community of property. That marriage contract was not registered, which was required by the law of Batavia. Since this was a matter of proof (rather than validity) it did not affect the inter se admissability of the claim, but could not affect the distribution of assets amongst the creditors, which was submitted to the lex fori. Strong preference for the lex loci contractus as the effect of the marriage contract.\\\\No known authors cited.
        \end{small}\\
        \rule{\textwidth}{0.5pt}
        
\end{document}